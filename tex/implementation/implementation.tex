\chapter{ Implementation }

In this chapter we will present the implementation of the \PT. 

The analysis of the algorithm in all its complexity called for the development of a more structurised template for carrying out numerical computations. The \PT, being an extension to the \MH, would naturally share many similarities in implementation when compared to the \PT. Both algorithms can have very abstract formulations and can be carried out theoretically on any countably-generated measurable space. Practical considerations will never go that far. However, one cannot, in principle, discard problems like bayesian linear model selection or some sort of analysis regarding calculations on lattices in the Ising model. Because of the mulitude of potential applications, it was vital that the implemention of the \PT was therefore a very flexible one. To attain this goal, the object-oriented paradigm was applied with the separation of the State Space and the Algorithm as two distinct entities as the possible solution to the uppermentioned problem. 

Having realised this fact, it seemed natural to generalise the whole idea one step further and develop a general template for Metropolis-Hastings-like computations, rather than a computer programme able to carry out only two sorts of specific tasks. For there are other potential developments of the \MH\, that are being used. Similarities in stuctures of these algorithms called for construction of one organised, yet modular, a structure. Our template's goal was therefore to provide basic building blocks used for simulations and leave user concentrate on analysis. To guarantee user-friendliness, the most common State Space, the multidimensional euclidean space, has been provided, so that the user could in principle carry out computations specifying only the required minimum - the density function of the measure of interest. With future releases of the software, more standard State Spaces are scheduled for implemention as well, leaving the users experience the potentials and drawbacks of stochastic simulations.