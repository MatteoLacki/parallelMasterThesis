\chapter{ Simulations and Results }\label{simulationsAndResults}

Recall that Liang-Wang example of a multimodial function:

\begin{equation*}
f(x) = 
\sum_{i=1}^{20} \frac{\omega_i}{ \sigma_i \sqrt{2 \pi} } \exp \Big( -\frac{(x - \mu_i)^\tran (x - \mu_i)}{2 \sigma_i^2} \Big),	
\end{equation*}
where $\sigma_1 = \dots = \sigma_{20} = 0.1$, $\omega_1 = \dots = \omega_{20} = 0.05 $ and the means $\mu_i$ are enlisted in Chapter \ref{motivation}.


Another example is inspired by paper by \cite{BaragattiLikelihoodFreeParallelTempering}. The goal is to test different swapping strategies in finding also less accentuated modes of probability. For that reason, the following mixture of normal distributions was considered

\begin{equation*}
f(x) = 
 	\frac{1}{10} \frac{1}{2 \pi \sigma_1^2} \exp \Big( -\frac{ ||x - \mu_1||^2}{2 \sigma_1^2} \Big) +
 	\frac{9}{10} \frac{1}{2 \pi \sigma_2^2} \exp \Big( -\frac{ ||x - \mu_2||^2}{2 \sigma_2^2} \Big)
 	,	
\end{equation*}
where $\sigma_1 = 0.7$ and $\sigma_2 = 0.05$ and the means are 

\begin{table}[ht]
	\centering
	\begin{tabular}{rr}
	  	\hline
			$\mu_1$ & $\mu_2$ \\ 
	  	\hline
			2 & 8 \\ 
			2 & 8 \\ 
	   	\hline
	\end{tabular}
\end{table}