%!TEX program = <xelatex>
\documentclass{book}
\usepackage{mystyle}
\begin{document}

\chapter*{Summary and Future Research Directions}
\addcontentsline{toc}{chapter}{Summary}

As seen in the Chapter \ref{motivation} the standard Metropolis-Hastings algorithm has intrinsic limitations when it comes to solving the \ref{Problem} of drawing sample points from multimodal distributions. The \PT\, serves as a potential solution to that \ref{Problem}, being at the same time quite a flexible approach. 

In this work we have envisaged several Swapping Strategies. These Strategies are nothing else but laws according to which the \PT\, travels through the \sspace. \ref{strat1} served together with \ref{strat5} and \ref{strat6} as reference points for the remaining three strategies, being respectively a strategy based on the idea of \textsc{PTEEM} devised by \	


\end{document}

