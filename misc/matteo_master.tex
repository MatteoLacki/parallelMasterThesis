documentclass{pracamgr}
\usepackage{natbib}
\usepackage{amssymb} %Odpowiadają za narzędzia matematyczne
\usepackage{amsmath} %Odpowiadają za narzędzia matematyczne
\usepackage{amsthm}  %Odpowiadają za narzędzia matematyczne
\usepackage[pdftex]{graphicx}
\usepackage[usenames,dvipsnames]{color} % Nie pamiętam co to robi
\usepackage[polish]{babel} % To pakiet odpowiadający za języki -- likwidujesz rosyjski
\usepackage[utf8]{inputenc}
%\usepackage[cp1250]{inputenc} 
\usepackage[T1]{fontenc}
\usepackage{color,hyperref} % To związane było z kolorowaniem
\usepackage{graphicx, subfig} % Dzięki temu mogę umieszczać kilka ilustracji w jednym miejscu dzieląc je na ilustrację a) i b) i ...
\usepackage{exscale,relsize}
\usepackage{pdfpages}

\definecolor{darkblue}{rgb}{0.0,0.0,0.3}
\hypersetup{colorlinks,breaklinks,
           linkcolor=Mahogany,urlcolor=Mahogany,
           anchorcolor=Mahogany,citecolor=Mahogany}
\hypersetup{colorlinks,breaklinks,
           linkcolor=black,urlcolor=black,
            anchorcolor=black,citecolor=black}

\label{srodowiska}
\newtheorem{tw}{Twierdzenie} 
\newtheorem{definicja}{Definicja}
\newtheorem{wn}{Wniosek}
\newtheorem{fact}{Fakt}
\newtheorem{lemat}{Lemat}
\newtheorem{ass}{Za{\l}o{\.z}enie}
\newtheorem{dowod}{Dowód}
\newtheorem{wniosek}{Wniosek}

\newcommand{\diff}[1]{\mathrm{d}\,#1}
\newcommand{\diffover}[2]{\frac{\mathrm{d}}{\mathrm{d}\,#1} #2}
\newcommand{\partialdiff}[2]{\frac{\partial}{\partial #1} #2}
\newcommand{\pardiff}[2]{\frac{\partial #1}{\partial #2}}
	
\newcommand{\brackets}[1]{[#1]}	
\newcommand{\bracketss}[2]{[#1,#2]}
\newcommand{\rbracket}{]}	

\linespread{1.3}
% Dane magistranta:

\author{Mateusz Łącki}
\nralbumu{39885}
\title{Gładkość operatorów w modelu gospodarki z podatkami}	
%\tytulang{}
\kierunek{Metody Ilościowe w~Ekonomii i~Systemy~Informacyjne}
%\zakres{Ekonomia matematyczna}

\opiekun{{dra Łukasza Woźnego}\\
Katedra Teorii Systemu Rynkowego\\
  }

\date{Październik 2011}

\keywords{gospodarki z dystorsjami, analiza funkcjonalna w ekonomii, metody homotopii}


\oddsidemargin +0.4in
\evensidemargin -0.4in

%%%%%%%%%%%%%%%%%%%%%%%%%%%%%%%%%%%%%%%%%%%%%%%%%%%%%%%%%%%%%%%%%%%%%%%%%%%%%%%%%%%%%%%%%%%%%%%%%%%%%%%%%%%%%%%%%%

	\begin{document}
%\maketitle
\renewcommand\bibname{Spis literatury}

\includepdf{titlepage.pdf}
\begin{small}
\tableofcontents
\end{small}

\chapter*{Wprowadzenie}
\addcontentsline{toc}{chapter}{Wprowadzenie }

	Celem niniejszej pracy jest podjęcie próby zbadania różniczkowalności abstrakcyjnych operatorów matematycznych o zasadniczym znaczeniu przy przeprowadzaniu obliczeń oraz jakościowej analizy rozwiązań problemu, który pojawia się w naturalny sposób przy badaniu pojęcia równowagi rekursywnej w stochastycznym modelu gospodarki nieskończenie okresowej z podatkami progresywnymi.	
	
	Badanie tego rodzaju zagadnienia jest ważne dla przeprowadzenia poprawnych wnioskowań ekonomicznych. Jak wskazują \citet*{Reffett_2008}, wprowadzony przez Brocka i Mirmana (\citeyear{Brock_1972}) stochastyczny model optymalnego wzrostu stał się jednym z ważniejszych narzędzi współczesnej makroekonomii. \citet{Reffett_2008} zauważają, że {\it dostarczył on ekonomistom ogólnej struktury umożliwiającej wspólne badanie zagadnień tak różnych, jak cykle gospodarcze, wzrost w równowadze, wycena aktywów, niezupełne rynki finansowe, nierówności majątkowe, altruizm w gospodarkach wzrostu, makroekonomiczne modele wzrostu ekonomii gałęziowej, akumulacja kapitału fizycznego i ludzkiego oraz wpływu aktywnej polityki fiskalnej i monetarnej}. Potrzeba modelowania coraz bardziej skomplikowanych zjawisk ekonomicznych sprawiła, że coraz częściej badane modele nie spełniały założeń Twierdzeń Dobrobytu, zobacz: \citet*[][str. 451]{Prescott}. Oznaczać to może, że alokacja w równowadze może nie być efektywna w sensie Pareto. Tym bardziej więc zachodzi potrzeba dokładniejszego jej zbadania. W tym celu wielu ekonomistów uznało za stosowne wykorzystanie metod obliczeniowych w celu uzyskania numerycznego przybliżania interesujących ich obiektów, niosących część informacji na temat właności równowag. 
	
	Jednym z klasycznych modeli tego rodzaju jest model z reprezentatywnym gospodarstwem domowym zaproponowany przez Colemana (\citeyear{Coleman1}) i stopniowo rozwijany w pracach z lat \citeyear{Coleman_1997} oraz \citeyear{Coleman2}. Jest to stochastyczny model gospodarki nieskończenie okresowej z kapitałem~\footnote{Praca z roku \citeyear{Coleman_1997} rozwija model o elastyczną podaż pracy.}, w której występują podatki~\footnote{Ang. {\it distortionary taxes}.} lub efekty zewnętrzne. Model ów można uznać za podstawowe narzędzie ekonomicznej analizy gospodarki, w której efektywność systemu ekonomicznego zostaje zaburzona poprzez wprowadzenie podatków. \citet{Coleman1} zajmuje się kwestią istnienia rekursywnej równowagi w tym modelu, która z założenia ma być ciągłą funkcją. Funkcja ta może być interpretowana jako pewna reguła postępowania reprezentatywnego gospodarstwa domowego. Owa reguła działania musi być wspólna dla wszystkich agentów występujących w gospodarce i działać wtedy, gdy zachodzi pełne zrównoważenie budżetu: transfery pieniężne do gospodarstwa zrównują się z podatkami przez nie płaconymi. Aby znaleźć ową regułę, Coleman rozpatruje operator na abstrakcyjnych przestrzeniach funkcyjnych. Punkt stały tego operatora jest właśnie poszukiwaną regułą.    
	
	W pracy z \citeyear{Coleman1} Coleman potwierdza istnienie równowagi rekursywnej dla preferencji o szczególnej postaci. W pracy z z \citeyear{Coleman2} roku znacząco uogólnia on warunki na jednoznaczność rozpatrywanego rozwiązania. Jako że dowody jego twierdzeń są konstruktywne, Coleman był w stanie zaproponować szkic algorytmu, dzięki któremu równowagę można by obliczyć posługując się komputerem.	
	
	Pomimo uzyskania bardzo ciekawych wyników dotyczących zbieżności opracowanych algorytmów w sensie normy supremum, stosowane tam narzędzia nie pozwolają wnioskować np. o tempie zbieżności rozwiązań. Określenie teoretycznego tempa zbieżności algorytmów jest ważne, bowiem umożliwia kontrolę błędu przybliżenia numerycznego, co jest ważne dla orzeczenia ostatecznej jego jakości. Do określenia teoretycznego tempa zbieżności wielu algorytmów bardzo często wykorzystuje się różniczkowalność funkcji będących obiektem zainteresowania badacza. Tym samym zbadanie różniczkowalności występujących w pracy operatorów mogłoby dostarczyć dodatkowej informacji o jakości zaproponowanych w pracach Colemana (\citeyear{Coleman1, Coleman2}) algorytmów.     


	Współczesna ekonomia nie może się obejść bez metod numerycznych. Skomplikowane modele ekonomiczne stosowane są w wielu instytucjach publicznych i prywatnych. Ich wykorzystanie służyć ma lepszemu zrozumieniu otaczających nas zjawisk ekonomicznych. Tym samym prowadzenie badań nad stosowanymi algorytmami jest bardzo potrzebne; tym bardziej, gdy badania te mogą dostarczyć informacji o jakości uzyskiwanych na komputerze rozwiązań. Bez prowadzenia skrupulatnej analizy błędów narastających podczas iterowania algorytmu nie możemy być pewni, że uzyskane wyniki prezentują jakąkolwiek wartość badawczą. Zachodzi potrzeba kontrolowania błędów wynikających z natury danego algorytmu oraz uwzględnienia błędów wynikających z rządzącej się swoimi prawami arytmetyki maszynowej. Powszechne użycie metod numerycznych w ekonomii wymaga zatem od ekonomistów zwrócenia uwagi na zagadnienia, które jeszcze do niedawna były domeną inżynierów, informatyków i numeryków. 
	
	
	Pomysł zbadania różniczkowalności operatorów występujących w niniejszej pracy to również część większego zamysłu, jakim było stworzenie analogonu algorytmu homotopijnego korektora-predyktora dla przestrzeni nieskończenie wymiarowych. Pokazanie różniczkowalności operatora mogłoby być wykorzystane bowiem do zbudowania różniczkowalnej homotopii, której przeciwobraz zera odpowiadałby ciągłym funkcjom równowagi. Badanie różniczkowalności operatorów zdefiniowanych na przestrzeniach funkcyjnych znalazło wykorzystanie całkiem niedawno w ekonomii przy okazji badania istnienia i jednoznaczności równowag spójnych czasowo w modelach gospodarek z dyskontowaniem hiperbolicznym \citep[zobacz:][]{Judd_2003}. Tam też wykorzystano metodę zaburzeń, będącą lokalnym odpowiednikiem metod homotopijnych. 

	Metody homotopijne w przestrzeniach skończenie wymiarowych powoli wchodzą w skład warsztatu obliczeniowego ekonomistów --- ich wykorzystanie propagują \citet*{Doraszelski_2008, Herings_2009, Bajari_2010}. Narzędzia, którymi się posługują pojawiły się w matematyce obliczeniowej już na przełomie lat siedemdziesiątych i osiemdziesiątych: rozpoczęto wówczas systematyczne badania nad zastosowaniem tego rodzaju metod do znajdowania rozwiązań systemów równań wielomianowych, patrz: \citet{Garcia_1979, Garcia, Garcia_1980}; rozwijano wtedy również i ogólną teorię, zobacz: \citet{Heijer_1981}. Wedle naszej wiedzy nie poczyniono starań, aby uogólnić spotykane w powyżej cytowanych dziełach algorytmy na przypadek przestrzeni nieskończenie wymiarowej. 
	
	Z drugiej strony: metody homotopijne w przestrzeniach nieskończenie wymiarowych od lat wykorzystywane są przy niekonstruktywnych dowodach istnienia rozwiązań skomplikowanych równań różniczkowych. Metody te korzystają z metod topologii ogólnej, patrz: \citet{Granas_1980,Granas_FPT}, jak również z teorii operatorów Fredholma i teorii analizy nieliniowej, patrz: \citet{Brown_2004}. Szczegóły naszego zamysłu, który przyświecał badaniu różniczkowalności operatorów w tej pracy, podajemy w rozdziale \ref{chap_conclusions}.
	
	Niniejsze badanie, będać ściśle związane z ekonomicznym zagadnieniem istnienia równowagi, ma charakter bardzo techniczny. Tym samym w zakresie metodyki pracy znalazły się: ({\it i}) prezentacja oraz dogłębna analiza dowodów istniejących rozumowań matematycznych stojących za istnieniem i jednoznacznością równowagi, ({\it ii}) przeprowadzenie własnych dowodów uszczegóławiające owe rozumowanie tam, gdzie zachodzą niejednoznaczności, ({\it iii}) stworzenie teorii różniczkowalności operatorów zadanych na funkcjach odwrotnych w oparciu o matematyczną teorię różniczkowalności operatorów Niemyckiego \citep[zobacz:][]{Appell} oraz teorię różniczkowalności operatorów odwracania funkcji \citep[zobacz:][]{Lanza1,Lanza2}.
	
	Dzięki zastosowaniu się do punktów ({\it i}) oraz ({\it ii}) byliśmy w stanie w ważny sposób uzupełnić rozumowanie Colemana (\citeyear{Coleman2}). Zauważyliśmy też, że rozpatrywane w jego pracach z lat \citeyear{Coleman1} oraz \citeyear{Coleman2} operatory są sprzężone w sensie algebraicznym. Wychodząc naprzeciw  punktowi ({\it iii}) dokonano ciekawego odkrycia, jakim jest możliwość zapisu rozpatrywanych operatorów w postaci analitycznej danej {\it explicit\' e}: jesteśmy zdania, że ów zapis może być przydatny w dalszych badaniach występujących w pracy operatorów niezależnie od rozstrzygania stopnia ich gładkości. Analizy takie mogą dostarczyć interesujących wniosków na temat ogólnych własności równowagi rekursywnej w konkretnej badanej gospodarce. 
	  
	  Praca składa się z trzech rozdziałów oraz dodatku matematycznego. W pierwszym rozdziale przedstawiono model Colemana (\citeyear{Coleman1}) oraz zdefiniowano pojęcie równowagi oraz pokazano warunki, przy której zachodzi jej jednoznaczność. W rozdziale drugim podniesiono kwestię różniczkowalności operatorów powiązanych z operatorami wprowadzonymi w poprzednim rozdziale. Rozdział trzeci podsumowuje wyniki pracy, umieszcza je w szerszym kontekście badań nad metodami homotopijnymi oraz zakreśla dalsze kierunki badań. Pracę wieńczy dodatek matematyczny, w którym zostają przedstawione podstawowe pojęcia matematyczne pojawiające się w pracy, oraz w którym zamieszczono dokładne dowody pojawiających się w pracy lematów i twierdzeń.

%%%%%%%%%%%%%%%%%%%%%%%%%%%%%%%%%%%%%%%%%%%%%%%%%%%%%%%%%%%%%%%%%%%%%%%%%%%%%%%%%%%%%%%%%%%%%%%%55

\chapter{Model Colemana}

	W rozdziale tym zostanie zdefiniowany model gospodarki z podatkami pochodzący od Colemana (\citeyear{Coleman1}). Jest to jeden z najprostszych modeli gospodarki z podatkami. Już w tak uproszczonym modelu jednakże nie można powoływać się na Twierdzenia Dobrobytu: wprowadzenie podatków powoduje zaburzenie efektywności systemu gospodarczego. W rozdziale tym opiszemy aktualny stan wiedzy na temat modelu oraz poczynimi szereg uwag, których celem jest poprawienie źle sformalizowanych twierdzeń oraz doprecyzowanie, które założenia są faktycznie istotne przy rozszerzaniu modelu. Rozdział ów podsumujemy ważną, pominiętą w literaturze uwagą o naturalnym wzajemnym powiązaniu modeli rozpatrywanych w obu pracach Colemana (\citeyear{Coleman1, Coleman2}).  

\section{Budowa modelu}\label{chap1sec1}

	Rozważmy za Colemanem (\citeyear{Coleman1}) model dyskretnej stochastycznej gospodarki z nieskończonym horyzontem czasowym. Model próbuje uchwycić zachowanie reprezentatywnego gospodarstwa domowego w gospodarce złożonej z pewnej dużej liczby gospodarstw domowych, dużej liczby firm trudniących się działalnością produkcyjną oraz z Państwa. Dla prostoty podmioty te będziemy dalej łącznie nazywali agentami. Model abstrahuje od określenia dokładnej liczby gospodarstw domowych i firm : ważne jest to, że z perspektywy reprezentatywnego gospodarstwa nie można rozróżnić poszczególnych podmiotów życia gospodarczego, a wyłącznie ich zbiorcze działanie pod postacią pewnych zagregowanych wartości, które można interpretować jako wskaźniki rynkowe. Zbiorcze działanie firm opisuje wyidealizowana funkcja produkcji. Występujące w modelu Państwo pełni rolę {\it stricte} fiskalną: zajmuje się ono opodatkowaniem graczy i dokonaniem bezstratnej redystrubucji tak uzyskanych dochodów. Przyjmujemy, że wszystkie elementy modelu --- gospodarstwa domowe, firmy i Państwo --- to wyłącznie nazwy pewnych matematycznych problemów optymalizacyjnych w abstrakcyjnych przestrzeniach matematycznych. 

Rozpatrywany model jest dynamiczny --- jawnie występuje w nim zmienna czasowa przyjmująca przeliczalną liczbę wartości. W każdej jednostce czasu konsumenci i firmy dokonują decyzji optymalizacyjnych dotyczących wykorzystania zasobów dostępnych w grze. W gospodarce dostępne zasoby to kapitał i praca. Zakładamy, że każde gospodarstwo domowe wykorzystuje w pełni swój zasób pracy. Dokonujemy również jego normalizacji: każde gospodarstwo w każdej turze wydaje dokładnie jedną jednostkę pracy. Przyjęte założenia upoważniają nas do rozpatrywania jednej tylko wartości opisującej zasoby: stosunku kapitału do pracy, którego jednostką jest iloraz arbitralnie dobranej jednostki opisującej wielkość kapitału i jednostki poświęconej ilości pracy. Zasoby reprezentacyjnego gospodarstwa będziemy oznaczać małą literą $ x $, a zmienne agregatowe dużą literą $ X $.

W grze występuje element niepewności: jest to zmienna losowa, która opisuje tzw. {\it egzogeniczne szoki}. Ten losowy parametr obrazuje pewien stopień niewiedzy agentów o działaniu mechanizmu gospodarczego: agenci są bowiem świadomi ogólnego działania gospodarki - znają parametry opisujące innych agentów. Zakładamy, że to wszystko czego nie wiedzą, uwzględnione jest w {\it szoku}. Przyjmujemy, że wszyscy agenci mają wspólne wyobrażenie o strukturze tej zmiennej losowej, czyli przyjmują za dany jeden i ten sam rozkład zmiennej i na podstawie tej wiedzy dokonują prognoz zgodnych z teorią statystyczną. Stąd też oprócz pojęcia zmiennej losowej, w modelu pojawia się pojęcia jej realizacji \footnote{A to jest koncept statystyczny, nie matematyczny.}:
na początku każdej tury, jeszcze przed dokonaniem decyzji, agenci poznają realizację zmiennej losowej: czyli pewną liczbę, której wartość zawiera się w nośniku~\footnote{Nośnikiem funkcji nazywamy domkięcie dopełnienia przeciwobrazu $0$ przy danej funkcji, czyli domknięcie zbioru w dziedzinie funkcji, dla którego przyjmuje ona niezerowe wartości.} zmiennej losowej opisującej szok. Zakładamy, że owa realizacja została zatem wylosowana z rozkładu rozpatrywanej zmiennej losowej.     

Działanie firm zbiorczo opisuje {\it funkcja produkcji} $f$. Zakładamy, że firmy funkcjonują w ramach doskonałej konkurencji. Zakładamy również, że w gospodarce występują stałe efekty skali. Założenie to jest równoważne z opisem gospodarki za pomocą funkcji jednorodnej stopnia pierwszego. Oznacza to, że w rozpatrywanej gospodarce nie występują zyski. Tym samym zbędne okazują się być jakiekolwiek dywagacje o przynależności firm do konkretnych graczy.  

Wyróżnijmy dodatkowe istotne założenia o funkcji produkcji

\begin{ass}\label{ass on prod}
Funkcja produkcji $ f: \mathbb{R}_{+} \times S \rightarrow \mathbb{R}_{+} $ jest względem pierwszego argumentu klasy $ \mathcal{C}^{1}(R_{+}) $, jest ściśle rosnąca, ściśle wklęsła. Dla dowolnego $ s $, $ f(0,s) = 0$. Nadto, istnieje poziom kapitału do pracy $ \overline{x} $, dla którego:
\begin{gather*}
	\underset{s \in S}{\forall}\,\, f(\overline{x}, s) + (1 - \delta)\overline{x} \leq \overline{x} \,,\\
	\underset{s \in S}{\exists}\,\, f(\overline{x}, s) + (1 - \delta)\overline{x} = \overline{x}\,. 
\end{gather*}
\end{ass}

Występujący powyżej zbiór $ S $ to zbiór wartości pewnej zmiennej losowej. Tym samym również i funkcja produkcji to zmienna losowa; samą wartość $ s $ możemy traktować jak parametr. Wprowadzamy stosowane przez resztę pracy oznaczenia
\begin{itemize}
	\item{$ K \equiv [0, \overline{x}]$,}
	\item{$ F(x, s) \equiv f(x,s) + (1 - \delta)x $.}
\end{itemize}

W prawie całej pracy nie będą mogły pojawić się wartości kapitału na jednostkę pracy spoza przedziału $ K $ ~\footnote{Wyjątek stanowi rozdział \ref{sec-lattice}, w którym poprawimy błędne rozumowanie Colemana \citeyear{Coleman2}}.   

Omówimy teraz sposób podejmowania decyzji przez reprezentacyjne gospodarstwo domowe. Zakładamy istnienie funkcji $ u : \mathbb{R} \rightarrow \mathbb{R} $ opisującej chwilowe preferencje gospodarstwa w danej turze, przypisującej danemu poziomowi konsumpcji $ c_t $ w turze $ t $ pewną liczbę rzeczywistą $ u(c_t) $, obrazującą poziom jego zadowolenia. Ponieważ jednak w modelu pojawia się element niepewności --- seria szoków $ s_t $ --- zakładamy, że gospodarstwo domowe określa swoje preferencje w terminach oczekiwanej konsumpcji. Dalej, zakładamy, że gospodarstwo domowe potrafi określić swoją użyteczność z wszystkich okresów czasu rozpatrywanych łącznie, czyli z pewnego ciągu poziomów konsumpcji $ \{ c_t \} $ obarczonych niepewnością za pomocą funkcji $ \mathcal{U}: \mathbb{D} \rightarrow \mathbb{R}$ danej przez

\begin{equation}\label{value_function}
\mathcal{U}(c) =  \mathcal{E} \Bigl \{ \overset{\infty}{\underset{t=0}{\sum}} \beta^t u(c_t) \Bigl \}\,,
\end{equation}

gdzie $ \mathbb{D} $ to pewien zbiór stochastycznych procesów Markowa, $ \beta \in (0,1]$ to współczynnik dyskontujący. Jedyne zródło niepewności w modelu to proces $ s_t $. Wprowadźmy założenia na temat owego procesu:

\begin{ass}
$ s_t : S' \rightarrow S$ to zmienna losowa. Przestrzeń stanów $ S \subset \mathbb{R}_{+}$ jest zbiorem skończonym. Nadto, struktura procesu Markowa zadana jest przez funkcję przejścia $ \pi(s'|s) = \mathbb{P}( s_{t+1} = s' | s_t = s) $. 
\end{ass}

Wprowadźmy dalsze założenia na temat sposobu dokonywania decyzji przez konsumenta.

\begin{ass}\label{ass on u}
 Funkcja chwilowej użyteczności $ u $ jest ciągle podwójnie różniczkowalna na zbiorze $ \mathbb{R}_{++} $: $ u \in \mathcal{C}^{2}(\mathbb{R}_{++})$.
 
Nadto: $ u $ ściśle rośnie, $ u'(0) = +\infty $ oraz $ u'' < 0 $.
\end{ass}

Założenie \ref{ass on u} jest silniejsze od założeń poczynionych przez Colemana (\citeyear{Coleman1, Coleman2}). Implikuje ono oczywiście standardowe założenie o ścisłej wklęsłości $ u $ \citep[patrz:][str. 38]{Avriel}. Samo żądanie podwójnej różniczkowalności nie wydaje się nazbyt wygórowane: powołajmy się na prace innych autorów, takich jak \citet{Greenwood} oraz \cite{Reffett}. Z poczynionych założeń wynika również, że $ u' $ jest funkcją ściśle malejącą. 

	Uściślijmy wreszcie rolę Państwa w opisywanej gospodarce. Państwo opodatkowuje gospodarstwo domowe za pomocą podatków o stopie równej $ \phi $. Tak uzyskany dochód oddaje reprezentatywnemu gospodarstwu z powrotem za pomocą transferu pieniężnego $ d $. 

\begin{ass}\label{ass_on_R}
	$ \phi : K \times K \times S \rightarrow [0,1) $ jest ciągła z pierwszym argumentem. $ (1 - \phi(y,X,s))y $ jest rosnąca i ściśle wklęsła z $ y $, $ \bigl\{ 1- \phi(f(x,s),x,s) - \phi_1 (f(x,s),x,s)f(x,s) \bigl\}f_1 (x,s) $ ściśle maleje z $ x $. $ d: K_{++} \times S \rightarrow K_{++} $ to funkcja taka, że transfer w okresie $ t $ dany jest przez $ d_t = d(X_t,s_t) $.
\end{ass}

Powyższe założenie wymaga objaśnienia. Założenie o doskonałej konkurencji wśród firm implikuje, że opłaty za użytkowanie kapitału i pracy są równe ich krańcowej produktywności. Ponieważ gospodarstwa domowe tworzą nieelastyczną podaż pracy, więc ich dochód z kapitału na jednostkę pracy wynosi $ y \equiv f(X_t, s_t) - (x_t - X_t)f_1 (X_t, s_t) $, co wynika z tożsamości Eulera. Rozpatrywanie wartości zagregowanych jako argumentu funkcji produkcji odpowiada temu, że firmy nie rozróżniają reprezentacyjnego gospodarstwa spośród wszystkich gospodarstw. Założenia o wklęsłości $ (1 - \phi)y $ oznacza, że nasza analiza skupia się na przypadku opodatkowania progresywnego, odczuwalnego przez reprezentacyjne gospodarstwo domowe~\footnote{Istnieje możliwość modelowania sytuacji, gdzie w wartościach zagregowanych podatek jest progresywny, ale gospodarstwo domowe tego tak nie postrzega.}. Pozostałe założenia mają charakter techniczny.

Gospodarstwo domowe chce maksymalizować wyrażenie \ref{value_function} poprzez dobór optymalnego procesu stochastycznego konsumpcji. Działanie to posiada ekonomiczną interpretację: jest to planowanie scenariuszowe~\footnote{Angielski termin {\it contingency planning} wywodzi się pierwotnie z nomenklatury wojskowej i oznacza opracowanie schematu działania w zależności od zaistniałych okoliczności.}. W swoim problemie optymalizacyjnym reprezentatywne gospodarstwo musi uwzglęnić również następujące ograniczenie:

$$ x_{t+1} = \Bigl\{ 1 - \phi(y_t, X_t, s_t) \Bigl\}y_t + d(X_t, s_t) + (1 - \delta)x_t  - c_t \,.$$

Jest to równanie ścieżki indywidualnego poziomu kapitału na jednostkę pracy. Jego postać wynika z tego, że poziom kapitału na jednostkę pracy w turze $ t+1 $ jest równy temu, co pozostaje z opodatkowanych przychodów po doliczeniu transferu pieniężnego od rządu i uwzględnieniu konsumpcji i deprecjacji poprzedniej wartości kapitału na jednostkę pracy.

Uściślijmy również jak wygląda pojawiający się w \ref{value_function} zbiór procesów Markowa $ \mathbb{D} $, po których następuje optymalizacja. Oznaczmy dany stan gospodarki przez wektor $ (x, X, s) \in K_{+} \times K_{+} \times S $. Załóżmy, że $ s $ to realizacja egzogenicznego szoku. Wówczas poziom konsupcji, na który może zdecydować się reprezentacyjne gospodarstwo pochodzi z przedziału postaci

\begin{equation*}
	M(x, X, s) \equiv \Bigl[ 0, (1 - \phi(y, X,s))y + (1-\delta)x + d(X,s) \Bigl]\,.
\end{equation*}

Stosunek reprezentacyjnego gospodarstwa do pozostałych uczestników gry w pełni opisuje funkcja $ g: K_{++} \times S \rightarrow K_{++} $. Zakładamy, że opisuje ona ewolucję w czasie zagregowanych wartości stosunku kapitału do pracy : $ X_{t+1} \equiv g(X_t, s_t)$.


Dla uproszczenia rachunków, przyjmijmy następujące oznaczenia:

\begin{itemize}
	\item{$y = y(x,X,s) \equiv f(X, s) + (x - X)f_1 (X,s)$,} 
	\item{$W = W(x,X,s) \equiv (1- \phi(y,X,s))y + 1 - \delta + d(X,s) - c $.}	
\end{itemize}

Tym samym możemy ostatecznie przepisać problem optymalizacyjny reprezentatywnego gospodarstwa w postaci rekursywnej, czyli odpowiadające mu równanie Bellmana

\begin{equation}\label{bellman1}
\mathcal{V}(x,X,s) = \underset{c \in M}{\sup} \Biggl\{ u(c) + \beta \mathcal{E}_s \bigl \{ \mathcal{V}\Big(W(x,X,s), g(X,s), s'\Big) \bigl\} \Biggl\}\,.
\end{equation}	

Możemy teraz powołać się na zestaw twierdzeń dotyczących takich sformuowań, które można znaleźć w monografii Stokey, Lucasa i Prescotta (\citeyear{Prescott}):

\begin{tw}
Dla dowolnych ciągłych funkcji $ g: K_{++} \times S \rightarrow	K_{++} $ i $ d: K_{++} \times S \rightarrow	K_{++} $ istnieje dokładnie jedna ograniczona i ciągła rzeczywista funkcja $ V: \mathbb{R}_{+} \times K_{++} \times S \rightarrow \mathbb{R} $, ściśle rosnąca i ściśle wklęsa względem pierwszego argumentu, która stanowi rozwiązanie równania funkcyjnego \ref{bellman1}. Powiązana jest z nią funkcja $ C: \mathbb{R}_{+} \times K_{++} \times S \rightarrow \mathbb{R}_{+}$, dla której osiągnięte zostaje supremum w \ref{bellman1}, która jest wyznaczona jednoznacznie i jest ciągła względem pierwszego argumentu. 
\end{tw}

Wynika stąd, że mamy do czynienia z rodziną rozwiązujących równanie Bellmana funkcji wartości $ \mathcal{V} $, indeksowaną funkcjami transferu $ d $ i ewolucji kapitału $ g $. Spośród tej rodziny interesuje nas wybranie bardzo konkretnego rozwiązania. Prowadzi to do następującej definicji

\begin{definicja}\label{def_equilibrium}
	Równowaga stacjonarna to para ciągłych funkcji $ (g,d): K_{++} \times S \rightarrow	K_{++}^2$ spełniających dodatkowo
	\begin{enumerate}
		\item{ \label{LaJ}$d \equiv \phi f$,}
		\item{ \label{CP}$g(x,s) \equiv F(x,s) - C(x,x,s)$.}
	\end{enumerate}
\end{definicja}

Warunek \ref{LaJ} oznacza, że pobrane podatki zostały bezstratnie rozdysponowane, czyli Państwo funkcjonuje optymalnie. Warunek \ref{CP} jest bardziej istotny: stwierdza on, że w równowadze wszystkie gospodarstwa domowe występujące w grze postępują w ten sam, optymalny dla nich sposób. Definicja ta nawiązuje tym samym do pojęcia równowagi Nasha. Badaniu zachowania się funkcji $ g $ będzie poświęcona niniejsza praca. 



%%%%%%%%%%%%%%%%%%%%%%%%%%%%%%%%%%%%5

\section{Równanie Eulera}\label{chap_Euler}

Problem funkcyjny zdefiniowany przez równanie \ref{bellman1} można zapisać w równoważnej postaci bez potrzeby odwoływania się do funkcji wartości {\it explicit\'e}~\footnote{Szczegóły -- patrz: \citet[][rozdz. 9.5, str. 280-283]{Prescott}}.
		
Wprowadźmy oznaczenia stosowane przez dalszą część pracy:

\begin{itemize}
		\item{ $R = R(x,s) \equiv \beta \Bigl \{[1 - \phi(f(x,s),x,s) - \phi_1 (f(x,s),x,s)f(x,s)]f_1 (x,s) + 1 - \delta \Bigl \} $,}
		\item{ $c = c(x,s) \equiv C(x,x,s) $.}
\end{itemize}

Przy powyższych oznaczeniach można udowodnić prawdziwość następującego wzoru 

\begin{equation}
u'(c(x,s)) =   \mathcal{E}_s \bigl \{ u'(c(F(x,s)-c(x,s),s'))R(F(x,s)-c(x,s),s')	\bigl \} 
\end{equation}

zapisywanego skrótowo~\footnote{Skrótowe zapisywanie działań w algebrze funkcji będzie często stosowane w dodatku matematycznym. Jego wykorzystanie znacząco upraszcza postać rachunków.}

\begin{equation}\label{Euler_easy}
u'(c) =   \mathcal{E}_s \bigl \{ u'(c(F-c, s'))R(F-c,s')	\bigl \}\,. 
\end{equation}

Obydwa wzory są równoważne rozpatrywanemu poprzednio zagadnieniu Bellmana \ref{bellman1}. Nadto, ich postać analityczna faktycznie uwalnia nas od funkcji wartości $\mathcal{V}$. Czyni to owe przedstawienie zagadnienia szczególnie wygodnym do dalszych badań.  

Dokładne wyprowadzenie owych wzorów odkładamy do rozdziału \ref{chap_appendice} --- patrz: Dowód \ref{proof_Euler}.

%%%%%%%%%%%%%%%%%%%%%%%%%%%%%%%%%%%%%%%%%%%%%%%%%%%%%%%%%%%%%%%%%%%%%%%%%%%%%%%%%%%%%%%%%%%%%%%%%%%%%


\section{Operator Colemana}\label{section_Coleman_operator}

Okazuje się, że problem znalezienia rozwiązania równania Eulera \ref{Euler_easy}, czyli funkcji konsumpcji $ c $, można znacząco uprościć, gdy uściślimy, na jakim rozwiązaniu nam zależy. \citet{Coleman1, Coleman2} wykazał, że w pewnej i tak stosunkowo szerokiej klasie funkcji, takie rozwiązanie jest jedyne. Sprecyzujemy teraz, o jaką klasę funkcji chodziło.

Rozważmy następujący zbiór funkcji

\begin{equation}\label{C_F}
\mathcal{C}_{F}(K \times S) = \left \{ \begin{matrix}
c: K \times S \overset{\mathcal{C}}{\rightarrow} K \cr
 \underset{(x,s)\in K\times S}{\forall}0 \leq c(x,s) \leq F(x,s) \cr
 y \geq x \Longrightarrow 0 \leq c(y,s)-c(x,s)\leq F(y,s) - F(x,s)  	\cr 
\end{matrix} \right \}\,.
\end{equation}  

Zatem $ \mathcal{C}_{F} $ to zbiór ciągłych, niemalejących funkcji zawartych w paśmie pomiędzy $ 0 $ a $ F $, takich, że ich odbicie względem $ F $ -- funkcja $ F-c $, jest również niemalejąca.

Zauważmy, że powyższy zbiór składa się z funkcji ograniczonych (z góry przez $ F $, z dołu przez funkcję stale równą $ 0 $), oraz równociągłych (gdyż ich moduł ciągłości jest zdominowany przez moduł ciągłości ciągłej funkcji F). Tym samym na mocy twierdzenia Arzeliego-Ascoli: ów zbiór jest zwarty w normie supremum ~\footnote{Dowód można : \citet[][str. 411]{Rudin}, lub \citet[][str. 194]{Engelking}}. Łatwy rachunek pozwala sprawdzić, że jest to również zbiór wypukły. 

Na zbiorze $ \mathcal{C}_{F} $ możemy zadać implicit\'e operator $ A $ w następujący sposób:

\begin{equation}\label{def A}
u'(A[c](x,s)) =   \mathcal{E}_s \bigl \{ u'(c(F(x,s)-A[c](x,s),s'))R(F(x,s)-A[c](x,s),s')	\bigl \}\,. 
\end{equation}

W przypadku gdy prawa strona równania przyjmuje wartość nieskończoną (np. gdy $ c(x,s) = 0 $), wówczas kładziemy dodatkowo $ A[c](0) = 0 $ i przyjmujemy, że zdanie $\infty = \infty$ może zachodzić. 

Zanim przejdziemy do badania własności operatora $ A $ zauważmy, że można jego zapis znacząco uprościć nie tracąc podstawowych własności, na których będzie nam zależało. Zdefniujmy mianowicie rzeczywistą funkcję dwóch zmiennych $ \rho_c : \overline{\mathbb{R}}_+ \times S \rightarrow \overline{\mathbb{R}}_+$ wzorem

\begin{equation}
\rho_{c}(\omega,s) \equiv \mathcal{E}_s \bigl \{ u'(c(\omega,s'))R(\omega,s')	\bigl \} \,.
\end{equation}

Zauważmy, że jeśli $ c \in \mathcal{C}_F $, oraz jeśli $ S $ jest zbiorem skończonym, to $ \rho $ jest ściśle malejąca z $ \omega $. Wynika to z tego, że $ u' \circ c $ jest nierosnąca z $ \omega $, oraz $ R $ jest ściśle malejąca z $ \omega $. A ponieważ $ u' \circ c $, $ R > 0$, zatem ich iloczyn jest funkcją ściśle malejącą. To, że iloczyn funkcji nierosnącej i ściśle malejącej, obu ściśle większych od zera, jest ściśle malejący wynika z tego, że dla dowolnych $ a,b,c,d \in \mathbb{R}_{++} $ zachodzi wynikanie:

$$ \Big(0 < a < b \,\,\wedge\,\, 0 < c \leq d \Big)\,\Longrightarrow \,0 < ac < bd \,.$$
 
Nadto, skończona kombinacja liniowa ściśle monotonicznych funkcji o stałym znaku, gdzie współczynniki kombinacji są tego samego znaku, również jest ściśle monotoniczna, modulo znak współczynników kombinacji. Tym samym $ \rho $ jest ściśle malejącą funkcją~\footnote{A więc i odwracalną, co przyda się w dalszej części pracy.}. 

Teraz łatwo wykazać następujący

\begin{lemat}\label{on A}
	$ A : \mathcal{C}_{F} \rightarrow \mathcal{C}_{F}  $ działa na $ \mathcal{C}_F $ w sposób ciągły.
\end{lemat}

Działanie (dla ścisłości - działanie z lewej strony) operatora rozumiemy w sensie teorio-grupowym: znaczy to tyle, że gdy $ g $ to element pewnego ustrukturyzowanego algebraicznie zbioru, wówczas $ A[g] $ jest również elementem tego zbioru. Dowód powyższego lematu odkładamy do dodatku matematycznego (patrz: Dowód \ref{A_acts_on_C_F}).

Jest jasne, że każdy funkcja konsumpcji $ c \in \mathcal{C}_F $ spełniająca $ c[K_{++}]>0 $ oraz będąca punktem stałym operatora $ A $ jest rozwiązaniem równania funkcyjnego \ref{Euler_easy} : wystarczy zamiast $A[c]$ podstawić $c$ aby odzyskać wzór definiujący równowagę. Ta zaś musi być niezerowa na mocy definicji \ref{def_equilibrium}.

Warto wspomnieć o tym, że zbiór $ \mathcal{C}_F $ posiada własność punktu stałego: dowolne ciągłe przekształcenie tego zbioru w siebie ma punkt stały -- jest to bezpośrednia konsekwencja znanego twierdzenia Schaudera~\footnote{Szczegóły: \citet[][str. 119]{Granas_FPT}.}. Jednakże twierdzenie Schaudera ma charakter egzystencjalny i nie precyzuje dalszych własności punktu stałego -- wiadomo wyłącznie, że jest to element zbioru na którym określamy ciągły operator. Nader wszystko wiadomo, że funkcja $ c \equiv 0 $ jest punktem stałym operatora $ A $, a taki punkt stały nas nie interesuje. Widząc te piętrzące się trudności, Coleman w pracach (\citeyear{Coleman1, Coleman2}) rezygnuje z czysto topologicznych argumentów, zwykle stosowanych przy dowodzeniu istnienia punktu stałego, i odwołuje się do argumentów teorio-kratowych. 

Przed przystąpieniem do zgłębienia szczegółów owego interesującego podejścia dokonajmy najpierw daleko idącego uproszczenia problemu badawczego, zaproponowanego przez Colemana (\citeyear{Coleman2}). 


%%%%%%%%%%%%%%%%%%%%%%%%%%%%%%%%%%%%%%%%%%%%


\section{Operator sprzężony}\label{conjugate_operator}

Rozważmy funkcję $ H: \mathbb{R}_{+} \rightarrow \mathbb{R}_{+} $ zdefiniowaną następująco:

\begin{align*}
\begin{split}
H\Bigl(\frac{1}{u'(c)} \Bigl) &= c\,, \\
H(0) &= 0 \,.\\
\end{split}
\end{align*}

Powyższy napis równoważny jest zatem następującemu

\begin{equation*}
H(\omega) = u'^{(-1)}\Bigl(\frac{1}{\omega}\Bigl)\,.
\end{equation*}

Wynika stąd, że $ H $ jest funkcją ściśle rosnącą, bowiem jest to złożenie dwóch funkcji ściśle malejących. Ciągłość $ H $ wynika z prostego lematu.

\begin{lemat}\label{homomorphisms}
Jeśli $ f : \mathbb{R} \rightarrow \mathbb{R}$ to funkcja ciągła i ściśle monotoniczna, to jest to homeomorfizmem, czyli $ f^{(-1)} $ jest również ciągła. 
\end{lemat}

Dowód lematu odkładamy do dodatku matematycznego (zobacz: Dowód \ref{proof_homeomorphism}).

Stąd też wynika, że również $ H $ jest ciągła: jest to bowiem złożenie funkcji ciągłych w topologii półprostej rzeczywistej domkniętej w nieskończoności, czyli $\mathbb{R}_{+} \cup \{ \infty \}$. 

Innym sposobem udowodnienia ciągłości H jest wykazanie jej różniczkowalności. Będzie nas interesowała różniczkowalność $ H $ jako funkcji będącej podzbiorem kwadratu $ K^2 $. Z zapisu $ H(\omega) = u'^{(-1)}(\frac{1}{\omega}) $ wynika od razu, że $ H $ jest różniczkowalna na $ (0, +\infty) $, jako że pochodna dana jest dobrze określonym wzorem:

\begin{equation}\label{H'}
	H ' (\omega	) = \frac{-1}{\omega^{2} u''(u'^{(-1)}(\frac{1}{\omega})) } > 0\,.
\end{equation}

Pozostaje sprawdzić, pod jakimi warunkami funkcja $ H' $ daje się w sposób ciągły rozszerzyć w $ 0 $. 
Na mocy założeń \ref{ass on u} o funkcji chwilowej użyteczności , $ u'' < 0 $. Chcielibyśmy zatem, żeby $ H'(0^{+}) = M \in \mathbb{R}_{++} $. Powody, dla których nie chcemy, aby $H'(0)=0$ staną się niedługo jasne. Tym samym, asymptotycznie w okolicy $ 0 $ powinno zachodzić: 

\begin{equation*}
\frac{-1}{\omega^{2} u''(u'^{(-1)}(\frac{1}{\omega}))} \approx M > 0\,.
\end{equation*} 

Przyjmując $ z =  u'^{(-1)}(\frac{1}{\omega})$ powyższy napis jest równoważny:

\begin{equation*}
u''(z) \approx - \frac{u'(z)^2}{M}\,.
\end{equation*}

Zastępując znak $ \approx $ równością i rozwiązując tak powstałe równanie różniczkowe otrzymujemy wzór na asymptotę chwilowej użyteczności w 0:

\begin{lemat}\label{u asymptotically} Przy dodatkowym założeniu o asymptotycznym zachowaniu funkcji chwilowej użyteczności postaci
   \begin{equation}
   u(\omega) \approx M \log \Bigl[ \frac{\omega}{C} + 1 \Bigl]
   \end{equation}	
	funkcja $ H $ jest klasy $ \mathcal{C}^{1}(\mathbb{R}_{+}) $.
\end{lemat}

Tym samym w okolicy zera funkcja użyteczności powinna zachowywać się jak pewien element rodziny funkcji logarytmicznych, indeksowanej stałymi $ M,C > 0$. Różniczkowalność $ H $ na $ K $ pociąga jej ciągłość. Ale jej różniczkowalność będzie nam potrzebna również z innych powodów.

Najpierw jednak zdefiniujmy za Colemanem operator o prostszej postaci, aniżeli ten zdefiniowany równaniem \ref{def A}. Zauważmy, że $ H $ spełnia również zależność $ u'(H(m)) = 1/m $. Załóżmy zatem, że $ c = H(m) $. Wówczas $ m $ jest wyznaczona jednoznacznie dzięki temu, że $ H $ jest funkcją wzajemnie jednoznaczną. Prawdziwy jest wówczas następujący ciąg przekształceń na równaniu \ref{Euler_easy} definiującym równowagę:

\begin{align*}
\begin{split}
u'\Big(c(x,s)\Big) 	&=   \mathcal{E}_s \Bigl \{ u'\Bigl(c\big(F(x,s)-c(x,s), s'\big)\Bigl)R\big(F(x,s)-c(x,s), s'\big)	\Bigl \} \,,\cr
u'\Bigl(H\Big(m(x,s)\Big)\Bigl)&=   \mathcal{E}_s \Bigl \{ u'\Big([H \circ m](F(x,s)-H\Big(m(x,s)\Big), s')\Big)R\Big(F(x,s)-H\big(m(x,s)\big),s'\Big)	\Bigl \} \,,\cr
\frac{1}{m(x,s)} &=   \mathcal{E}_s \Bigl \{ \frac{R(F(x,s)-H\big(m(x,s)\big),s')}{m(F(x,s)-H\big(m(x,s)\big), s')}	\Bigl \} \,.\cr	
\end{split}
\end{align*} 
 
	Ostatni wynik skrótowo zapisujemy 

\begin{equation*}
	\frac{1}{m} =   \mathcal{E}_s \Bigl \{ \frac{R(F - H(m),s')}{m(F - H(m), s')}	\Bigl \} \,.
\end{equation*}	
	 
Podobnie jak poprzednio przeprowadzamy przekształcenia na równaniu definiującym operator $ A $: dla uproszczenia rachunków pomijamy argumenty funkcji.

\begin{align*}
\begin{split}
	u'(A[c]) &=   \mathcal{E}_s \bigl \{ u'(c(F-A[c],s'))R(F-A[c],s')	\bigl \} \,,\cr
	u'(H \circ H^{-1} \circ A[H(m)]) &=   \mathcal{E}_s \bigl \{ u'([H\circ m](F-H \circ H^{-1} \circ A[H(m)],s'))R(F-H \circ H^{-1} \circ A[H(m)],s')	\bigl \} \,,\cr
	\frac{1}{H^{-1} \circ A[H(m)]} &=   \mathcal{E}_s \Bigl \{ \frac{R(F-H \circ H^{-1} \circ A[H(m)],s')}{m(F-H \circ H^{-1} \circ A[H(m)], s')}	\Bigl \}\,.\cr	
\end{split}
\end{align*}
 
Zatem wprowadzając oznaczenie $ B \equiv H^{-1} \circ A \circ H $, prosto otrzymujemy wyrażenie zapisywane skrótowo

\begin{equation}\label{op B}
\frac{1}{B[m]} =   \mathcal{E}_s \Bigl \{ \frac{R(F-H(B[m])],s')}{m(F-H(B[m]), s')}	\Bigl \} \,.
\end{equation}

Wyrażenie \ref{op B} definiuje operator. Przestrzeń, na której będzie on działał definiujemy działając na $ \mathcal{C}_F $ funkcją $ H $. Sposób wyznaczania wartości funkcji $B[m]$ można łatwo przedstawić graficznie, patrz Rysunek \ref{rys1}. Dla różnych poziomów kapitału na jednostkę pracy spełniających $x_1 < x_2 < x_3$ naniesione na wykres trzy krzywe $\frac{R}{m}\circ [F(x_i) - H(y)]$, które należy interpretować jako funkcje zmiennej $y$. Aby otrzymać wartości operatora $B[m]$ wystarczy popatrzeć na punkt przecięcia każdej takiej funkcji indeksowanej $x$ z hiperbolą $\frac{1}{y}$, a następnie zrzutować tak otrzymane punkty na oś odciętych. Umownie oznaczamy najwyższe punkty rysunku jako nieskończoności. Zauważmy, że to właśnie wartość nieskończoną przyjmują funkcje $\frac{R}{m}\circ [F(x_i) - H(y)]$ w punkcie $y = H^{(-1)}(x_i)$. Wtedy, ponieważ topologicznie uzwarciliśmy półprostą i ponieważ funkcja $\frac{R}{m}$ jest ciągła, zatem jej wartość w zerze powinna przyjmować wartość nieskończoną,  $\frac{R}{m}\circ [0] = \infty$. 

\begin{figure}
	\begin{center}
		\includegraphics[scale=1]{essais2.pdf} 
		\caption[Wyznaczanie wartości $B\brackets{m}$]{Wyznaczanie wartości $B[m]$. \\ \rightline{\footnotesize{Źródło: opracowanie własne.}}}\label{rys1}
	\end{center}
\end{figure}

Przywołajmy ważne twierdzenie pochodzące z książki Mardsena, Ratiu i Abrahama (\citeyear[][Str. 92]{Mardsen})~\footnote{Uogólnienie twierdzenia, patrz: Twr. \ref{true_omega}}

\begin{tw}[Lemat Omega]\label{First diff} 
	Niech $ M $ to zwarta przestrzeń topologiczna, a $ \mathbb{E} $ oraz $ \mathbb{F} $ to przestrzenie Banacha. Niech również $ U $ jest otwarty w topologii na $ \mathbb{E} $. Wówczas jeśli $ g: U \rightarrow F $ jest klasy $ \mathcal{C}^{r}(U) $, gdzie $ r >0 $, to przekształcenie $ \Omega_g : \mathcal{C}^{0}(M, U) \rightarrow \mathcal{C}^{0}(M, \mathbb{F}) $ dane wzorem $ \Omega_g (f) = g \circ f $ jest również klasy $ \mathcal{C}^r $ w topologii normy supremum. 
	
Wzór na różniczkę operatora $ \Omega_g $ w punkcie $ f $ na funkcji $ h $ dany jest przez:
\begin{equation*}
 [\diff{\Omega_g}(f)h](x) \equiv \diff{g}(f(x))h(x)\,.
\end{equation*}
\end{tw} 
 
Żeby skorzystać z tego twierdzenia brakuje nam otwartego zbioru, zawierającego $ K $, na którym określona byłaby pochodna. Ponieważ znamy asymptotyczne zachowanie się $ H $ w otoczeniu zera i dzięki temu wiemy, że funkcja ta w $ 0 $ przyjmie skończoną wartość, jest rzeczą zupełnie naturalną rozszerzyć funkcję $ H' $ na lewo od zera w sposób zupełnie dowolony. Nadto, nie przejmujemy się obecnością losowego parametru $ s \in S$. Ponieważ $ S $ jest skończona, więc możemy patrzeć na $ H $ tak, jakby to było działanie $H$ na iloczynie kartezjańskim złożonym z kilku kopii przestrzeni funkcji ciągłych.

Z twierdzenia \ref{First diff} wynika różniczkowalność $ H $ jako działania pomiędzy zbiorami funkcji ciągłych określonych na zwartym przedziale - przestrzeni topologicznej znacznie większej od $ \mathcal{C}_F $. Wynika stąd też jej ciągłość na tym większym zbiorze. Ponieważ zaś $ \mathcal{C}_F $ to podprzestrzeń topologiczna~\footnote{Patrz: Dodatek matematyczny \ref{preliminaries}} przestrzeni, na której dowiedliśmy ciągłości $ H $, zatem $ H $ jest też oczywiście ciągłe na podprzestrzeni. Tym samym, zbiór $ \mathcal{M} \equiv \{H(c) | c \in \mathcal{C}_F \} $ jest zbiorem zwartym w normie supremum, bo $\mathcal{C}_F $ jest takim zbiorem. Tym samym operator $ B $ działa na $ \mathcal{M} $, czyli $ B: \mathcal{M} \rightarrow \mathcal{M} $. Dla porządku podajmy explicit\'e jakie funkcje wchodzą w skład tego zbioru~\footnote{\label{dupa}Przy okazji zauważmy, że w pracy Colemana (\citeyear{Coleman2}) pojawił się błędnie zapisany 2 warunek poniższego opisu zbioru $ \mathcal{M} $}: 


\begin{equation*}
\mathcal{M}= \left \{ \begin{matrix}
m: K \times S \overset{\mathcal{C}}{\rightarrow} K \cr
 \underset{(x,s)\in K\times S}{\forall} 0 \leq m(x,s) \leq H^{(-1)}\bigl(F(x,s)\bigl) \cr
 y \geq x \Longrightarrow 0 \leq H(m(y,s))-H(m(x,s))\leq F(y,s) - F(x,s)  	\cr 
\end{matrix} \right \}\,.
\end{equation*}  

Jako że $ H $ jest klasy $ C^{1}(\mathbb{R}_{+}) $, więc ostatni warunek definiujący zbiór $ \mathcal{M} $ możemy przedstawić w postaci bardziej eksponującej ograniczenie na przyrost funkcji $ m $. Wystarczy tylko skorzystać ze znanego wzoru Lagrange'a: wprowadzamy pomocniczą funkcję $ \rho: [0,1] \rightarrow K $ daną przez $ \rho(\tau) = m(x,s) + \tau \big( m(y,s) - m(x,s) \big) $. Ponieważ $ \rho $ jest gładka względem parametru $ \tau $, zatem zachodzi


\begin{equation*}
\begin{split}
H(m(y,s))-H(m(x,s)) = H(\rho(1))-H(\rho(0)) = \int_{0}^{1} (H \circ \rho)'(\tau) \mathrm{d} \tau =\\
=  [m(y,s)-m(x,s)]\int_{0}^{1} H'\Bigl(m(x,s) + \tau \big[ m(y,s) - m(x,s) \big] \Bigl)\mathrm{d} \tau\,.
\end{split}
\end{equation*}

Ponieważ zakładamy, że $ H' \in \mathcal{C}(\mathbb{R}_{+}) $ oraz wiemy, że przy odpowiednim rozszerzeniu $ H' $ jest ściśle dodatnia na $ K $. Korzystając ze zwartości $ K $ wnioskujemy, że istnieje $ \omega $ dla której $ H'(\omega) = \underset{x \in K}{\min}\{H'(x) \} $ oraz $ H'(\omega) > 0 $ . Korzystając teraz dodatkowo z różniczkowalności $ F $ i jej ścisłej wklęsłości oraz ścisłej monotoniczności konkludujemy, że $ F'(x,s) \leq F'(0,s) $. Tym samym:

\begin{equation}\label{lipschitzness of M}
0 \leq m(y,s)-m(x,s) \leq \frac{F(y,s) - F(x,s)}{\int_{0}^{1} H'\Bigl((1-\tau)m(x,s) + \tau m(y,s\Bigl)\mathrm{d} \tau} \leq \frac{F'(0,s)}{H'(\tilde{\omega})}(y-x)\,.
\end{equation}  

Wielkiej wagi jest to, że współczynnik $ L_s \equiv \frac{F'(0,s)}{H'(\tilde{\omega})} $ to stała mniejsza od nieskończoności i zależna tylko od parametru $ s $. Powyższy wzór oznacza zatem, że zbiór $ \mathcal{M} $ to pewien podzbiór zbioru funkcji spełniających warunek Lipschitza ze stałą $ L_s $. Zauważmy, że na mocy twierdzenia Rademachera \footnote{Patrz: \citet[][twr. 3, str. 19]{Heinonen}}, funkcje ze zbioru $ \mathcal{M} $ są prawie wszędzie różniczkowalne względem miary Lebesgue'a. Wniosek ten motywuje naturalność rozpatrywania pewnej innej przestrzeni funkcyjnej: szczegóły tej uwagi pozna czytelnik w rozdziale \ref{chap_diff}.  

Na koniec zauważmy, że skoro $A$ to operator ciągły w normie supremum, to również $B$ jest ciągły w normie supremum. Wynika to z następujących oszacowań

\begin{equation*}
	\delta > || c_1 - c_2||= || H(m_1) - H(m_2)|| = || (m_1 - m_2)\int_{0}^{1} H'\bigl((1-\tau )m_1 + \tau m_2 \bigl)|| \geq \underset{K}{\min}(H') ||m_1 - m_2 ||\,.
\end{equation*}

Ponieważ zaś warunek $|| c_1 - c_2||_{\infty} < \delta$ pociąga $|| A[c_1] - A[c_2]||_{\infty} <\epsilon$, oraz

\begin{equation*}
	|| A[c_1] - A[c_2]||_{\infty} = || H(B[m_1]) - H(B[m_2])||_{\infty} \geq \underset{K}{\min}(H') ||B[m_1] - B[m_2] ||_{\infty}\,.
\end{equation*}

Zatem, odpowiednio skalując delty i epsilony, konkludujemy, że jeśli $ ||m_1-m_2||_{\infty} < \tilde{\delta}$, to $|| B[m_1] - B[m_2]|| < \tilde{\epsilon}$, co dowodzi ciągłości operatora $B$.


Dodajmy w końcu ekonomiczną interpretację elementów zbioru $\mathcal{M}$: dokładnie rzecz ujmując jest to złożenie hiperboli z funkcją krańcowej użyteczności z danego poziomu konsumpcji $c$. W tym sensie można elementy $\mathcal{M}$ uznać za funkcje mierzące krańcowy brak użyteczności z danego poziomu konsumpcji. Poświęcając prostotę ekonomicznej interpretacji badanego obiektu potrafimy jednak głębiej wniknąć w matematyczną naturę problemu, o czym przekonamy się w następnych rozdziałach. Oczywiście, działając na $\mathcal{M}$ funkcją $H^{(-1)}$ zawsze odzyskamy obiekty o ciekawym znaczeniu z punktu widzenia ekonomii teoretycznej.   


%%%%%%%%%%%%%%%%%%%%%%%%%%%%%%%%%%%%%%%%%%%%%%%%%%%%%%%%%%%%%%%%%%%%%%%%%%%%%%%%%%%%%%%%%%%%

\section{Rozważania teorio-kratowe}\label{sec-lattice}




Największym wkładem Colemana (\citeyear{Coleman1, Coleman2}) w teorię ekonomii było wykorzystanie teorii krat do analizy problemu istnienia i jednoznaczności równowagi w zbiorze $ \mathcal{C}_F $ lub $ \mathcal{M} $. W artykule z \citeyear{Coleman1} roku udowodniono twierdzenia o istnieniu ściśle niezerowego na $ K_{++} $ punktu stałego operatora $ A $ oraz o jego jedyności.  Ograniczono się tam jednakże wyłącznie do preferencji o stałej relatywnej awersji do ryzyka~\footnote{Ang.: Constant Relative Risk Aversion}, o postaci $ u(x) = \frac{x^{1 - \sigma}}{1-\sigma} + C $. To podejście nie może nas satysfakcjonować z powodu braku kompatybilności z wymienionym w lemacie \ref{u asymptotically} warunkiem na postać asymptoty w $ 0 $ funkcji użyteczności. Niemniej w artykule z \citeyear{Coleman2} roku, Coleman znacznie rozszerzył wyniki jednoznaczności rozwiązań, nie zajmując się jednak ich istnieniem. Okazuje się, że postać użyteczności nie ma wpływu na określenie jednoznaczności punktu stałego, dopóki spełnione są założenia \ref{ass on u}. 

Metody użyte przez Colemana oparte są na twierdzeniach, które można znaleźć w pracach Granasa oraz Dugundji'ego (\citeyear{Granas_FPT}) oraz w pracy Krasnoselskiego i Zabreiki (\citeyear{Krasno}). W dowodzeniu własności operatora na przemian będziemy korzystać z jego postaci pierwotnej i sprzężonej. Całość dowodów opiera się na geometrycznych własnościach rozpatrywanych operatorów.

Rozpocznijmy od kluczowej uwagi

\begin{lemat}\label{lemat_o_monotonicznosci}{ 
 
 Zbiory funkcji $ \mathcal{C}_F $ oraz $ \mathcal{M} $ to kraty zupełne względem relacji porządku zadanego następująco dla $ f,g \in W \in \{\mathcal{C}_F,\mathcal{M}  \} $
 
 $$ f \geq g \leftrightarrow \underset{x \in K}{\forall} f(x) \geq g(x)\,. $$

Wówczas też $ 1_{\mathcal{C}_F} = F $, $ 0_{\mathcal{C}_F} = 0 $, $ 1_{\mathcal{M}} = H^{-1}\circ F $ oraz $ 0_{\mathcal{M}} = 0 $. Nadto, operator $ A:\mathcal{C}_F \rightarrow \mathcal{C}_F $ to operator monotoniczny na $ \mathcal{C}_F $, a operator $ B:\mathcal{M} \rightarrow \mathcal{M} $ to operator monotoniczny na $ \mathcal{M} $.  
}\end{lemat}

Dowód powyższego lematu odkładamy do dodatku matematycznego (Patrz: Dowód \ref{proof_lattices}). Przywołamy tylko pochodzącą zeń ważną uwagę, która ułatwia geometryczne wyobrażenie sobie działania rozpatrywanych operatorów: $ B[m](x,s) $ jako funkcja powstaje poprzez rozpatrzenie rzutów na oś odciętych punktów przecięcia hiperboli $ \frac{1}{y} $ z funkcją $ \tilde{\rho}_m (y,s) \equiv \mathcal{E}_s \Bigl \{\Bigl[ \frac{R}{m} \Bigl] (F(x,s)- H(y), s') \Bigl \} $. Rozpatrując funkcję $n$ mniejszą od $ m $, czyli taką, że dla każdego $x \in K$ i $s\in S$ zachodzi $ n(x,s) \leq m(x,s) $ zauważamy, że nowe punkty przecięcia znajdują się wyżej przy każdym ustalonym $x $, patrz Rysunek \ref{rys2}. Tym samym ich rzuty na oś odciętych będą miały mniejsze wartości od tych wyznaczonych przez krzywą $m$. Odpowiada to temu, że $ B[n] \geq B[m] $. 


\begin{figure}
	\begin{center}
		\includegraphics[scale=1]{essais3.pdf} 
		\caption[Porównanie wartości $B\brackets{m}$ z $B\brackets{n}$, gdy $n \leq m$]{Porównanie wartości $B[m]$ z $B[n]$, gdy $n \leq m$. \\ \rightline{\footnotesize{Źródło: opracowanie własne.}}}\label{rys2}
	\end{center}
\end{figure}

Wiedząc już, że oba operatory są monotoniczne możemy wykorzystać pewną modyfikacją cytowanego już twierdzenia Tarskiego-Kantorowicza (patrz, Twierdzenie \ref{Tarski}) do udowodnienia istnienia punktu stałego operatora $B$. Tym samym udowodnimy też istnienie punktów stałych operatora $A$. Jeśli bowiem $B[m]=m$, to także $H(m) = H \circ B[m] = H \circ H^{(-1)} \circ A \circ H (m) = A[H(m)]$, skąd wynika, że $H(m) \in \mathrm{Fix}(A)$. Zwróćmy uwagę, że sprzężenie operatorów pozwala tym samym bardzo prosto przenosić własności jednego operatora na drugi.
 

\begin{tw}\label{Tarski2}
	Przekształcenie $B$ jest ciągłe w topologii porządkowej. Istnieje element $n \in \mathcal{M}$, dla którego zachodzi $B[n] \leq n$. Nadto, iterując przykładanie operatora $B$ do elementu $n$ uzyskujemy ciąg zbieżny w normie supremum do punktu stałego $B$, który jest największym elementem w podkracie $\mathrm{Fix}(B)$ kraty $\mathcal{B}$. 	
\end{tw}

Dowód odkładamy do dodatku matematycznego (patrz: Dowód \ref{proof_Tarski}).
W tym miejscu nakreślimy jednak jego geometryczne uzasadnienie. Rysunek \ref{rys3} obrazuje główną myśl dowodu: jest nim iterowanie działania operatora na elemencie, o którym wiemy, że z całą pewnością znajduje się pod przekątną iloczynu kartezjańskiego $\mathcal{M} \times \mathcal{M}$. W kolejnych krokach rozwiązanie powinno być w określonym sensie coraz bliżej punktu stałego. Twierdzenie \ref{Tarski2} orzeka, że zbieżność następuje w normie supremum. 

\begin{figure}
	\begin{center}
		\includegraphics[scale=.7]{order_convergence1.pdf} 
		\caption[Zbieżność w topologii porządkowej i w topologii supremum -- przypadek wielu równowag]{Zbieżność w topologii porządkowej i w topologii supremum -- przypadek wielu równowag. \\ \rightline{\footnotesize{Źródło: opracowanie własne.}}}\label{rys3}
	\end{center}
\end{figure}

Z powyższego twierdzenia wynika, że zbiór punktów stałych obu operatorów nie jest pusty. Mogłoby się jednak zdarzyć, że jedynym punktem stałym jest zero odpowieniej kraty. Przejdźmy więc do kwestii istnienia funkcji konsumpcji, która czyniłaby za dość równaniu Eulera \ref{Euler_easy}, była elementem jednej z rozpatrywanych krat oraz była ściśle niezerowa na $ K_{++}$. W przypadku gospodarek stochastycznych jej istnienie można udowodnić przy dodatkowych założeniach 

\begin{ass}\label{ass_on_R2}
	Istnieje pewien poziom kapitału do pracy $ x_0 \in K $ taki, że dla wszystkich stanów $ s $ zachodzi $ F(x_0,s) > x_0 $ oraz $ \mathcal{E}_s \bigl \{ R(x_0,s')	\bigl \} \leq 1  $.
\end{ass}

Zwróćmy uwagę, że w przypadku deterministycznym (rezygnując z szoków) powyższe założenie wynikałoby z założenia \ref{ass on prod}, bowiem wówczas funkcja $ F $, jako ściśle wklęsła i ściśle rosnąca i spełniająca $ F(\bar{x}) = \bar{x} $ poniżej $ \bar{x} $ miałaby graf nad odcinkiem łączącym środek układu współrzędnych z punktem $ (\bar{x}, \bar{x}) $. Drugie stwierdzenie wynika stąd, że w punkcie $ \bar{x} $ w równowadze zachodzić musi:

\begin{equation*}
	\frac{1}{m(\bar{x})} = \frac{R[F(\bar{x})-H(m(\bar{x}))]}{m[F(\bar{x})-H(m(\bar{x}))]} \geq \frac{R[F(\bar{x})-H(m(\bar{x}))]}{m(F(\bar{x}))} = \frac{R[F(\bar{x})-H(m(\bar{x}))]}{m(\bar{x})}\,,
\end{equation*} 

skąd wynika, że $ R[\bar{x}] \leq R[F(\bar{x})-H(m(\bar{x}))] \leq 1 $. Argument ten  motywuje naturalność powyższego założenia. Dysponując powyższym założeniem łatwo pokazać, że żądana równowaga istnieje: okazuje się, że wystarczy iterować przykładanie operatora $ B $ poczynając od funkcji $ H^{(-1)}\circ F $.

\begin{lemat}\label{non-zero_fixed_point}
	Przy dodatkowym założeniu \ref{ass_on_R} istnieje ściśle dodatni punkt stały.
\end{lemat}

Dowód powyższego lematu na koniec pracy (patrz: Dowód \ref{proof_existence_of_a_fp}).

Wiemy zatem, że istnieje punkt stały, który jest niezerowy na $K_{++}$. Nadal jednak potrafimy wyznaczyć tylko i wyłącznie najwyższy punkt stały. W Dowodzie \ref{A_acts_on_C_F} lematu \ref{on A} stwierdziliśmy, że przykładanie operatora $A$ do funkcji, której nośnik jest przedziałem zawartym w $K_{++}$, powoduje częściowe podniesienie się części jej grafu. Tym samym zachodzi podejrzenie, że $0$ nie jest równowagą stabilną i być może poszukiwania możaby rozpocząć również od tak wybranego punktu z dziedziny $B$. Owo spostrzeżenia stają się tym bardziej prawdziwe w kontekście twierdzeń zapewniających istnienie dokładnie jednej równowagi.  


Za Krasnoselskim i Zabrejkiem~
%\footnote{\begin{otherlanguage}{russian} М.А. Красносельский, П.П. Забрейко \end{otherlanguage} to znani rosyjscy specjaliści z zakresu analizy nieliniowej. Patrz: \citet[][str. 382]{Krasno}}
 zauważmy, że w przypadku skalarnym bardzo łatwo znaleźć warunki gwarantujące jedyność niezerowej równowagi: wystarczy, aby dana funkcja była ściśle wklęsła oraz jej graf powinien być w pewnym miejscu nad przekątną układu współrzędnych \footnote{Alternatywnie: powinna być ściśle wypukła i jej graf w pewnym miejscu powinien znajdować się pod przekątną}. Okazuje się, że powyższe pojęcia można uogólnić na przypadek dowolnych operatorów określonych na przestrzeniach Banacha. W oparciu o pracę Krasnoselskiego i Zabrejki (\citeyear{Krasno}), \citet{Coleman1} opracował warunki zapewniające jednoznaczność niezerowej równowagi. Nie eksponuje on jednak pewnej szczególnej własności operatora, na której mogłoby zależeć teoretykom ekonomii. Wprowadźmy zatem nowe pojęcia i przedyskutujmy je

\begin{definicja}\label{pseudo_concavity}
	Monotoniczny operator $B: \mathcal{M} \rightarrow \mathcal{M}$ nazywamy pseudo-wklęsłym, jeśli dla dowolnego $m \in \mathcal{M}$ ściśle dodatniego na $K_{++}$ oraz dowolnego skalara $t \in (0,1)$ zachodzi dla dowolnego $x > 0$ oraz $s \in S$
	
\begin{equation}
	B[tm](x,s) > t\bigl( B[m](x,s) \bigl)\,,
\end{equation}	

czyli, w skrócie --- $B[tm] > tB[m]$	
\end{definicja}

Tym samym widzimy, że $B$ spełnia wtedy pewną nierówność w rozpatrywanej algebrze funkcji. Tego pojęcia nie chcemy modyfikować.

\begin{definicja}[$x_0$-monotoniczność według Colemana]\label{x0_monotonicity}
	
	Operator $B : \mathcal{M} \rightarrow \mathcal{M}$ nazywamy $x_{0}$-monotonicznym według Colemana, jeżeli 
	
\begin{enumerate}
	\item{jest monotoniczny,}
	\item{dla dowolnej $m \in \mathrm{Fix}(B)$ ściśle dodatniej na $K_{++}$ istnieje $x_0(m) \in K$ takie, że dla dowolnego $x_1$ z przedziału $[0,x_0(m)]$ oraz dowolnego $n \in \mathcal{M}$ takiego, że dla wszystkich $x \geq x_1$ oraz $s \in S$ zachodzi implikacja

\begin{equation*}
m(x,s) \geq n(x,s)  \Longrightarrow m(x,s) \geq B[n](x,s)\,.
\end{equation*}	
	} 
\end{enumerate}	
\end{definicja}

Powyższą definicję zastępujemy pozornie mniej ogólną. Jest ona również bardziej stosowna do przeprowadzanego wnioskowania o jednoznaczności rozwiązań. Niech na razie $\mathcal{W}$ to pewna klasa funkcji, oraz $W$ to podzbiór $\mathbb{R}_{++}$ 


\begin{definicja}[$x_0$-monotoniczność]\label{my_x0_monotonicity}
	
	Operator $B : \mathcal{W} \rightarrow \mathcal{W}$ nazywamy $x_{0}$-monotonicznym, jeżeli 
	
\begin{enumerate}
	\item{jest monotoniczny,}
	\item{dla dowolnej $m \in \mathrm{Fix}(B)$ ściśle dodatniej na $W_{++}$ istnieje $x_0 (m) \in W_{++}$ takie, że dla dowolnego $x_1$ z przedziału $[0,x_0(m)]$ oraz dowolnego $n \in \mathcal{M}$ takiego, że dla wszystkich $x \geq x_1$ oraz $s \in S$ zachodzi implikacja
	
\begin{equation*}
m(x,s) \geq n(x,s)  \Longrightarrow m(x,s) \geq B[n](x,s)\,.
\end{equation*}	
	
W skrócie: $m \geq n$ pociąga $m \geq B[n]$ na każdym prostokącie $[x_1, \sup W] \times S$, gdy $x_1 \in [0,x_0 (m)]$.}
	\item{
dla pewnej pary $m , n \in \mathrm{Fix}(B)$ niezerowych na $W_{++}$ oraz odpowiadających im punktów $x_0 (m)$ i $x_0 (n)$ zdefiniowanych powyżej, połóżmy $x_0 \equiv \max\{ x_0 (m), x_0 (n)\}$. Dla każdego $s \in S$ funkcja $\frac{m(x,s)}{n(x,s)}$ przyjmuje dla $x \in [x_0, \sup W]$ wartości z pewnego skończonego odcinka zawartego w $\mathbb{R}_{++}$. }\label{cond3}
\end{enumerate}	
\end{definicja}

Kładąc $W \equiv K$ oraz $\mathcal{W} = \mathcal{M}$ warunek \ref{cond3} spełniony jest automatycznie dla dowolnych dwóch $m,n \in \mathrm{Fix}(B)$, bowiem $\Bigl\{K \cap [x_0, \infty) \Bigl\}\times S$ to zbiór zwarty, a rozpatrywane funkcje $m(\circ,s), n(\circ,s)$ są ciągłe~\footnote{Formalnie przyjmujemy, że zbiorami otwartymi na $S$ są wszystkie jego podzbiory - jest to standardowa topologia na zbiorach dyskretnych. Tym samym zbiorami otwartymi w dziedzinie powyższych funkcji są podzbiory $U \times A$ iloczynu kartezjańskiego, gdzie $U$ to standardowy zbiór otwarty na prostej w metryce euklidesowej, a $A$ to dowolny podzbiór $S$.} i niezerowe na $[x_0, \infty)$. Wówczas więc definicja pokrywa się z definicją $x_0$-monotoniczności według Colemana. Wprowadzenie dodatkowego warunku można motywować tym, że wielu teoretyków ekonomii zainteresowanych jest przypadkiem, gdy $W \equiv \mathbb{R}_{++}$. W takim przypadku możliwe jest bowiem badanie zjawiska zrównoważonego wzrostu, które zdaje się być potwierdzane przez dane makroekonomiczne. Coleman w artykule z \citeyear{Coleman2} roku zauważa ową możliwość i wyprowadza kryterium na jednoznaczność rozwiązań w przestrzeni funkcyjnej ogólniejszej od rozpatrywanego przez nas zbioru funkcji $\mathcal{M}$, posiadających zwarty nośnik zawarty w $K$. Ów zbiór funkcji to~\footnote{Po poprawieniu błędu wspomnianego w przypisie \ref{dupa}.}:

\begin{equation*}
\mathcal{Z}= \left \{ \begin{matrix}
m: \mathbb{R}_{++} \times S \overset{\mathcal{C}}{\rightarrow} \mathbb{R}_{++} \cr
 \underset{(x,s)\in K\times S}{\forall} 0 < m(x,s) \leq H^{(-1)}(F) \cr
 y > x \Longrightarrow \frac{R(x,s)}{m(x,s)}>\frac{R(y,s)}{m(y,s)}\cr 
\end{matrix} \right \}\,.
\end{equation*}

 Niemniej nie udowadnia on prawdziwości $x_0$-monotoniczności dla tak ogólnego zbioru funkcji i niepoprawnie powołuje się na twierdzenie ze swojego artykułu z \citeyear{Coleman1} roku, gdzie rozpatruje po prostu zbiór $\mathcal{C}_F (K\times S)$ opisany warunkami \ref{C_F}. Tym ciekawsze wydaje się z tej perspektywy wprowadzenie warunku \ref{cond3} w definicji $x_0$-monotoniczności. Pozwala on bowiem przeprowadzić pełne rozumowanie w ogólnym przypadku dla $\mathcal{W} \equiv \mathcal{Z}$ oraz $W = \mathbb{R}_{++}$.

Własność $x_0$-monotoniczności jest silniejszym rodzajem monotoniczności. Dla operatorów $x_0$-monotonicznych potrafimy powiedzieć, że jeśli graf pewnej funkcji $n(\circ, s)$ jest tylko częściowo poniżej grafu rozpatrywanego punktu stałego $m(\circ, s)$, dokładniej -- jest ona poniżej grafu $m_1(\circ, s)$ począwszy od pewnego punktu $x_1 \in [0, x_0 (m)]$, to graf $B[n](\circ, s)$ nadal leży poniżej grafu $m(\circ, s)$ na tym samym przedziale $[x_0, \sup W]$. Obrazuje to Rysunek \ref{rys4}. Zwróćmy uwagę, że nie interesuje nas w ogóle zachowanie funkcji $n$ oraz $B[n]$ przed punktem $x_1$. W szczególności, $m$ oraz $n$ mogą być nieporównywalne w rozpatrywanym porządku, to jest -- gdy ich grafy rozpatrujemy na całym przedziale $W$, tak jak na Rysunku \ref{rys4}, gdzie grafy obu funkcji $m, n$ są raz nad sobą, raz pod sobą. Z drugiej jednak strony, w definicji $x_0$-monotoniczności nie wymagamy, aby istniał jeden punkt $x_0$ dla dowolnego niezerowego na $W_{++}$ punktu stałego operatora $B$: dla dowolnych dwóch punktów stałych możemy mieć dwa różne punkty typu $x_0$~\footnote{Tym samym owa własność operatora powinna mieć inną nazwę, nie sugerującą istnienia jednego tylko punktu $x_0$}.

\begin{figure}
	\begin{center}
		\includegraphics[scale=1]{x0_monotonicity.pdf} 
		\caption[Idea $x_0$-monotoniczności]{Idea $x_0$-monotoniczności: $m$ to punkt stały operatora $B$. Porównujemy z nim dowolną funkcję $n$, której graf na odcinku $[x_1, \sup W]$ na pewno znajduje się pod grafem $m$. Wówczas graf $B[n]$ musi być na tym samym przedziale pod grafem $m$. \\ \rightline{\footnotesize{Źródło: opracowanie własne.}}}\label{rys4}
	\end{center}
\end{figure}

Zauważmy w końcu, że jeżeli operator $B$ jest $x_0$-monotoniczny, to i operator doń sprzężony również jest $x_0$-monotoniczny -- stosujemy powielony już kilka razy poprzednio trik z zapisem przez sprzężenie obu operatorów, często stosowany w dodatku matematycznym \ref{chap_appendice}. Natomiast sprzężenie nie przenosi własności pseudo-wklęsłości: trzeba ją udowadniać dla każdego operatora z osobna. Jednym z ciekawszych aspektów pracy Colemana z \citeyear{Coleman2} roku było zauważenie, że to właśnie operator $B$ jest w naturalny sposób pseudo-wklęsły, a w połączeniu z własnością $x_0$-monotoniczności, posiada on tylko jedną, niezerową na $K_{++}$ równowagę. 

Sformalizujmy teraz wspomniany przez nas wynik. 

\begin{lemat}\label{uniqueness} 
Operator $x_0$-monotoniczny i pseudo-wklęsły $B$ posiada co najwyżej jedną, niezerową na $W_{++}$ równowagę. 
\end{lemat}

Szczegóły zawarte są w dodatku matematycznym (patrz: Dowód \ref{proof_uniqueness}). Poprawiliśmy tym samym wynik Colemana (\citeyear{Coleman2}).

Pozostaje udowodnić, że interesujący nas operator $B: \mathcal{M} \rightarrow \mathcal{M}$ istotnie posiada obie własności. Są to zupełnie standardowe wyniki~\footnote{Patrz: \citet{Coleman1, Coleman2}}.  

\begin{lemat}\label{on_pseudo_concavity_of_B}
	Operator $B: \mathcal{M} \rightarrow \mathcal{M}$ jest pseudo-wklęsły.
\end{lemat}

Szczegóły rachunku zawiera Dowód \ref{proof_quasi_concavity} z dodatku matematycznego.

Aby dowieść $x_0$-monotoniczności $B$ korzysta się z następującego założenia

\begin{ass}\label{ass on R 2} Dla dowolnych $s,s' \in S$ zachodzi
	$$R(0,s')\pi (s' | s) > 1\,. $$
\end{ass}

Wyżej występująca funkcja $\pi$ to funkcja przejścia opisująca proces stochastyczny~\footnote{Patrz rozdział \ref{chap1sec1}}. Zauważmy, że założenie to wydaje się naturalne, ponieważ zwykle przyjmuje się, że $R(0,s')$ to bardzo duża wartość. Zwróćmy też uwagę, że powyższe założenie implikuje, że rozpatrywany proces stochastyczny nie zawiera stanów nieosiągalnych bowiem zakładamy, że $1 < R(0,s') < \infty$, co pociąga, że $\pi(s'|s) > \frac{1}{R(0,s')}$. Posługując się powyższym założeniem wyprowadzamy następujący

\begin{lemat}\label{x0_monotonicity_of_B}
	Operator $B: \mathcal{M} \rightarrow \mathcal{M}$ posiada własność $x_0$-monotoniczności.
\end{lemat}

Szczegóły dowodu zawiera dodatek matematyczny (patrz: Dowód \ref{proof_x0_monotonicity_of_B}). 

Zauważmy, że pierwsza część Dowodu \ref{proof_x0_monotonicity_of_B} w pełni ukazuje geometryczny sens założenia \ref{ass on R 2}. Implikuje ono bowiem istnienie takiego punktu $x_0$ dla którego na prostokącie $[0,x_0]\times S$ zachodzi $m(x,s) \leq H^{(-1)}\bigl( F(x,s) - x\bigl)$. Mówiąc prościej: istnieje otoczenie zera, w którym funkcja m jest zdominowana przez inną funkcję. Zwróćmy uwagę, że wprowadzenie tego ograniczenia było możliwe dzięki temu, że funkcję $R$ mogliśmy uczynić dostatecznie dużą w otoczeniu zera. 

Odwołajmy się po raz ostatni do pracy Colemana (\citeyear{Coleman2}). Zauważyliśmy już, że operator $B: \mathcal{M} \rightarrow \mathcal{M}$ posiada własności $x_0-monotoniczności$ i pseudo-wklęsłości. Własności tych, jak już pokazaliśmy, można dowieść tylko na zbiorze $\mathcal{M}$. Równocześnie zauważmy, że sam operator $B$ można poprawnie określić na znacznie większym zbiorze, podobnym do zbioru $\mathcal{Z}$. Wprowadźmy więc zbiór $\hat{\mathcal{Z}}$ różniący się od $\mathcal{Z}$ tylko dziedziną zawartych w nim funkcji i rozszerzeniem funkcji na wartość w zerze:

\begin{equation*}
\hat{\mathcal{Z}}= \left \{ \begin{matrix}
m: K \times S \overset{\mathcal{C}}{\rightarrow} K \cr
\underset{(x,s)\in K_{++}\times S}{\forall} 0 \leq m(x,s) \leq H^{(-1)}\big(F(x,s)\bigl) \cr
 \underset{s\in S}{\forall}\,\,m(0,s) = 0 \cr
 0 \leq x < y \Longrightarrow \underset{s\in S}{\forall}\,\, \frac{R(x,s)}{m(x,s)}>\frac{R(y,s)}{m(y,s)}\cr 
\end{matrix} \right \}.
\end{equation*}

Bez trudu dowodzimy, że operator $B$ zadany na $\hat{\mathcal{Z}}$ przeprowadza ten zbiór w siebie: wystarczy powielić rozumowania przeprowadzone w tej pracy dla zbioru $\mathcal{M}$. To co pozostawało do tej pory niezauważone w literaturze przedmiotu, to zależność pomiędzy tymi zbiorami funkcji. Prawdą jest bowiem

\begin{lemat}\label{genialna_uwaga}
	$B[\hat{\mathcal{Z}}] \subset \mathcal{M}$.
\end{lemat}

Krótki dowód możemy przytoczyć w tym miejscu:

\begin{proof} Jeśli $z \in \hat{\mathcal{Z}}$, to $B[z]$ jest ściśle rosnąca (patrz: Dowód \ref{A_acts_on_C_F}). Ale jednocześnie, z definicji operatora $B$, spełniona jest zależność

\begin{equation*}
	\frac{1}{B[z](x,s)} = \mathcal{E} \Bigl \{ \Big[\frac{R}{m}\Big] \bigl( F(x,s) - H(B[m(x,s)],s')\bigl) \Bigl\}\,.
\end{equation*}

Prawa strona równania jest funkcją ściśle malejącą z $x$ dla każdego $s$. Ponieważ jednak dla każdego $s'$ funkcja $\frac{R}{m}(\circ, s')$ jest ściśle malejąca, zatem nie może się zdarzyć, aby funkcja $F(x,s) - H(B[m(x,s)]$ była stała lub malejąca z $x$, bo wówczas cała prawa strona równania też będzie stała lub rosnąca, co jest niemożliwe, bo równość jest spełniona dla każdego $x$. Zatem $F(x,s) - H(B[m(x,s)]$ jest ściśle rosnąca. Ale to oznacza, że należy do zbioru $\mathcal{M}$. 
\end{proof}

Ta banalna uwaga pozwala zauważyć, że przy założeniach o istnieniu równowagi w $\mathcal{M}$, równowaga istnieje również w znacznie większym zbiorze $\hat{\mathcal{Z}}$ oraz jest tam wyznaczona jednoznacznie. Cały ten zbiór bowiem już przy pierwszym przyłożeniu do niego operatora $B$ wpada do $\mathcal{M}$, gdzie równowaga jest jedyna. Lemat \ref{genialna_uwaga} można więc postrzegać jako ostateczne domknięcie rozumowania Colemana. Jego znaczenie jest jednak ogromne: ów zbiór zawiera bowiem wszystkie funkcje rosnące $m$ niezależnie od tego, jak szybko rosną. Zawiera również sporo funkcji niekoniecznie rosnących, ale takich, że $\frac{R}{m}$ pozostaje ściśle rosnąca, choć takie funkcje zdecydowanie nie mogą posiadać ogromnych lokalnych wahań. Wynika to z tego, że stosunkowo łatwo sprawić, żeby funkcja $R$, która jest ściśle malejąca, przy dzieleniu przez inną funkcję już malejąca nie była: krzywizna $R$ nadaje tu oczywiste ograniczenia. 


	Ostatnią uwagą, jaką na jaką chcielibyśmy zwrócić uwagę w tym rozdziale jest to, założenia dotyczące postaci funkcji $R$ zdają się być istotne, jeśli rozpatrywać zagadnienie istnienia równowagi. Oto bowiem, jeśli rozpatrzymy $R \equiv 1$ i będziemy badali zachowanie operatora $B$ na przestrzeni funkcji ściśle rosnących z pasma $\mathcal{M}\setminus\{ 0_{\mathcal{M}}\}$, to wtedy będzie miał miejsce następujący lemat:

\begin{lemat}\label{no_solution}
	Równanie $m(x) = m\big( F(x) - H(m(x)) \big)$ nie posiada rozwiązania w podzbiorze funkcji ściśle rosnących w zbiorze $\mathcal{M}\setminus\{ 0_{\mathcal{M}}\}$.
\end{lemat}

Szczegóły dowodu zawiera dodatek matematyczny (patrz: Dowód \ref{proof_x0_monotonicity_of_B}).


%%%%%%%%%%%%%%%%%%%%%%%%%%%%%%%%%%%%%%%%%%%%%%%%%%%%%%%%%%%%%%%%%%%%%%%%%%%%%%%55%%%%


\chapter{Różniczkowalność operatorów}\label{chap_diff}

W rozdziale zostaną przedstawione nowe wyniki dotyczące własności operatorów, którymi można się posłużyć do otrzymania rozwiązania zasadniczego problemu naszej pracy, czyli znalezienia funkcji rozwiązującej równanie \ref{Euler_easy}. Wedle naszej wiedzy, nikt nie podjął do tej pory kwestii związanych z różniczkowalnością owych operatorów. Celem tego rozdziału jest przedstawienie warunków, przy których przedstawione operatory będą w odpowiednim sensie różniczkowalne. Przedstawione zostaną również wnioski, które dzięki tej własności możemy uzyskać. 

W rozdziale tym skupimy się na przypadku gospodarek niestochastycznych, przez wzgląd na prostotę zapisu postaci analitycznej rozpatrywanych wzorów. Dodajmy, że tego rodzaju postępowanie nie jest obce w literaturze przedmiotu --- stosują je np. \cite{Reffett}. 

\section{Operatory zadane na funkcjach odwrotnych}

W rozdziale tym przedstawimy intuicję stojącą za obranym w tej pracy nowym podejściem do problemu. Dlatego też ograniczymy się tu do przekształceń analitycznych pozostawiając dookreślenie dziedzin badanych operatorów na sam koniec rozdziału.

Rozpatrzmy raz jeszcze problem sprzężony w postaci równości dwóch funkcji

\begin{equation*}
\frac{1}{y} = \mathcal{E}_s \Bigl \{ \frac{R(F(x,s) - H(y))}{m(F(x,s) - H(y))}, s' \Bigl \}\,. 
\end{equation*}

Jeśli pominiemy część stochastyczną, to w powyższym zapisie znikną parametry $s$, $s'$ oraz operator wartości oczekiwanej. Tak uproszczone równanie możemy łatwo zapisać w równoważnej postaci 

\begin{equation*}
y= \Bigl [\frac{m}{R}\Bigl] \Big ( F(x) - H(y) \Big) \,.
\end{equation*}


Zauważmy, że wyrażenie $\frac{m}{R}$ traktowane łącznie jako funkcja względem pewnego argumentu, jest funkcją odwracalną. Wynika to z uwag poczynionych w rozdziale \ref{section_Coleman_operator}. Dla ustalenia uwagi możemy wręcz nazwać tę funkcję, przyjmując $\Omega_m \equiv \frac{m}{R}$. Możemy teraz dokonać kluczowej dla tego rozdziału obserwacji: miast definiować $y$ jako rozwiązanie powyższego problemu uzależnione od $x$ i $m$~\footnote{Czyli, {\it de facto}, zdefiniowania za Colemanem operatora $B: \mathcal{M}\rightarrow\mathcal{M}$.}, możemy rozpatrzeć $x$ jako rozwiązanie powyższego problemu uzależnione od $y$ i funkcji $m$. Naturalność i wykonalność tego podejścia wynika z prostej obserwacji: jedno działanie operatora $B$ na zbiorze $\mathcal{M} \setminus \{0_{\mathcal{M}}\}$ dostarcza nam zbioru funkcji ściśle rosnących (patrz: Dowód \ref{A_acts_on_C_F}). Możemy się zatem skupić na zbiorze funkcji $B\big[ \mathcal{M} \setminus \{0_{\mathcal{M}}\}\big]$. Interesująca nas funkcja $x = x(y,m)$ powinna być funkcją odwrotną do pewnego elementu z tego zbioru. Dokładniej, powinna to być funkcja odwrotna do $B[m]$, gdzie $m\in B\big[ \mathcal{M} \setminus \{0_{\mathcal{M}}\}\big]$. Z definicji, funkcja $x = x(y,m)$ czyni zadość wyrażeniu

\begin{equation*}
y= \Omega_m \Big ( F(x(y,m)) - H(y) \Big). 
\end{equation*}

Po prostych operacjach, polegających na przyłożeniu do obu stron funkcji odwrotnej do $\Omega$, uporządkowaniu wyrażeń i przyłożeniu funkcji odwrotnej do $F$~\footnote{Przypomnijmy, że na mocy założeń nałożonych na $F$ jest to funkcja odwracalna.}, otrzymujemy

\begin{equation*}
x(y,m) = F^{(-1)} \Bigl \{ H(y) + \Omega_m^{(-1)}(y) \Bigl \}.
\end{equation*} 

Tym samym, porzucając już chwilowo przyjęte oznaczenie $\Omega_m$, w naturalny sposób możemy zdefiniować operator nierozłącznie powiązany ze sprzężonym operatorem Colemana

\begin{equation}\label{not_so_easy_to_differentiate}
C[h](y) \equiv F^{(-1)} \Bigl \{ H(y) + \Big[ \frac{h^{(-1)}}{R} \Big]^{(-1)}(y) \Bigl \}\,.
\end{equation}

Rozpatrzenie funkcji $h^{(-1)}$ miast $m$ w powyższym wyrażeniu nie jest przypadkowe: tylko wtedy powyższe zagadnienie sprowadza się do poszukiwania punktu stałego. Punkt stały $ C $ jest oczywiście funkcją odwrotną do poszukiwanej w modelu Colemana funkcji konsumpcji, $ C[h] = h \equiv m_{*}^{(-1)} $. Funkcja $m_{*}^{(-1)}$ istnieje, bowiem wykazaliśmy, że $ m_{*} $ jest ściśle rosnąca. 


Sporej gimnastyki wymaga bezpośrednia interpretacja ekonomiczna opisywanych obiektów. Jeżeli bowiem $m \in \mathcal{M}$ było funkcją mierzącą brak użyteczności z danego poziomu konsumpcji, to teraz $h = m^{(-1)}$ należy uznać za poziom konsumpcji, który konsument musi osiągnąć aby odczuwać dany poziom braku zadowolenia. Pozostajemy przy poglądzie, że w tym przypadku intuicja ekonomiczna powinna ustąpić miejsca abstraktowi matematycznemu, który dostarczy pełniejszego wglądu w naturę problemu. Mimo to, trzeba zdawać sobie sprawę, że wartości $h$ to w istocie pewne rosnące krzywe na prostej wyznaczającej poziomy konsumpcji.


Zauważmy, że jest gdyby nie wykorzystanie dwukrotnie operacji inwersji funkcji, ów operator byłby operatorem Niemyckiego. Faktycznie: dla gospodarki ze współczynnikiem R stale równym jedności otrzymalibyśmy następujące wyrażenie:

\begin{equation}\label{easy_to_differentiate}
\overline{C}[h](y) \equiv F^{(-1)} \Bigl \{ H(y) + h(y) \Bigl \}\,,
\end{equation}

które jest dokładnie postaci nieautonomicznego operatora Niemyckiego

\begin{equation*}
\overline{C}[h](y) = n(y, h(y)).
\end{equation*}


Kwestią dziedziny i przeciwdziedziny funkcji $n$ zajmiemy dalej. Ważne jest, że istnieje rozbudowana teoria matematyczna zajmująca się operatorami Niemyckiego~\footnote{Patrz: \citet{Appell,Granas_FPT,Goebel,Lanza1,Lanza2, Nugari1}.}. Teoria owa została rozbudowana głównie w odpowiedzi na problemy stawiane przez teorię cząstkowych równań różniczkowych oraz różnych problemów wariacyjnych. Jednym z problemów o fundamentalnym znaczeniu właśnie w problemach wariacyjnych jest poszukiwanie punktów ekstremalnych. Poszukiwanie przejrzystych kryteriów na to, czy dana funkcja jest ekstremalą operatora, zawiodło matematyków w stronę uogólnienia pojęcia różniczkowalności na przekształcenia pomiędzy abstrakcyjnymi przestrzeniami Banacha. Aby jednak posiadać odpowiednik twierdzenia Rolle'a w ogólnym przypadku, trzeba jednak zdefiniować różniczkę. Spośród wielu pojęć, drogę do powszechności utorowało sobie pojęci różniczki w sensie Fr\' echeta, patrz: \ref{preliminaries}~\footnote{Jeszcze w latach '50-ych prowadzona była akademicka dysputa na temat, którą definicję należałoby uznać za lepszą. Zwolennicy ogólności z różnych powodów uznawali, że to badanie różniczkowalności w sensie G\^ateau jest ciekawsze, nawet w przypadku przestrzeni Banacha tak prostych, jak $\mathbb{R}^n$. Dopiero rozpowszechnienie się badań z zakresu mechaniki płynów doprowadziło do powszechnego użycia pochodnej w sensie Fr\' echet.}.  


	W świetle tej dyskusji naturalnym pytaniem jest kwestia różniczkowalności w sensie Fr\' echet operatora $\overline{C}$. Znacznie ciekawszym pytaniem jest jednak, czy potrafimy zbadać różniczkowalność bardziej ogólnego operatora $C$, którego zdefiniowanie okazało się być naturalnym ominięciem przeszkody, jaką jest brak analitycznego wzoru na operator Colemana $B$~\footnote{Gdyż ten jest zdefionany {\it implicit\'e}.}. Głównym wynikiem tej pracy jest, że faktycznie ów problem posiada rozwiązanie. 

Wprowadzenie uproszczonego operatora $\overline{C}$ było nam potrzebne również i z innych powodów niż rozbudowanie intuicji odnośnie tego, że problem można by sprowadzić do badania operatorów Niemyckiego. 

Nadto, zdefiniowanie powyższego operatora podsuwa pomysł na dalszą faktoryzację operatora $C$ i stopniowego sprawdzania różniczkowalności jego składowych, oraz na ostateczne złożenie wyników cząstkowych powołując się na ogólne twierdzenie o pochodnej funkcji superponowanej \citet[][twr. VII.4.2, str. 152]{Maurin}.


Wprowadźmy zatem operator $\Upsilon$ dany wzorem $\Upsilon[m](x)= \frac{m(x)}{R(x)}$. Wprowadźmy również dwa operatory odwracania funkcji $\mathcal{J}_1$ i $\mathcal{J}_2$ takie, że $\mathcal{J}_i [g] = g^{(-1)}$. To, że musimy stosować formalnie dwa różne operatory wyniknie w toku przeprowadzonego dalej rozumowania; uprzedzając fakty dodajmy już teraz, że operatory te będą różniły się dziedzinami i przeciwdziedzinami na których będą określone. Przy powyższych oznaczeniach możemy zapisać operator $C$ w następujący sposób

\begin{equation*}
	C[h] \equiv \Bigl[\overline{C} \circ \mathcal{J}_2 \circ \Upsilon \circ \mathcal{J}_1 \Bigl][h]\,.
\end{equation*}

Nasze rozumowanie będzie teraz następująco: przede wszystkim określimy dobrą dziedzinę dla operatora $C$. Następnie udowodnimy różniczkowalność składowych $\mathcal{J}_i, \Upsilon, \overline{C}$ operatora $C$.


%%%%%%%%%%%%%%%%%%%%%%%%%%%%%%%%%%%%%%%%%%%%%%%%%%%%%%%%%%%%%%%%%%%%%%%%5


\section{W poszukiwaniu odpowiedniej przestrzeni}\label{chap_right_space}


W rozdziale tym doprecyzujemy szczegóły dotyczące uprzednio zdefiniowanych operatorów. Zastanówmy się najpierw nad tym, co może być uznane za rozsądne rozwiązanie problemu zdefiniowanego przez

\begin{equation*}
y= \Bigl [\frac{m}{R}\Bigl] \Big ( F(x) - H(y) \Big)\,.
\end{equation*}

Naturalny sposób postrzegania tego problemu poznaliśmy w rozdziale \ref{section_Coleman_operator}. Zdefiniujmy funkcję $\theta: \mathcal{Q} \rightarrow \mathbb{R}$ daną wzorem

\begin{equation*}
\theta(x,y,m) = \Bigl [\frac{m}{R}\Bigl] \Big ( F(x) - H(y) \Big) - y\,.
\end{equation*}

Cała rzecz polega na badaniu przeciwobrazu zera, $\theta^{-1}[\{ 0 \}]$, w dziedzine $\mathcal{Q}$, którą teraz trzeba dobrze zdefiniować. Przede wszystkim żądamy, żeby dla każdej funkcji $m$ ów przeciwobraz też był funkcją. Uwagi poczynione w poprzednim rozdziale pozwalają nam stwierdzić, że dobrze jest, jeśli ów przeciwobraz będzie funkcją nie tylko zmiennej $x$, ale również i zmiennej $y$, czyli naraz można zapisać $y=y(x,m)$, jak i $x=x(y,m)$ (czyli obie funkcje są odracalne). Wtedy bowiem możemy zapisać {\it explicit\' e} operator $C$, bowiem nabiera sensu rozpatrywanie operatorów odwracania funkcji $\mathcal{J}_{i} (h) = h^{(-1)}$. Widzimy więc, że jeżeli chcemy w ogóle liczyć na różniczkowalność operatora $C$, to musimy ze zbioru $\hat{\mathcal{Z}}$, poznanego pod koniec rozdziału \ref{sec-lattice}, wybrać podzbiór jego elementów odwracalnych. Uzyskane w ten sposób funkcje odwracamy i na takim zbiorze możemy faktycznie zdefiniować operator $C$. Tym samym określiliśmy pierwszego, najbardziej ogólnego kandydata na dziedzinę $C$ --- zbiór $\mathcal{H}$ zadany następująco

\begin{equation*}
\mathcal{H}= \left \{ h : \underset{ h^{(-1)}}{\exists} \, \mathrm{oraz}\, \underset{ m \in \hat{\mathcal{Z}}}{\exists}  h^{(-1)} \equiv m \right \}.
\end{equation*} 

Zwróćmy uwagę, że powyższa definicja wydaje się nie być standardową konstrukcją matematyczną: zwykle zbiór definiuje się jako podzbiór pewnej większej klasy elementów. Zbiór $\mathcal{H}$ nie jest jednak podzbiorem żadnej standardowej przestrzeni funkcyjnej. Problem polega na tym, że jeżeli patrzeć na elementy tego zbioru jak na funkcje, to w jego skład wchodzą obiekty o różnych dziedzinach: ta oczywista obserwacja wynika z tego, że na odwracanie funkcji możemy patrzeć jak na odbicie jej grafu względem przekątnej iloczynu kartezjańskiego dziedziny i zbioru, w którym znajduje się obraz. Widać to m.in. na rysunkach \ref{H_set_1} i \ref{H_set_2}. W naszym przypadku grafy funkcji z $\mathcal{H}$ zawarte są w prostokącie $[0, H^{(-1)}(\bar{x})]  \times K$. Potencjalnie zachodzić mogą dwa przypadki: punkt $H^{(-1)}(\bar{x})$ może się znaleźć poniżej przekątnej lub na niej (patrz: rysunek \ref{H_set_1}), lub ponad przekątną (patrz: rysunek \ref{H_set_2}). Szczęśliwie oba przypadki możemy rozpatrywać dalej łącznie.

\begin{figure}
	\begin{center}
		\subfloat[$\bar{x} \vee H^{(-1)}(\bar{x}) = \bar{x}$]{\label{H_set_1} \includegraphics[scale=1]{H_set_1.pdf} }
		\subfloat[$\bar{x} \vee H^{(-1)}(\bar{x}) = H^{(-1)}(\bar{x})$]{\label{H_set_2} \includegraphics[scale=1]{H_set_2.pdf} }
	\end{center}
	\caption[Przestrzeń $\mathcal{H}$ funkcji dla operatora $C$]{Przestrzeń $\mathcal{H}$ funkcji dla operatora $C$: dwie możliwe sytuacje w zależności od zachowania funkcji $H^{(-1)}$.
	\\ \rightline{\footnotesize{Źródło: opracowanie własne}}}
\end{figure}

 Interesujące nas funkcje ze zbioru $\hat{\mathcal{Z}}$ to dokładnie funkcje, których grafy zawarte są w paśmie $\mathcal{P} \equiv \{ (x,y): x \in K, y \in [0,H^{(-1)}(F(x))]\}$ oraz ściśle rosnące --- tylko one bowiem są w tym zbiorze jednocześnie ciągłe i odwracalne. Elementy zbioru $\mathcal{H}$ są ich odbiciami względem przekątnej. 
 
 	Ponieważ, jak już zauważyliśmy, elementy $\mathcal{H}$ rozpatrywane razem nie tworzą podprzestrzeni znanych z analizy funkcjonalnej przestrzeni funkcyjnych, nasuwa się pytanie o możliwość sensownego zdefiniowania na tym zbiorze operacji różniczkowania. Ponieważ jednak operator $C$ faktoryzuje się w określony sposób, $C = \overline{C} \circ \mathcal{J}_2 \circ \Upsilon \circ \mathcal{J}_1 $, widzimy zatem, że już po pierwszym przyłożeniu operatora $\mathcal{J}_1$ do zbioru $\mathcal{H}$ nasze elementy wejdą w skład pewnej przestrzeni funkcji określonych na zbiorze $K$. Dlatego też do operatora $\Upsilon$ można zastosować standardowe metody stosowane przy różniczkowalności nieautonomicznych operatorów Niemyckiego. Pewne dalsze ograniczenia na dziedziny pozostałych operatorów muszą być jednak nałożone. 
 	
 	Jest w miarę oczywiste, że wszystkie problemy narastają przez wprowadzenie operacji odwracania funkcji: to właśnie ona sprawia, że dwie funkcje określone na tej samej dziedzinie mogą zostać zamienione w dwie funkcje o różnych dziedzinach. Zauważmy jednakże, że problem ten można rozwiązać poprzez nałożenie dalszych restrykcji na dziedzinę $C$. Rozważmy bowiem pewną funkcję $h^{(-1)} \in \mathcal{H}$ (indeksujemy elementy zbioru $\mathcal{H}$ superskrytpem $^{(-1)}$ dla podkreślenia tego, że są to funkcje odwrotne do funkcji, które stanowiły pierwotny obiekt zainteresowania). Następnie wprowadźmy relację równoważności w zbiorze $\mathcal{H}$ zadaną tak: $f^{(-1)} \sim h^{(-1)}$ wtedy i tylko wtedy, gdy obrazy $f$ i $h$ są takie same. W terminach funkcji  wchodzących w skład zbioru $\mathcal{H}$ oznacza to ograniczenie uwagi tylko do funkcji określony na tej samej dziedzinie. Tym samym w zbiorze $\mathcal{H}$ wprowadziliśmy podklasy zbiorów, indeksowane górnym punktem zaczepienia grafu (patrz: rysunki \ref{np} oraz \ref{p}). W każdej klasie równoważności względem relacji $\sim$ łatwiej nam będzie zdefiniować operację różniczkowania. Zauważmy bowiem, że dzięki takiemu ograniczeniu potrafimy powiedzieć, co oznacza różnica $\mathcal{J}[f+h]-\mathcal{J}[f]$. Rysunek \ref{np} pokazuje zaś, że bez takiego ograniczenia operacja algebraicznej różnicy funkcji pozbawiona by była sensu: nie wiedzielibyśmy jak ją zdefiniować w żółtym pasie o czerwonej obwódce, dokładnie dlatego, że funkcja $f^{(-1)}$ jest zdefiniowana również dla argumentów, dla których nie jest jest zdefiniowana funkcja $(f+h)^{(-1)}$. 

\begin{figure}
\begin{center}
		\subfloat[Nieprawidłowo]{\label{np} \includegraphics[scale=.8]{improper_diff.pdf} }
		\subfloat[Prawidłowo]{\label{p} \includegraphics[scale=.756]{proper_diff.pdf} }
	\end{center}
	\caption[Operacja odwracania funkcji]{Operacja odwracania funkcji: przykład nieprawidłowo i prawidłowo zdefiniowanego działania algebraicznego w algebrze funkcji.
	\\ \rightline{\footnotesize{Źródło: opracowanie własne}}}
\end{figure}
 	
 	Powiedzmy, że interesuje nas teraz różniczkowalność $\mathcal{J}_i$ w określonej funkcji $h_0 \in \mathcal{H}$. W świetle powyższych rozważań, zagadnienie to trzeba rozpatrywać na podzbiorze $[h_0]_{\sim} \subset \mathcal{H}$. Podobnie, jeżeli chcielibyśmy rozpatrzeć dalej pojęcie różniczki w innej funkcji $h_1 \in \mathcal{H}$, takiej że $h_1 \not \sim h_0$, to uwagę naszą musielibyśmy przenieść na inną klasę równoważności, mianowicie $[h_1]_{\sim}$. Wydaje się ważne podkreślenie tego faktu, bowiem jest to sytuacja diametralnie różna od znanej przy standardowym podejściu do problemu dziedziny różniczkowalności. Różnica ta nie spowoduje dalej jednak żadnych problemów formalnych. 
 
 	Niestety orzeczenie różniczkowalności operatorów $\mathcal{J}_{i}$ wymaga wprowadzenia dalszych jeszcze ograniczeń: spośród obiektów $\mathcal{H}$ wybrać będziemy musieli te, które są w odpowiednim stopniu różniczkowalne. Okazuje się bowiem, że już na poziomie wzorów analitycznych różniczka operatora $\mathcal{J}_{i}$ jest nierozerwalnie związana z różniczką funkcji, w której jest on ewaluowany. Nie wchodząc w techniczne szczegóły, które dokładnie zostały wyłożone w dodatku matematycznym (patrz: Dowód \ref{proof_macho_geek_ze_mnie}), dodajmy tylko, że wynika to z cytowanego już lematu Omega (patrz: lemat \ref{First diff}): funkcja definiująca działanie z lewej strony musi być bowiem różniczkowalna, żeby różniczkowalny był indukowany przez nią operator Niemyckiego. Operator Niemyckiego pojawia się w naturalny sposób, bowiem dowód różniczkowalności $\mathcal{J}_i$ korzysta z ogólnego twierdzenia o funkcji uwikłanej w przestrzeniach Banacha \citep[ogólne twierdzenie o funkcji uwikłanej --- patrz:][]{Maurin}. Ponieważ operator $C$ w naturalny sposób faktoryzuje się tak, że w jego skład wchodzą dwa operatory odwracania, zatem będziemy musieli ograniczyć naszą uwagę do funkcji dwukrotnie różniczkowalnych. Dzięki temu ograniczeniu możemy jednak również poczynić szereg zupełnie naturalnych założeń o postaci pojęcia bliskości w tak zdefiniowanej przestrzeni funkcyjnej: możemy wprowadzić topologię podprzestrzeni przestrzeni Banacha funkcji dwukrotnie różniczkowalnych (zobacz: dodatek matematyczny \ref{preliminaries}) na zbiorze $[h_0]_{\sim}$. Pojęcie różniczki na opisanym przez nas zbiorze $[h_0]_{\sim}$ będziemy mogli doprecyzować dzięki znalezieniu jednoznacznie wyznaczonego rozszerzenia naszego operatora na pewnym otwartym otoczeniu $[h_0]_{\sim}$ --- otwartym właśnie w sensie normy $||\circ||_{\mathcal{C}^{2}}$. 
 	
Kwestia jednoznaczności różniczki jest problemem dobrze zbadanym na otwartych podzbiorach przestrzeni Banacha: dlatego właśnie szukamy otwartego otoczenia zbioru $[h_0]_{\sim}$. W przestrzeni funkcji ciągłych wyposażonych w normę supremum nie znajdziemy otwartego zbioru punktów, które byłyby funkcjami odwracalnymi. Jest to oczywiste i wynika z postaci kul w tej przestrzeni: kula o środku w danej funkcji to po prostu pasmo o szerokości równej średnicy kuli, w którym leżą wykresy funkcji. Biorąc kulę o środku w funkcji odracalnej łatwo znajdziemy funkcję, która nie jest odwracalna --- patrz: rysunek \ref{non_open}. Nadto, jak już wspomnieliśmy, nie będziemy w stanie zdefiniować różniczki dla operatora $\mathcal{J}_i$, jeżeli w jego dziedzinie będą funkcje nieróżniczkowalne. 

\begin{figure}
	\begin{center}
		\includegraphics[scale=1]{non_open.pdf} 
		\caption[Zbiór funkcji odwracalnych nie jest otwarty w normie supremum]{Zbiór funkcji odwracalnych nie jest otwarty w normie supremum: świadczy o tym obecność funkcji $h$ w paśmie wokół funkcji $f$. \\ \rightline{\footnotesize{Źródło: opracowanie własne.}}}\label{non_open}
	\end{center}
\end{figure}

	Aby móc ograniczyć się do zbioru funkcji różniczkowalnych potrzebujemy jeszcze drobnego założenia na funkcję $R$: dokonamy delikatnego wzmocnienia założenia \ref{ass_on_R}. Na razie wiemy, że $R(s)$ ściśle maleje ze swoim argumentem $s$. Załóżmy, że jeśli jest to funkcja różniczkowalna, to dodatkowo jej pochodna jest ściśle mniejsza od zera. 

\begin{ass}
\begin{equation}\label{diff_of_R}
	\underset{w \in K}{\forall}R'(w) < 0
\end{equation}
\end{ass} 
 
Uzbrojeni w powyższe założenie dowodzimy następujący lemat odwołujący się jeszcze do operatora $B$ 
 
\begin{lemat}\label{diff_of_elements_of_M_with_C}
	Dla funkcji $m \in \Bigl(\mathcal{M}\setminus\{ 0_{\mathcal{M}}\}\Bigl)\cap \mathcal{C}^{m}(K)$, funkcja $B[m]$ też jest klasy $\mathcal{C}^{m}(K)$, o ile funkcje $R, F, H$ są klasy $\mathcal{C}^{m}(K)$.
\end{lemat} 

Dowód powyższego lematu odkładamy do dodatku matematycznego (patrz: Dowód \ref{proof_diff_of_elements_of_M_with_C}). Zauważmy, że w przypadku naszej pracy musimy przyjąć, że funkcje $R,F,H$ są klasy $\mathcal{C}^{2}[K]$. Wyróżniamy to zdanie:

\begin{ass}\label{needed_diff}
Funkcje $R,F,H$ są klasy $\mathcal{C}^{2}[K]$
\end{ass}

	Zauważmy, że mamy pewne preferencje co do postaci analitycznej funkcji $R$. Dla przypadku deterministycznego dana jest ona wzorem~\footnote{Można bowiem wykorzystać cały aparat matematyczny stanowiący przedmiot owej pracy do modelowania innych zjawisk ekonomicznych. \citet{Coleman2} zauważa, że możliwie jest m.in. modelowanie wzrostu endogenicznego.}
	
\begin{equation*}
	R(x) \equiv \beta \Bigl \{[1 - \phi(f(x),x) - \phi_1 (f(x),x)f(x)]f_1 (x) + 1 - \delta \Bigl \},.
\end{equation*}	
	 
Ponieważ zaś $F(x) \equiv f(x) + (1 - \delta)x $, zatem widzimy, że {\it de facto} wymagamy od funkcji produkcji $f$ i funkcji stopy podatkowej $\phi$, aby były klasy $\mathcal{C}^{m+1}(K)$. Nie wydaje się to nazbyt wygórowanym warunkiem: teoretycy ekonomii bardzo często rozpatrują funkcje, które są nawet klasy $\mathcal{C}^{\infty}(U)$, a czasem wręcz analityczne $\mathcal{C}^{\omega}(U)$, czyli posiadające w każdym punkcie rozwinięcie w szereg Taylora.

	Lemat \ref{diff_of_elements_of_M_with_C} orzeka, przy jakich założeniach operator $B$ działa na zbiór funkcji różniczkowalnych. Ponieważ operator $C$ jest po prostu działaniem na funkcjach odwrotnych do tych, na które działa $B$, zatem widzimy, że w większości przypadków operator $C$ będzie zamieniał funkcje tak, jak chcemy: nie psując ich różniczkowalności. Problem pojawia się dopiero wtedy, gdy na pewną funkcję, której pochodna znika, podziałamy operatorem $\mathcal{J}_i$: wtedy pochodna funkcji odwrotnej będzie przyjmowała wartości nieskończone. Taka funkcja nie leży w przestrzeni Banacha funkcji $\mathcal{C}^m$: to nie jest przez nas pożądane, bowiem zależy nam na wykorzystaniu zbiorów otwartych w przestrzeni Banacha $\mathcal{C}^2$.

	Istotą rozpatrywania pojęcia różniczkowalności na otwartych podzbiorach przestrzeni Banacha jest to, że na takich zbiorach łatwo otrzymujemy jednoznaczność różniczki. Na dowolnych zbiorach wcale tak nie musi być, co obrazuje kontrprzykład podany przez Maurina (\citeyear[][S. 13, str. 200]{Maurin}):

 Niech $ V \subset \mathbb{R}^2 $, $ V = \bigl\{ (x,y) : x = y  \bigl\}$ oraz $ f(x,y) = x+y $. Wówczas $V$ nie jest otwartym podzbiorem $\mathbb{R}^2$ w sensie topologii euklidesowej. Rozpisując przyrost funkcji $f$ otrzymujemy

\begin{equation*}
f(x_0 + h_1, y_0 + h_2) - f(x_0, y_0) = h_1 + h_2 \,.
\end{equation*} 
 

	Ponieważ chcemy, aby $ (x_0, y_0),(x_0 + h_1, y_0 + h_2) \in V $, zatem $ h_1 = h_2 $. W takim razie dowolne przekształcenie liniowe, które w standardowej bazie przestrzeni sprzężonej do $ \mathbb{R}^2 $ możemy zapisać jako wektor $ (1-a, a) $, gdzie $ a \in \mathbb{R} $, spełnia definicję różniczki. Zachodzi bowiem

\begin{equation*}
	(1-a, a) \bullet (h_1, h_2) = (1-a) h_1 + a h_2 = 2 h_1 = h_1 + h_2 \,.
\end{equation*}

	Widzimy zatem, że mamy całe kontinuum możliwości wyboru postaci różniczki. Tym samym wygodniej jest zdefiniować różniczkowalność na zbiorach otwartych \footnote{Prosty dowód jednoznaczności, który można dostosować do dowolnej przestrzeni Banacha, patrz: \citet[][str. 16]{Spivak}.}, gdzie otrzymujemy jednoznaczność. 
	
	Co jednak zrobić, gdy topologia, w której interesujący nas zbiór mógłby być otwarty~\footnote{Można zawsze rozpatrzeć tak zwaną topologię dyskretną, w której zbiorem otwartym jest również singleton. Ogólnie, przestrzeń topologii jest kratą w sensie zawierania zbiorów z jedynką, która jest właśnie topologią dyskretną i zeram, którym jest topologia antydyskretna złożona tylko ze zbioru pustego i zbioru pełnego. Tym samym wybór topologii jest praktycznie nieograniczony i przy wyborze odpowiedniej wersji należy się kierować głównie względami matematycznej wygody.} jest, z różnych przyczyn, patologicznie nieporęczna? Wówczas naturalnym podejściem jest rozpatrzenie pewnej wygodnej dla nas topologii i podjęcie próby pokazania, że nasza funkcja (nasz operator) daje się rozszerzyć na otwarte otoczenie naszego zbioru: bierzemy zbiór otwarty w sensie naszej topologii i to na nim zajmujemy się różniczkowalnością. W tym momencie ujawnia się również przewaga przestrzeni funkcji $(\mathcal{C}^2(\Omega),||\circ||_{\mathcal{C}^2})$ nad $(\mathcal{C}^0(\Omega),||\circ||_{\infty})$: w tej pierwszej kule są na tyle małe, że można znaleźć zbiór funkcji, który składa się prawie ze wszystkich interesujących nas funkcji odwracalnych. Będziemy musieli po raz ostatni zrezygnować z rozpatrywania części funkcji, żeby uzyskać różniczkowalność operatora $C$: będziemy musieli założyć, że pochodna rozpatrywanych funkcji na rozpatrywanym przedziale nigdy nie jest równa zero. Zbiór ten zaproponował \citet{Lanza2}.
		  	 	
	Ponieważ dalej będziemy zajmować się tylko zbiorami postaci $[g_0^{(-1)}]_{\sim}$ lub $\mathcal{J}_{2}\Big[ [g_0^{(-1)}]_{\sim}\Big]$, a dla obu operatorów $\mathcal{J}_{i}$ będziemy chcieli przeprowadzić jeden dowód, zatem wygodnie będzie wprowadzić oznaczenia na dziedzinę i obraz funkcji z tych przestrzeni niezależnie od postaci $g_0$. Od tej chwili niech zatem $g: \overline{U}_1 \rightarrow	\overline{U}_2$ to pewna funkcja odrwacalna, a $U_i$ to zbiory otwarte. W naszym przypadku jeden z nich będzie zawsze dokładnie zbiorem $K^{\circ}$. Przy powyższych oznaczeniach dziedziną dla operatora odwracania funkcji będzie 


\begin{equation*}
	\mathcal{I}_{m} \equiv \Bigl\{ g \in \mathcal{C}^{m}(U_1;U_2): g - \mathrm{iniekcja},\, g[\overline{U}_1] = \overline{U}_2,\, g'(x) \not = 0 \underset{x\in \overline{U}_1}{\forall}   \Bigl\}\,,
\end{equation*}

	gdzie, w ogólności, $m \in \{1, 2, \dots\}$. Pokażmy teraz, za \citet{Lanza1}, skąd wziąć odpowiednie otwarte otoczenie tego zbioru. Rozważmy bowiem funkcję $ l_U : \mathcal{C}^{1}(\overline{U}) \rightarrow \mathbb{R}_{+} $ daną wzorem

\begin{equation}\label{l_function}
	l_U [g] \equiv \inf \Bigl\{ \frac{|g(x) - g(y)|}{|x-y|} : x, y \in \overline{U}, \, x \not= y \Bigl\}\,.
\end{equation}

Co ciekawe, zbiór $ U $ w ogólności może być zbiorem otwartym w $ \mathbb{R}^n $: w tej pracy będzie interesował nas oczywiście przypadek jednowymiarowy. Następujący lemat podamy jednak w pełnej ogólności, co mogłoby być interesujące w przypadku uogólnienia naszej gospodarki o więcej czynników produkcji. Przywołajmy następujący lemat \citep[][Str. 107, lemat 4.29]{Lanza1} o funkcji $ l_{U}[g] $: 
 
\begin{lemat}\label{continuous in c1}

Niech $ U \in \mathcal{T}(\mathbb{R}^n) $ to zbiór ograniczony i spójny, taki że 

$$ c[U] \equiv \sup \Bigl\{  \frac{\lambda(x,y)}{|x-y|} : x,y \in U, x \not= y \Bigl\} < \infty\,,$$
$$ \lambda(x,y) = \inf\Bigl\{ \int_{\gamma_{x,y}}1 : \gamma_{x,y} - \mathrm{krzywa\,klasy\,}\mathcal{C}^1, \, \gamma_{x,y}(0)=x \,, \gamma_{x,y}(1)=y \Bigl\}\,. $$ 

Wówczas $ g \mapsto l_U [g] $ jest ciągła z przestrzeni $ \mathcal{C}^1 (\overline{U}) $ wyposażoną w półnormę\footnote{Definicja półnormy jest taka jak normy, za wyjątkiem tego, że $ p(x) = 0$ nie implikuje, że $ x = 0 $.} $ p(g) = \underset{1 \leq k \leq n}{\sum} || \pardiff{g}{x_k} ||_{\infty}$ w liczby rzeczywiste.

\end{lemat}

Dowód odkładamy do dodatku matemetycznego (patrz: Dowód \ref{proof_continuous in c1}). 

Ponieważ przestrzeń $ (\mathcal{C}^{1}, || \circ||_{\mathcal{C}^1}) $ jest izomorficzna z $ \mathbb{R} \times \mathcal{C}^{0} $, zatem widzimy, że lemat \ref{continuous in c1} implikuje, że $ l $ jest ciągłym przekształceniem z $  (\mathcal{C}^{1}, || \circ||_{\mathcal{C}^1})  $ w $ \mathbb{R}_{+} $. Wynika to z tego, że norma w $\mathcal{C}^{1}$ jest równoważna normie $|f(0)|+p(f)$. 
 

Występująca w \ref{continuous in c1} funkcja $ c: \mathcal{T}(\mathbb{R}^n) \rightarrow \mathbb{R}_{+}$ mierzy wypukłość danego zbioru $ U \in \mathcal{T}(\mathbb{R}^n)$: jeśli bowiem $ U = \mathrm{conv}(U) $, to oczywiście dla $ x,y \in U $ zachodzi $  \lambda(x,y) = |x-y| $, bowiem najkrótszą możliwą drogą łączącą dwa punkty jest odcinek, a skoro $ U $ jest wypukły, więc ów odcinek jest w nim zawarty. Tym samym $ U = \mathrm{conv}(U) $ pociąga $ c[U] = 1 $. Występująca w powyższej definicji całka to zwykła całka krzywoliniowa, mierząca długość krzywej, wzdłóż której całkujemy. Jest to dobrze określona operacja, bowiem rozpatrujemy krzywe klasy $ \mathcal{C}^{1} $, czyli jednowymiarowe rozmaitości. Zauważmy, że w interesującym nas przypadku jednowymiarowym $c[U_i] =1$. Tym samym funkcja $l: \mathcal{C}^{1}(K) \rightarrow \mathbb{R}_{+}$ jest ciągła. Zatem przeciwobrazy zbiorów postaci $(\delta, \infty)$ dla $\delta \geq 0$ przy przekształceniu $l$ są otwarte:

\begin{equation*}
	Y_{\delta} \equiv l^{-1}\Big[ (\delta, \infty)\Big] \in \mathcal{T}(\mathcal{C}^{1}(K)).
\end{equation*} 

 

 
Z postaci analitycznej  $ l_U $ danej równością \ref{l_function} łatwo odczytujemy, że $ l[f] > 0 $ wtedy i tylko wtedy, gdy $ \det \mathbb{d}\,f(x) \not = 0 $ (czyli, na prostej $\mathbb{R}$ zachodzi $f'(x) \not = 0$) dla dowolnego $ x \in \overline{U} $. I to właśnie na zbiorze $ Y_{0} $ możemy jednoznacznie określić pojęcie różniczkowalności. Potrzebny nam będzie jednak jeszcze pewna techniczna uwaga orzekająca, że znajomość zbioru $Y_0$ pozwala nam również znaleźć odpowiednie zbiory otwarte w przestrzeniach funkcji $\mathcal{C}^{m}(K)$, dla $m \geq 1$. 

\begin{lemat}\label{openess_of_Y_0}
$Y_0\cap \mathcal{C}^{m}(K)$ jest otwarty w $\mathcal{T}\Big( \mathcal{C}^{1}(K)\Big)$.
\end{lemat}

Dowód lematu odkładamy do dodatku matematycznego (patrz: Dowód \ref{proof_openess_of_Y_0}). 

	Tym samym kandydat na otoczenie otwarte zbioru $\mathcal{I}_{m}$ to właśnie $Y_0 \cap \mathcal{C}^{m}(K)$. Wprowadźmy dodatkowo oznaczenie 
	
\begin{equation*}
	\mathcal{I} \equiv \Bigl \{ g \in Y_0 : g[\overline{U}_1] = \overline{U}_2 \Bigl \}\,.
\end{equation*}

Wówczas oczywiście $\mathcal{I}_{m} = \mathcal{I} \cap \mathcal{C}^{m}(K)$. 

Przed przystąpieniem do wysłowienia twierdzeń o różniczkowalności składowych $C$, poczyńmy jeszcze ogólną uwagę na temat zbiorów otwartych w przestrzeniach funkcyjnych. Niech $\mathbb{E}$ to przestrzeń Banacha a $\mathcal{M}$ to pewna topologiczna przestrzeń zwarta. Wówczas przez $\mathcal{C}^{0}(\mathcal{M}; \mathbb{E})$ rozumiemy oczywiście przestrzeń wszystkich funkcji ciągłych o nośniku zawartym w $\mathcal{M}$ i wartościach w $\mathbb{E}$. Jeśli $U$ jest podzbiorem otwartym w $\mathbb{E}$, to zbiór $\mathcal{C}^{0}(\mathcal{M}; U) = \{ f \in \mathcal{C}^{0}(\mathcal{M}; \mathbb{E}): f[M] \subset U \}$ to zbiór otwarty w $\mathcal{C}^{0}(\mathcal{M}; \mathbb{E})$ \citep[patrz:][str. 91]{Mardsen}. Korzystając z izomorficzności $\mathcal{C}^{m}(\mathcal{M}; \mathbb{E})$ z iloczynem $\mathbb{E}^{m} \times \mathcal{C}^{0}(\mathcal{M}; \mathbb{E})$ potrafimy również określić postać zbiorów otwartych w takich przestrzeniach: na pierwszych współrzędnych owego iloczynu trzeba rozpatrzeć zbiory otwarte w $\mathbb{E}$. W języku funkcji różniczkowalnych oznacza to, że na zbiór otwarty składają się funkcje, których pierwsze $m$ pochodnych ewaluowanych np. w zerze przyjmuje wartości z pewnych otwartych zbiorów w topologii określonej na $\mathbb{E}$, a $m$-ta pochodna jest funkcją ze zbioru $\mathcal{C}^{0}(\mathcal{M}; U)$, dla pewnego $U$ otwartego w $\mathbb{E}$. Powyższe uwagi są potrzebne dla zrozumienia, że rozpatrywane dalej zbiory funkcji są rzeczywiście otwarte i faktycznie można zdefiniować na nich operację różniczkowania w sposób jednoznaczny. Dalej ograniczamy się do przypadku $\mathbb{E}=\mathbb{R}$.

\section{Twierdzenia o różniczkowalności składowych operatora funkcji odwrotnych}

Ponieważ określiliśmy już dziedzinę, na której działa operator $C$, zatem możemy teraz przejść do kwestii różniczkowalności poszczególnych jego składowych. Ponieważ $C = \overline{C} \circ \mathcal{J}_2 \circ \Upsilon \circ \mathcal{J}_1$, zatem rozpoczniemy nasze rozważania od operatorów odwracania.

Niech zatem $g_0 \in \mathcal{C}^{m+q}(\overline{U}_1,\overline{U}_2)\cap \mathcal{I}$ to punkt, wokół którego chcemy różniczkować operator odwracania. Wówczas zachodzi następujący

 
\begin{lemat}\label{macho_geek_ze_mnie}
 	Niech $m, q \in \mathbb{N}$ oraz $q > 0$. Niech $\mathcal{J}: \mathcal{C}^{m+q}(\overline{U}_1, \overline{U}_2) \cap \mathcal{I} \rightarrow \mathcal{C}^{q}(\overline{U}_2,\overline{U}_1)$ będzie dany przez $\mathcal{J}[g] = g^{(-1)}$. Niech $g_0 \in \mathcal{C}^{m+q}(\overline{U}_1, \overline{U}_2)\cap \mathcal{I}$. Wówczas 
\begin{enumerate}

\item{Istnieje otoczenie $\mathcal{W}_{g_0}$ punktu $g_0$ w przestrzeni $\mathcal{C}^{m+q}(\overline{U}_1, \overline{U}_2)$ oraz jednoznacznie wyznaczony operator $\hat{\mathcal{J}}: \mathcal{W}_{g_0} \rightarrow \mathcal{C}^{q}(\overline{U}_2, \mathbb{R})$ klasy $\mathcal{C}^{q}$, który jest rozszerzeniem $J$ na rozpatrywanym zbiorze $\mathcal{W}_{g_0}$, czyli zachodzi
 	
\begin{equation*}
	\hat{\mathcal{J}}[g] = \mathcal{J}[g] \underset{g \in \mathcal{W}_{g_0} \cap \mathcal{I}}{\forall}\,.	
\end{equation*} 	
	}
\item{Różniczka operatora $\hat{\mathcal{J}}$ w dowolnym punkcie $g \in \mathcal{W}_{g_0} \cap \mathcal{I}$ dana jest formułą

\begin{equation*}
	\mathrm{d}\, \hat{\mathcal{J}}[g](h) \equiv - \Bigl[ \mathrm{d}\, g \circ g^{-1} \Bigl]^{(-1)} \cdot \, h \circ g^{(-1)} \,,
\end{equation*} 	

dla dowolnego $h$ ze zbioru $\mathcal{C}^{m+q}(\overline{U}_1, \mathbb{R})$.
	}
\end{enumerate} 	
\end{lemat}

Dowód lematu odkładamy do dodatku matematycznego (patrz: Dowód \ref{proof_macho_geek_ze_mnie}). 

Przyjmując $m=q=1$ oraz rozpatrując $g_0^{(-1)} \in \mathcal{H}$ z dziedziną $\overline{U}_1 = g_0 [K]$ oraz przeciwdziedziną $\overline{U}_2 = K$ widzimy, że operator $\mathcal{J}_1$ jest różniczkowalny w $g_0^{(-1)}$. Przyjmując $m=0$ oraz $q=1$ oraz rozpatrując $g_0 \in \mathcal{J}_1 \Bigl[\mathcal{H}\Bigl]$ z dziedziną $\overline{U}_1 = K$ oraz przeciwdziedziną $\overline{U}_2 = g_0 [K]$ widzimy, że operator $\mathcal{J}_2$ jest różniczkowalny w $g_0^{(-1)}$.   

Następnie, dość łatwo pokazać, że zachodzi następujący lemat o różniczkowalności operatora $\Upsilon$:

\begin{lemat}\label{nonautonomous_nemytskii}
	Nieautonomiczny operator Niemyckiego $\Upsilon: \mathcal{C}^{1}(K) \rightarrow \mathcal{C}^{1}(K)$, $\Upsilon[m](x)=\frac{m(x)}{R(x)}$ jest różniczkowalny w normie $|| \circ||_{\mathcal{C}^{1}}$. Różniczka operatora ewaluowana w punkcie $k \in \mathcal{C}^{1}(K)$ dana jest wzorem
	
\begin{equation*}
	\mathrm{d}\, \Upsilon[k] \equiv \Upsilon \,.
\end{equation*}	
\end{lemat}

Dowód odkładamy do dodatku matematycznego (patrz: Dowód \ref{proof_nonautonomous_nemytskii2}).

Dowiedzenie ostatniego twierdzenia wymaga poczynienia dodatkowego, technicznego założenia na temat funkcji produkcji: dla wygody rozumowania, chcemy ją dookreślić na pewnym małym otoczeniu zera z lewej strony. Również w tym miejscu będziemy wymagali większego stopnia różniczkowalności funkcji $F$, co założyliśmy już uprzednio, aby operator $C$ mógł działać na przestrzeni funkcji $\mathcal{C}^2$, zobacz: lemat \ref{needed_diff}. Niech zatem:

\begin{ass}\label{on_prod_last_time}
	Niech funkcja $F^{(-1)}$ jest dwukrotnie różniczkowalna na $\mathbb{R}_{+} \cup (-\epsilon, 0].$
\end{ass}

Przy powyższych założeniach zachodzi

\begin{lemat}\label{diff_of_C_barred_in_C_0} Operator $\overline{C}: \mathcal{C}(K) \rightarrow \mathcal{C}(K)$ jest klasy $C^1$, jeśli funkcja $F^{(-1)}$ jest dwukrotnie różniczkowalna na $\mathbb{R}_{+} \cup (-\epsilon, 0].$ Pochodna operatora $ \overline{C} $ w punkcie $ g_0 \in \mathcal{C}^{1}(K)$ ewaluowana w $ h \in \mathcal{C}^{1}(K) $ dana jest wzorem:
	
\begin{equation}
\Bigl[\mathrm{d}\, \overline{C}[g_0](h)\Bigl] (y) \equiv \frac{h(y)}{F'\Bigl( F^{(-1)}\bigl( H(y) + g_0(y)\bigl) \Bigl)} \,.
\end{equation}	
\end{lemat}

Dowód lematu odkładamy do dodatku matematycznego (patrz: Dowód \ref{proof_diff_of_C_barred_in_C_0}).

Znając już cząstkowe wyniki zauważamy, że różniczkowalność operatora $C$ wynika teraz z ogólnego twierdzenia o pochodnej funkcji superponowanej \citet[][twr. VII.4.2, str. 152]{Maurin}: wystarczy bowiem zauważyć, że zamiast działa ono dla proponowanych przez lemat \ref{macho_geek_ze_mnie} rozszerzeń rozpatrywanych operatorów odwrócenia funkcji, zatem działa również, gdy ograniczymy uwagę do odpowiednich klas zbiorów funkcji. Dokładniej: jeśli chcemy zróżniczkować $C$ w funkcji $g_{0}^{(-1)}$, to rozpatrujemy pierwotnie zbiór $[g_{0}^{(-1)}]_{\sim}$, otaczamy go otwartym zbiorem, odpowiednio rozszerzamy na niego operator odwracania funkcji i otrzymujemy $\hat{\mathcal{J}}_1$. Następnie superponujemy to rozszerzenie z operatorem $\Upsilon$. I dalej robimy to samo dla pozostałych rozszerzeń i różniczkujemy. Pod koniec obcinamy całość na powrót do zbioru $[g_{0}^{(-1)}]_{\sim}$.

Korzystając wielokrotnie z zasady różniczkowania funkcji superponowanych widzimy, że dla $a^{(-1)} \in \mathcal{H}$ zachodzi

\begin{equation*}
	\mathrm{d}\, C[a^{(-1)}](h) = \mathrm{d}\, \overline{C}[\mathcal{J}_2 \circ \Upsilon \circ \mathcal{J}_1(a^{(-1)})] \circ \mathrm{d}\, \mathcal{J}_2 [\Upsilon \circ \mathcal{J}_1(a^{(-1)})] \circ \mathrm{d}\, \Upsilon  [\mathcal{J}_1(a^{(-1)})] \circ \mathrm{d}\, \mathcal{J}_1(a^{(-1)}) h\,,
\end{equation*}

co upraszcza się nieco, jeśli zaczniemy ewaluować wartości operatorów. Wówczas zachodzi
\begin{equation*}
	\mathrm{d}\, C[a^{(-1)}](h) = \mathrm{d}\, \overline{C}[\Bigl[ \frac{a}{R}\Bigl]^{(-1)}] \circ \mathrm{d}\, \mathcal{J}_2 [\frac{a}{R}] \circ \mathrm{d}\, \Upsilon  [a] \circ \mathrm{d}\, \mathcal{J}_1(a^{(-1)}) h\,.
\end{equation*}

Odcyfrowywanie tej zależności rozpoczniemy od prawej strony:

\begin{align*}
\mathrm{d}\, \mathcal{J}_1(a^{(-1)}) h = -\Bigl[\mathrm{d}\,a^{(-1)} \circ a\Bigl]^{(-1)} \cdot h \circ a\,, \\
\mathrm{d}\, \Upsilon  [a] \mathrm{d}\, \mathcal{J}_1(a^{(-1)}) h = \Upsilon \circ \mathrm{d}\, \mathcal{J}_1(a^{(-1)}) h = -\frac{\Bigl[\mathrm{d}\,a^{(-1)} \circ a\Bigl]^{(-1)} \cdot h \circ a}{R} \,. \\
\end{align*}

Dalej zaś

\begin{align*}
	\mathrm{d}\, \mathcal{J}_2 [\frac{a}{R}]  \mathrm{d}\, \Upsilon  [a]  \mathrm{d}\, \mathcal{J}_1(a^{(-1)}) h = \Bigl[\mathrm{d}\,\big[\frac{a}{R}\big] \circ \big[ \frac{a}{R}\big]^{(-1)}\Bigl]^{(-1)} \cdot \frac{\Bigl[\mathrm{d}\,a^{(-1)} \circ a\Bigl]^{(-1)} \cdot h \circ a}{R} \circ \big[ \frac{a}{R}\big]^{(-1)}\,.
\end{align*}

Ostatecznie 

\begin{align*}
	\mathrm{d}\, C[a^{(-1)}](h) = \frac{\mathrm{d}\, \mathcal{J}_2 [\frac{a}{R}]  \mathrm{d}\, \Upsilon  [a]  \mathrm{d}\, \mathcal{J}_1(a^{(-1)}) h}{F'\Bigl( F^{(-1)}\bigl( H + \big[ \frac{a}{R}\big]^{(-1)} \bigl) \Bigl)} = \frac{\Bigl[\mathrm{d}\,\big[\frac{a}{R}\big] \circ \big[ \frac{a}{R}\big]^{(-1)}\Bigl]^{(-1)} \cdot \frac{\Bigl[\mathrm{d}\,a^{(-1)} \circ a\Bigl]^{(-1)} \cdot h \circ a}{R} \circ \big[ \frac{a}{R}\big]^{(-1)}}{F'\Bigl( F^{(-1)}\bigl( H + \big[ \frac{a}{R}\big]^{(-1)} \bigl) \Bigl)}\,.
\end{align*}

Tak uzyskany wynik podsumowujemy w wieńczącym ten rozdział twierdzeniu:

\begin{tw}\label{on_C_diff_final}
	Operator $C$ jest różniczkowalny w punkcie $g_0^{(-1)} \in \mathcal{H}$ sensie różniczki na każdym zbiorze $[g_{0}^{(-1)}]_{\sim}$. Pochodna operatora $ C $ w punkcie $ g_0^{(-1)}$ ewaluowana w $ h \in \mathcal{C}^{1}(g[K]) $ dana jest wzorem:
	
\begin{equation*}
\mathrm{d}\, C[g_{0}^{(-1)}](h) = \frac{\Bigl[\mathrm{d}\,\big[\frac{g_{0}}{R}\big] \circ \big[ \frac{g_{0}}{R}\big]^{(-1)}\Bigl]^{(-1)} \cdot \left( \frac{\Bigl[\mathrm{d}\,g_{0}^{(-1)} \circ g_{0}\Bigl]^{(-1)} \cdot h \circ g_{0}}{\mathlarger{R}} \circ \big[ \frac{g_{0}}{R}\big]^{(-1)}\right)}{F'\Bigl( F^{(-1)}\bigl( H + \big[ \frac{g_{0}}{R}\big]^{(-1)} \bigl) \Bigl)}\,.
\end{equation*}
\end{tw}

Zauważmy, że powyższa zależność szczęśliwie jest liniowa z $h$, co potwierdza, że mamy do czynienia z liniowym przybliżeniem operatora $C$.



%%%%%%%%%%%%%%%%%%%%%%%%%%%%%%%%%%%%%%%%%%%%%%%%%%%%%%%%%%%%%%%%%%%%%%%%%%%%%%%%%%%%%%%%%%%%%%%%%%%% 

\chapter{Podsumowanie oraz dalsze kierunki badań}\label{chap_conclusions}

	W rozdziale tym zostaną przedstawione wnioski z pracy oraz zakreślone zostną możliwe dalsze kierunki badań. 


\section*{Wnioski z pracy}

	
	Podsumujmy teraz wyniki niniejszego badania. 
	
	W rozdziale pierwszym udało się dokonać syntezy wyników z prac Colemana (\citeyear{Coleman1,Coleman2}). Wedle naszej wiedzy, nikt nie podjął się wcześniej formalnego połączenia wniosków z obu prac tak, aby stało się zupełnie jasne to, kiedy oba modele są w pełni równoważne: m.in. w literaturze nie pojawiła się sformalizowanie pojęcia algebraicznego sprzężenia obu operatorów (które tu nazwaliśmy quasi-sprzężeniem). Nie pojawia się również nigdzie warunek na asymptotyczne zachowanie się w zerze funkcji użyteczności, który zdaje się być wymagany do wykazania topologicznej równoważności problemów punktu stałego operatora $A$ na zbiorze $\mathcal{C}_F$ oraz operatora $B$ na zbiorze $\mathcal{M}$. W pracy sprecyzowano również poprawny zestaw warunków wymaganych do dowiedzenia $x_0$-monotoniczności operatora $B$ zdefiniowanego na przestrzeni funkcyjnej rozpatrywanej w pracy Colemana z \citeyear{Coleman2}-ego roku. W końcu, poczyniona uwaga, że $B[\hat{\mathcal{Z}}] \subset \mathcal{M}$, ostatecznie i w sposób formalny łączy oba zaproponowane przez Colemana podejścia w nierozerwalną całość. Co więcej, zupełnie niezauważony pozostawał fakt, że równowaga musi być funkcją ściśle rosnącą. Tym samym, w pracy rzucono odrobinę więcej światła na własności analityczne równowagowej funkcji najlepszej odpowiedzi. 
	
	W rozdziale drugim dokonaliśmy szeregu obserwacji dotyczących sposobu rozumienia problemu definiowania operatora $B$. Zaproponowaliśmy tam również nieistniejącą dotąd w literaturze konstrukcję operatora funkcji odwrotnych, który, w przeciwieństwie do badanych do tej pory operatorów, można zapisać postaci analitycznej danej {\it explicit\' e}, kosztem zastosowanie dwukrotrnie operacji odwracania funkcji. Zauważyliśmy, że takie spojrzenie na problem może nasuwać skojarzenie z operatorem Niemyckiego. Uwaga ta w naturalny sposób zaprowadziła nas do postawienia kwestii różniczkowalności tak zdefiniowanego operatora. Zauważyliśmy, że występująca dwukrotnie operacja odwracania funkcji wymaga ograniczenia uwagi do funkcji dwukrotnie różniczkowalnych. Nadto: zauważyliśmy, że próba zdefiniowania różniczki w sposób wykorzystujący klasyczne twierdzenia matematyczne może się powieść wyłącznie na pewnych ściśle określonych podzbiorach przestrzeni funkcji. Dla tak zdefiniowanych zbiorów funkcji zdefiniowaliśmy pojęcie różniczkowalności i wyprowadziliśmy analityczne wzory na różniczkę operatora.   

	Różniczkowalność operatora $C$ można wykorzystać w sposób następujący: przy iterowaniu standardowego algorytmu obliczania równowagi pochodzącego od Colemana (\citeyear{Coleman1}). Załóżmy bowiem, że funkcja $H^{(-1)}\circ F$ jest dwukrotnie różniczkowalna. Wówczas lemat \ref{diff_of_elements_of_M_with_C} implikuje, że również funkcje pozyskane poprzez iteracyjne przykładanie do tej funkcji operatora $B$, czyli $h_n \equiv B^{n}[H^{(-1)}\circ F]$, będą tej samej klasy różniczkowalności. Co więcej, łatwo pokazać, że będą to również funkcje spełniające założenia twierdzenia \ref{on_C_diff_final}. Licząc pochodną operatora $\mathcal{C} - \mathrm{Id}$ możemy uzyskać informację o {\it nachyleniu} operatora $C$ w punkcie $h_n$, którą można wykorzystać do orientacyjnego badania siły zbieżności algorytmu w sensie normy supremum.  
	
	Zauważmy również, że mylny może wydawać się pogląd, że wyprowadzona formuła analityczna na różniczkę operatora $C$ jest zawiła. Biorąc pod uwagę komputerowy przybliżony sposób przechowywania funkcji pod postacią długiego wektora wartości w pewnych węzłach bardzo łatwo wyobrazić sobie sposób dochodzenia postaci funkcji powstającej przez wymnożenie wartości dwóch funkcji. Korzystając z geometrycznej interepretacji odwracania funkcji łatwo jest też znaleźć zapis funkcji odwrotnej. Do obliczania odpowiednich pochodnych można zaś stosować standardowe metody różniczkowania numerycznego \citep[zobacz:][str.447]{Kincaid}.
		
\section*{Możliwe dalsze kierunki badań}

Jak już wspomnieliśmy w rozdziale \ref{sec-lattice}, zbiór funkcji $\hat{\mathcal{Z}}$ jest do tej pory największą klasą funkcji, na jakiej jesteśmy w stanie dowieść istnienia i jednoznaczności równowagi. W literaturze przedmiotu podejmowane są jednak próby rozpatrzenia tych wyników w ogólniejszych przestrzeniach funkcyjnych \citep[patrz:][]{Reffett}. Przedmiotem badań stają się funkcje rosnące i półciągłe z góry \citep[patrz:][\S 6, IV. 28]{Bourbaki_1971}, bowiem takie funkcje posiadają własność Darboux \citep[patrz:][]{Guillerme_1995} i można jednoznacznie określić operator $B$ w sposób analogiczny do tego przedstawionego w rozdziale \ref{sec-lattice}. Ów kierunek badań pozostaje jednak na razie niekompatybilny ze stosowanym do tej pory aparatem pojęciowym odnoszącym się do teorii optymalnego sterowania prezentowanym przynajmniej w formie zaproponowanej przez Stokey, Lucasa i Prescotta (\citeyear[][]{Prescott}). Problem w tym, że z twierdzeń zamieszczonych w owym opracowaniu wynika, że poszukiwana funkcja polityki jest ciągła, a nie półciągła z góry. Tym samym każda próba udowodnienia, że funkcja półciągła z góry faktycznie jest punktem stałym odpowiednio zdefiniowanego operatora powinna być uzupełniona o lemat pokazujący, że jest tak zdefiniowany punkt stały to również funkcja rozwiązująca podstawowe zagadnienie optymalizacyjne reprezentatywnego konsumenta. Opracowanie takiego lematu może wymagać rozbudowania całkiem nowej teorii matematycznej.
 

Zauważmy również, że nic nie wiemy o optymalności ciągłych funkcji leżących w paśmie $\mathcal{P} \equiv \{ (x,y): x \in K, y \in [0,H^{(-1)}(F(x))]\}$ i nie wchodzących w skład zbioru $\hat{\mathcal{Z}}$. Nie jest wykluczone, że znajduje się wśród nich wiele równowag. W stosunku do nich wydaje się jednak, że metody operte na definiowaniu abstrakcyjnych operatorów zdefiniowanych {\it implicit\' e} jako przeciwobrazy zera pewnej funkcji biorącej jako argument trójkę $(x,y,m)$ wyczerpały się. W ramach moich własnych badań stwierdziłem w szczególności, że nie ma sensu definiować operatorów w sposób następujący:

\begin{itemize}
	\item{$B_1[m](x) \equiv H^{(-1)}\Bigl[ F(x) - \bigl[ \frac{R}{m} \big]^{(-1)} \bigl( \frac{1}{m(x)}\bigl)  \Bigl]$,}
	\item{$B_2[m](x) \equiv \Bigl[\frac{m}{R}\Bigl] \Bigl( F(x) - H(m(x)) \Bigl)$,}
\end{itemize}  

jeśli za ich dziedzinę uznać zbiór $\mathcal{M}$. Wówczas bowiem nie przeprowadzają one tego zbioru w siebie, co czyni je z punktu widzenia metod punktu stałego pozbawionym zupełnie bezużytecznymi. Być może bardziej owocne byłoby rozważanie dla nich innej dziedziny  -- tego rodzaju badania nie zostały jeszcze przeprowadzone.	 

Dla porządku dodajmy, że również zbiór $\hat{\mathcal{Z}}$ nie jest największym możliwym zbiorem, na którym w ogóle można stosować metody opierające się o zdefiniowanie operatorów wyznaczonych przez punkt przecięcia pewnych dwóch funkcji. Jeżeli zdefiniujemy, tak jak w rozdziale \ref{chap_right_space}, funkcję będącą różnicą funkcji, z których jedna jest indeksowana jednym parametrem więcej niż druga oraz obie zależą w jakiś sposób od całej funkcji $m$ oraz są w odpowiedni sposób monotoniczne z każdym argumentem z osobna

\begin{equation*}
\theta(x,y,m) = \Phi_m(x,y) - \Delta_m(y),
\end{equation*}

a następnie zaczniemy definiować operator jako obiekt zależny od $m$, leżacy w przeciwobrazie $\theta_m^{-1}[\{ 0\}]$, to wiemy, że chcemy aby ów obiekt spełniał tylko dwie własności:

\begin{enumerate}
	\item{powinna to być funkcja zmiennej $x$, czyli $y = y(x,m)$},
	\item{owa funkcja powinna być ciągła}.
\end{enumerate}

Jak wskazują rozważania z rozdziału \ref{chap_right_space}: nie można porzucić tego warunku, aby nie zerwać więzi z podstawowym problemem naszego przedsięwzięcia, którym jest znalezienie ciągłej funkcji konsumpcji w równowadze. Najbardziej naturalnym podejściem jest zażądać, aby $\theta_m^{-1}[\{ 0\}]$ był ciągłą funkcją. Istnieje jednak ogólniejsze spojrzenie na problem: moglibyśmy bowiem zażądać, żeby $\theta_m^{-1}[\{ 0\}]$ było korespondencją \citep[patrz:][]{Rockafellar_2010}, której pewny selektor~\footnote{Ang.: {\it selection}.} jest funkcją ciągłą. Następnie, spośród zbioru ciągłych selektorów można by wybrać pewne wyróżnione, np. największe, co było naturalnym postępowaniem, aby uniknąć wybierania funkcji stale równej zero. Próby wykorzystanie tego typu metod można spotkać w pracy Reffetta (\citeyear{Reffett}): za ich pomocą próbuje on udowodnić istnienie równowagi w przypadku podatków regresywnych w klasie funkcji półciągłych z góry i rosnących. Zachowanie pewnego stopnia ogólności przy poszukiwaniach równowagi w tej klasie problemów spowodowane jest wynikami badań Santosa (\citeyear{Santos_2002}), który stwierdza, że z pewnością nie może istnieć ciągła równowaga w modelu z podatkami regresywnymi. By to udowodnić, powołuje się on na aparat matematyczny z zakresu teorii układów dynamicznych. 

	Stan badań nad modelem nieoptymalnej gospodarki z podatkami jest zatem taki: z jednej strony potrafimy pokazać jednoznacznie wyznaczoną równowagę w przypadku podatków progresywnych (z możliwością istnienia wielu innych równowag); z drugiej zaś strony zachodzi podejrzenie, że takowa równowaga nie istnieje w przypadku podatków regresywnych. W tej sytuacji narzuca się zupełnie naturalne pytanie:  w jaki sposób rodzaj opodatkowania może tak znacząco wpływać na istnienie bądź zatracenie się optymalnej, ciągłej reguły decyzyjnej dla reprezentatywnego gospodarstwa? Pierwotnym i nadrzędnym celem niniejszego badania była odpowiedź na to pytanie poprzez użycie aparatu matematycznego, jakim są abstrakcyjne metody homotopijne. 
	
	Metody homotopijne to bardzo ogólne pojęcie, u którego podstaw leży wykorzystanie homotopii --- ciągłego przekształcenia jednego zbioru w inny \citep[ogólnie o homotopiach -- patrz:][str. 315]{Dugundji}. Teoria homotopii to wspólna nazwa dla wielu, stosunkowo luźno powiązanych metod, które  często odwołują się do pojęcia topologicznego indeksu: funkcji zliczającej liczbę rozwiązań pewnego problemu \citep[patrz:][str. 239]{Granas_FPT}. Ponieważ rozpatrywany przez nas problem jest jednak nieskończenie wymiarowy, chcieliśmy użyć bardziej ogólnego podejścia pochodzącego od Granasa (\citeyear{Granas_1980, Granas_FPT}) i odwołującego się pojęcia topologicznej transwersalności. Ów aparat matematyczny korzysta z metod topologii ogólnej. Jego uogólniona wersja cieszy się uznaniem zwłaszcza przy problemach postawionych jako poszukiwanie rozwiązań układów równań cząstkowych \citep[patrz:][]{Precup_1995}. Operatory, którymi trudni się owa teoria muszą być {\it zupełnie ciągłe}~\footnote{Ang. {\it completely continuous}. Powyższe tłumaczenie nie jest ogólnie przyjętym polskim odpowiednikiem. Zdaje się, że podobne sformuowanie funkcjonuje w teorii liniowych przekształceń pomiędzy abstrakcyjnymi przestrzeniami Banacha.}, czyli zamieniać każdy ograniczony zbiór na zbiór prezwarty. W przypadku rozpatrywanych w naszej pracy operatorów taka sytuacja ma oczywiście miejsce: wszak cały zbiór $\hat{\mathcal{Z}}$, traktowany jako podprzestrzeń topologiczna przestrzeni Banacha funkcji ciągłych wyposażoną w normę supremum, zostaje przez $B$ zamieniony w zbiór funkcji prezwarty w normie supremum $\mathcal{M}$. Nie jest trudno udowodnić, że operator $B: \hat{\mathcal{Z}} \rightarrow \hat{\mathcal{Z}}$ będzie {\it dopuszczalnie homotopijny}~\footnote{Ang.: {\it admissibly homotopic}: jedna z kluczowych własności wymaganych od operatora w celu użycia metod homotopijnych. Z grubsza oznacza ona tyle, że jeżeli rozważymy brzeg pewnego zbioru, na który ów operator nie przyjmuje punktów stałych, to każde ciągłe rozszerzenie operatora, które zgadza się z nim na brzegu, posiada punkt stały.} na odpowiednio zdefiniowanym zbiorze. Tak przygotowany aparat matematyczny można by wykorzystać do badania istnienia punktów stałych operatorów, które różniły by się stopniem regresywności-progresywności podatków, wyrażających się w nachyleniu i krzywiźnie funkcji $R$.  

W przestrzeniach skończenie wymiarowych istnieje teoria homotopii odwołująca się do pojęć geometrii różniczkowej. Jej metody dostarczają nie tylko teoretycznych, ale również i praktycznych, zalgorytmizowanych rozwiązań problemów z zakresu poszukiwań rozwiązań nieliniowych układów równań. Metody owe, znane adeptom analizy numerycznej po prostu jako {\it metody homotopijne} lub jako {\it metody kontynuacji}, stają się coraz bardziej powszechnym i uznanym narzędziem, wykorzystywanym zwłaszcza przy silnie nieliniowych problemach. Bogate źródło wiedzy na ich temat zawierają m.in. monografie Allgowera (\citeyear{Allgower_1990}) oraz Sommese (\citeyear{Sommese_2005}). Metody te nie są obce ekonomistom: ich wykorzystanie przy obliczeniach związanych z poszukiwaniem równowagi Nash'a sugerują m.in. \citet{Doraszelski_2008} oraz \citet{Judd_NME}. W różniczkowalności operatorów zdefiniowanych w naszej pracy upatrywaliśmy możliwości zbudowania analogonu powyższych metod w przestrzeniach nieskończenie wymiarowych. Takie podejście nie jest w żadnen sposób obce teorii matematyki: z dobrodziejstw zwartości odpowiednich operatorów korzysta m.in. bardzo popularna ostatnimi czasy teoria równań Naviera-Stokes'a, opisująca fizyczny model ruchu turbulentnego. W owej teorii wprowadza się pojęcie zwartego atraktora, wokół którego można przybliżać rozwiązania ogólnego problemu nieskończenie-wymiarowego za pomocą jego skończenie-wymiarowych przybliżeń \citep[patrz:][]{Foias_2004}. To właśnie możliwość dobrego przybliżania zbiorów zwartych w przestrzeniach nieskończenie wymiarowych za pomocą przestrzeni skończenie wymiarowych chcieliśmy wykorzystać w tej pracy~\footnote{Dokładniej: zależało nam na zbudowaniu odpowiednika metody homotopijnego korektora-predyktora, patrz: \citep[][rozdziały 3-9]{Allgower_1990}.}. 

	Nieprzewidywalna przeszkoda pojawiła się dopiero po udanym przekształceniu operatora $B$ i stworzeniu jego odpowiednika wśród funkcji odwracalnych --- operatora $C$, którego {\it explicit\' e} zapisana postać analityczna okazała się tym, czego potrzebowaliśmy do zastosowania teorii różniczkowalności operatorów Niemyckiego. Niestety, ograniczenie w postaci ciągłości tak zdefiniowanej różniczki drastycznie ogranicza jej stosowalność: już po pierwszej iteracji nie możemy być pewni, że przybliżenie uzyskane taką drogą będzie bliskie punktowi stałemu w sensie normy $||\circ||_{\mathcal{C}^2}$. Wykorzystanie większości algorytmów wymaga jednak iterowania pewnej procedury przybliżania. W tym wypadku niemożliwe jest zatem bezpośrednie wykorzystanie metody Newtona do znalezienia rozwiązania \citep[ogólny algorytm Newtona -- zobacz:][rozdz. 12.8.3, str. 690]{Kompedium}. Wydaje się zatem, że próba kontynuowania budowania analogonu dla różniczkowych metod homotopijnych w przestrzeniach nieskończenie wymiarowych nie może się powieść, dopóki rozpatrywane operatory faktoryzują się w sposób podobny do operatora $C$, dla którego niezbędne jest wykorzystanie operatora odwracania $\mathcal{J}$~\footnote{Jednakże: zaproponowana powyżej teoria ma duże szanse powodzenia w przypadku operatorów różniczkowalnych, których różniczki nie powodują tego, że pomiar bliskości pochodnych wszystkich stopni staje się niemożliwy.}.
	 
	Otwarta pozostaje jednakże ścieżka badań związana z metodami homotopijnymi nie odwołującymi się do pojęcia różniczkowalności i to na nich powinna skupić się uwaga teoretyków ekonomii zainteresowanych dalszym rozwijaniem teorii nieoptymalnych gospodarek z podatkami.  
 
 
%%%%%%%%%%%%%%%%%%%%%%%%%%%%%%%%%%%%%%%%%%%%%%%%%%%%%%%%%%%%%%%%%%%%%%%%%%%%%%%%%%%%%%%%%%%%%%%%%%%%%%%%%%%%%%% 
%%%%%%%%%%%%%%%%%%%%%%%%%%%%%%%%%%%%%%%%%%%%%%%%%%%%%%%%%%%%%%%%%%%%%%%%%%%%%%%%%%%%%%%%%%%%%%%%%%%%%%%%%%%%%%% 







\chapter{Dodatek matematyczny}\label{chap_appendice}
\section{Podstawowe pojęcia}\label{preliminaries}

W tej części pracy zostaną wprowadzone najważniejsze pojęcia matematyczne wykorzystane w pracy. 

Zbiór liczb rzeczywistych oznaczamy przez $ \mathbb{R} $. Przez $ \mathbb{R}_{+} $ rozumiemy liczby rzeczywiste nieujemne, $\{r :  r \geq 0 \}$. Przez $ \mathbb{R}_{++} $ rozumiemy liczby rzeczywiste ściśle dodatnie $ \{r : r > 0 \} $. Przez $ \overline{\mathbb{R}}_{+} $ rozumiemy $ \mathbb{R}_{+} \cup \{ \infty \} $. Podobną konwencję stosujemy dla domkniętych przedziałów zawartych w $\mathbb{R}_{+}$. 

Dla funkcji odwracalnej $f: A \rightarrow B$ jej odwrotność oznaczamy przez $f^{(-1)}: f[A] \rightarrow A$. Przez $f^{-1}[C]$ oznaczamy natomiast przeciwobraz zbioru $C \subset B$.

Przyjmujemy za znane podstawowe pojęcia algebry liniowej i topologii metrycznej. 

Zbiorem wypukłym w przestrzeni liniowej nazywamy taki zbiór $ W $, że jeśli $ \{x,y\} \subset W$ to dla $ t \in [0,1] $ zachodzi $ tx + (1-t)y \in W $. Piszemy wówczas $ W = \mathrm{conv}(W) $. Geometrycznie oznacza to, że jeśli dany zbiór zawiera wierzchołki odcinka, to zawiera i cały odcinek.

\subsubsection*{Topologie}

Niech $ \mathcal{X} $ to dowolny zbiór. Zbiór potęgowy $ \mathcal{P}(\mathcal{X}) $ to klasa wszystkich podzbiorów $ \mathcal{X} $. Topologią nazywamy klasę zbiorów $ \mathcal{T} \subset \mathcal{P}(\mathcal{X})$, taką że:

\begin{enumerate}
	\item{$ \emptyset, \mathcal{X} \in \mathcal{T}$,}
	\item{$ \mathcal{H} \subset \mathcal{T} \Longrightarrow \bigcup \mathcal{H} \in \mathcal{T}$,}
	\item{$ \mathcal{H} \subset \mathcal{T} \wedge |H|<\aleph_0 \Longrightarrow \bigcap \mathcal{H} \in \mathcal{T} $.}
\end{enumerate}

Zatem jest to klasa zbiorów zamknięta na branie uogólnionej sumy dowolnej rodziny elementów i branie skończonego przecięcia. Elementy topologii nazywa się zbiorami otwartymi. Parę $ (\mathcal{X}, \mathcal{T})$ nazywamy przestrzenią topologiczną. Zbiór domknięty definiujemy jako dopełnienie zbioru otwartego do całej przestrzeni $ \mathcal{X} $ i jest to pojęcie dualne do zbioru otwartego. Dualną do topologii $ \mathcal{T} $ klasę zbiorów domkniętych oznaczamy $ \overline{\mathcal{T}} $. Domknięciem dowolnego zbioru $ A \subset \mathcal{X} $ nazywamy zbiór $ \overline{A} \equiv \bigcap \Bigl \{ F \in \overline{ \mathcal{T} } : A \subset F \Bigl\}$. Wnętrzem dowolnego zbioru $ B \subset \mathcal{X}$ nazywamy zbiór $ B^{\circ} \equiv \bigcup \Bigl\{ U \in \mathcal{T} : U \subset B  \Bigl \}$. Brzegiem zbioru $ C $ nazywamy zbiór $ \partial{C} \equiv \overline{C} \setminus C^{\circ}  $. Dla ustalonego punktu $ x \in \mathcal{X} $ wydzielamy z topologii $ \mathcal{T} $ te zbiory otwarte, które zawierają $ x $. Tak powstałą klasę zbiorów zwiemy otwartymi otoczeniami $ x $ i oznaczamy przez $ \mathcal{O}(x) $. 

Bazą topologii nazywamy rodzinę zbiorów taką, że za pomocą operacji brania skończonych przecięć a następnie dowolnie sumując owe przecięcia otrzymamy topologię. Dla przykładu, w przypadku skończenie wymiarowych przestrzeni euklidesowych jedną z możliwych baz jest klasa zbiorów zawierająca otwarte kule o środkach w punktach o wymiernych współrzędnych i o promieniach będących liczbami wymiernymi. W tym przypadku widzimy też, że w tak zdefiniowanej topologii jest równo $\aleph_0$ zbiorów. 

Przestrzeń $ \mathcal{X} $ nazywamy przestrzenią Hausdorffa, jeśli dla dowolnych $ a, b \in \mathcal{X} $ istnieją rozłączne otwarte $ U \in \mathcal{O}(a)$ i $ V \in \mathcal{O}(b) $. W takiej przestrzeni sensowne jest rozpatrywanie pojęcia zwartości. Zbiór $ K $ nazywamy zwartym~\footnote{Lub kompaktywnym} i piszemy $ K \in \mathrm{Comp}(\mathcal{X}) $, jeśli z dowolnego pokrycia $ K $ zbiorami otwartymi można wybrać podpokrycie skończone, czyli

\begin{equation*}
	K \in \mathrm{Comp}(\mathcal{X}) \Longleftrightarrow \underset{\mathcal{H} \subset \mathcal{T} }{\forall} \Big \{  K \subset \bigcup{\mathcal{H}}  \Longrightarrow \exists \mathcal{S} \subset \mathcal{H} \wedge |\mathcal{S}| < \aleph_0 \wedge K \subset \bigcup \mathcal{S}\Big \}.
\end{equation*}

Powyższa charakteryzacja jest równoważna temu, że z każdego ciągu puntów w zbiorze zwartym można wyjąć podciąg zbieżny do punktu z tego zbioru.

Rozważmy dwie przestrzenie topologiczne $ (\mathcal{X}, \mathcal{T}) $ i $ (\mathcal{Y}, \mathcal{S}) $. Funkcja $ F : \mathcal{X} \rightarrow \mathcal{Y} $ jest ciągła, jeśli jej przeciwobraz transportuje zbiory otwarte $ \mathcal{S} $ w zbiory otwarte $ \mathcal{T} $, czyli

$$ \underset{V \in \mathcal{S}}{\forall} \, f^{-1}[V] \in \mathcal{T}. $$

Łatwo dowodzi się, że obraz zbioru zwartego przy funkcji ciągłej jest zwarty. Dzięki temu funkcje rzeczywiste, czyli $F: K \rightarrow \mathbb{R} $ osiągają na zwartym $K$ swoje kresy. 

Zwróćmy uwagę, że na jednym zbiorze $ \mathcal{X} $ możemy określić na wiele różnych sposobów czym jakiej postaci są zbiory otwarte. Zauważmy również, że ważną podklasą topologii określonych na zbiorze $ \mathcal{X} $ są topologie generowane przez metryki, czyli takie, w których baza topologii zadana jest za pomocą zbiorów otwartych postaci $ U = \{x \in \mathcal{X}: d(x,y) < \epsilon \}$, gdzie $ d: \mathcal{X}^{2} \rightarrow [0, +\infty) $ to metryka. Ważną podklasą metryk są normy, które możemy określić na przestrzeniach zaopatrzonych w strukturę liniową. Zatem one też generują pewną topologię. Dowodzi się, że w przypadku przestrzeni skończenie wymiarowej wszystkie normy nań określone generują tę samą topologię. To może nie być prawda w przestrzeniach nieskończenie wymiarowych. Tym samym zachodzi potrzeba porównywania topologii. Topologie porównujemy w sensie teorio-mnogościowego zawierania się. W przypadku topologii zadania przez normy istnieje łatwe kryterium służące ich porównaniu: jeśli dwie normy to $ ||\circ||_1 $ oraz $ ||\circ||_2 $, oraz dla każdego $ f \in \mathcal{X} $ zachodzi $ ||f||_1  \leq C||f||_2 $ dla pewnego skalara $ C \in \mathbb{R}_{++}$. Oznaczywszy kule w odpowiednich normach jako zbiory postaci

$$ B_i (f, R) \equiv \Bigl\{	g : ||f-g||< R	 \Bigl\}.$$

Łatwo sprawdzamy, że zachodzi relacja $ B_1(f,R) \supset B_2 (f, \frac{R}{C})$. Oznacza to, że w dowolnej kuli z topologii $ \mathcal{T}_1 $ jest pewna kula z topologii $ \mathcal{T}_2 $, co oznacza, że ta druga topologia jest bogatsza (większa) od pierwszej, czyli $ \mathcal{T}_1 \subset	\mathcal{T}_2 $. Dla przykładu rozpatrzmy zbiór $ \mathcal{C}^{1}(K) $ funkcji jednokrotnie różniczkowalnych na zwartym przedziale $ K $, o skończonych normach. Możemy na tym zbiorze zdefiniować dwie normy: $ ||f||_{\infty} = \underset{x \in K}{\sup} |f(x)| $ oraz $ || f ||_{\mathcal{C}^1} = ||f||_{\infty} + ||f'||_{\infty}$, gdzie $ f' $ to pochodna funkcji $ f $. Wówczas oczywiście dla dowolnego $ f \in \mathcal{C}^{1}(K) $ zachodzi $ || f ||_{\mathcal{C}^1} \geq ||f||_{\infty}  $, co oznacza, że $ \mathcal{T}(||\circ||_{\mathcal{C}^1}) \supset \mathcal{T}(||\circ||_{\infty}) $.

Rozpatrzmy przestrzeń topologiczną $ (\mathcal{X}, \mathcal{T}) $ oraz dowolny podzbiór $ A \subset \mathcal{X} $. Przestrzeń $ (\mathcal{X}, \mathcal{T}) $ generuje nową przestrzeń -- podprzestrzeń $(A, \mathcal{T}\restriction_A )$, gdzie $ \mathcal{T}\restriction_A  \equiv \Bigl\{	A \cap U : U \in \mathcal{T} \Bigl\}$. To znaczy, że w topologii podprzestrzeni zbiorami otwartymi są ślady zbiorów otwartych otaczającej przestrzeni. W topologii podprzestrzeni zachodzą następujące proste do udowodnienia własności~\footnote{Patrz: \citet[][str. 77-78]{Dugundji}}: 
\begin{lemat}
	Jeśli $ B \subset A \subset \mathcal{X} $ i $ (\mathcal{X}, \mathcal{T}), (A, \mathcal{T}\restriction_A)$ to przestrzeń i podprzestrzeń topologiczna, to zachodzą następujące relacje:
	
$$ \overline{B}^{\mathcal{T}\restriction_A} = A \cap \overline{B} ,$$
$$ A \cap B^{\circ} \subset B^{\circ, \mathcal{T}\restriction_A},$$
$$ \partial_{\mathcal{T}\restriction_A} B \subset A \cap \partial{B}.	$$
\end{lemat}

\subsubsection*{Przestrzenie Banacha}

	Przestrzenią Banacha nazywamy przestrzeń liniową, unormowaną i zupełną w tej normie: dowolny ciąg Cauchy'ego w tej przestrzeni posiada granicę będącą elementem przestrzeni. Sztandarowym przykładem przestrzeni Banacha jest para $(\mathbb{R}^n, ||\circ||_2)$, gdzie $||x||_2 = \sqrt{\underset{1\leq i \leq n}{\sum}  x_i^2}$. Przykłady przestrzeni Banacha można mnożyć: okazuje się, że zbiór funkcji ciągłych o zwartym nośniku o wartościach w przestrzeni Banacha sam jest przestrzenią Banacha. Dzięki temu wiemy, że przestrzenią Banacha jest np. zbiór funkcji ciągłych rzeczywistych określonych na zwartym zbiorze $K$ rozpatrywany z normą supremum. 

	Ponieważ przestrzeń Banacha to pojęcie łączące struktury algebraiczne (przestrzeń liniowa) i topologiczne (zbieżność w normie), zatem sensowne jest rozpatrywanie klasycznych struktur znanych z algebry liniowej, takich jak np. suma prosta przestrzeni. Standardowym przykładem jest przestrzeń $\mathbb{R}^{n}$ będąca sumą $n$ kopii prostej rzeczywistej. Na takiej przestrzeni można określić normę biorąc sumy norm składowych. W przypadku $\mathbb{R}^{n}$ zatem normą może być wyrażenie $||x||= \underset{i = 1,\dots,n}\sum |x_i|$. 

	Mówimy, że dwie przestrzenie banacha są izomorficzne, jeżeli istnieje liniowy izomorfizm algebraiczny pomiędzy nimi, który jest homeomorfizmem. Przestrzenie Banacha są więc niezmiennikami swoich izomorfizmów: są identyczne co do izomorfizmu. Okazuje się, że interesująca z perspektywy naszej pracy przestrzeń Banacha funkcji $m$-krotnie ciągle różniczkowalnych, określonych na zwartym zbiorze $K$ o wartościach w przestrzeni Banacha $\mathbb{E}$ rozpatrywana z normą będącą sumą norm supremów wszystkich pochodnych i normy supremum funkcji, czyli $||f||_{\mathcal{C}^{m}} = \underset{k = 0,\dots, m}{\sum} || f^{(k)}||_{\infty}$ jest izomorficzna z sumą prostą $m$-kopii przestrzeni $\mathbb{E}$ i przestrzenią funkcji ciągłych z normą supremum $\Big( \mathcal{C}^0(K,\mathbb{E}), || \circ ||_{\infty}\Big)$, $\mathbb{E} \oplus \dots \oplus \mathbb{E} \oplus \mathcal{C}^0(K,\mathbb{E})$. Żeby się o tym przekonać wystarczy zauważyć, że poszukiwany algebraiczny izomorfizm jest zadany przez wzór Taylora z resztą w postaci całkowej. 


\subsubsection*{Porządki i kraty}

Częściowym porządkiem na $ \mathcal{X} $ nazywamy relację $ \geq $, która jest zwrotna, przechodnia i słabo antysymetryczna~\footnote{Szczegóły w \citet[][str. 20-28]{Turek}}. Mówimy, że dwa elementy są porównywalne, jeśli $ x \geq y $ lub $ y \geq x $. Podzbiór $ C \subset \mathcal{X} $ nazywamy łańcuchem, jeśli wszystkie jego elementy są porównywalne.

Rozważmy podzbiór $ A \subset \mathcal{X} $. Wówczas górnym ograniczeniem zbioru $ A $ nazywamy dowolny element $ x \in \mathcal{X} $ taki, że dla dowolnego elementu $ a \in A $ zachodzi $ x \geq a $. Supremum nazywamy najmniejsze względem relacji $ \geq $ ograniczenie górne zbioru. Dualnym pojęciem jest ograniczenie z dołu i infimum. Infima i suprema pewnych podzbiorów w częściowym porządku nie muszą zawsze istnieć~\footnote{Ale gdy istnieje, to jest oczywiście jednoznacznie wyznaczony}. Dlatego też wyróżniamy pewne klasy częściowych porządków. Jeśli dla dowolnej pary elementów $ x, y \in \mathcal{X} $ istnieje supremum tych elementów, oznaczane przez $ x \vee y \in \mathcal{X}$, oraz infimum, oznaczane przez $ x \wedge y \in \mathcal{X}$, to parę $ (\mathcal{X}, \geq) $ nazywamy kratą. Podkratą nazywamy dowolny podzbiór $ L \subset \mathcal{X} $, dla którego $\bigwedge B , \bigvee B \in L$. Jeżeli dla dowolnej podkraty $ B \subset \mathcal{X} $ prawdą jest, że  $\bigwedge B , \bigvee B \in \mathcal{X} $, to kratę $ \mathcal{X} $ nazywamy zupełną. Zauważmy, że dla zupełnej kraty istniej element maksymalny i minimalny w tym zbiorze~\footnote{Element maksymalny to taki element, dla którego każdy inny element w zbiorze jest od niego mniejszy}: są to $ 1_{\mathcal{X}} \equiv \bigvee \mathcal{X} $ i $ 0_{\mathcal{X}} \equiv \bigwedge \mathcal{X} $.   

Po wprowadzeniu pewnej struktury matematycznej, jak zdefiniowana przed chwilą struktura porządku częściowego, naturalne staje się badanie przekształceń, które przenoszą ową stukturę na inny zbiór. Takim działaniem dla struktury porządkowej są funkcje monotoniczne: rosnące i malejące.  W tej pracy operatorem monotnicznym będziemy nazywali wyłącznie operator ściśle rosnący w następującym sensie: $ A : \mathcal{X} \rightarrow \mathcal{Y} $ jest funkcją monotoniczną wtedy, gdy z tego, że dla $ x_1,x_2 \in \mathcal{X}  $ zachodzi $ x_1 \leq x_2 $ wynika, że $ A(x_1) \leq A(x_2)$. 

Punktem stałym dowolnego przekształcenia $ A : \mathcal{X} \rightarrow \mathcal{X} $ nazywamy $ x \in \mathcal{X} $, dla którego zachodzi $ A(x)=x $ . Zbiór punktów stałych danego operatora $ A $ będziemy oznaczali przez $ \mathrm{Fix}(A) $. Dla pewnych monotonicznych przekształceń łatwo wykazać istnienie punktu stałego. Powiemy, że odwzorowanie $A: \mathcal{X} \rightarrow \mathcal{X}$ jest ciągłe w topologii porządkowej\footnote{W literaturze zwanej także topologią strzałki}, jeśli dla każdego przeliczalnego łańcucha $\{c_i\} \subset \mathcal{X}$ takiego, że $\bigvee \{c_i\} \in \mathcal{X}$, zachodzi $A[\bigvee \{c_i\}] = \bigvee \{ A[c_i]\}$. Temu warunkowi czynią zadość wyłącznie operatory monotoniczne, bowiem jeśli $x \leq y$, to oczywiście $y = x \vee y$, a więc z ciągłości $A$ wynika, że $A[y] = A[x] \vee A[y]$. Zatem $A[y]\geq A[x]$.  

Zachodzi ważne~\footnote{Szczegóły: \citet[][str. 26-27]{Granas_FPT}}

\begin{tw}[Twierdzenie Tarskiego-Kantorowicza]\label{Tarski}
	Jeśli $ (\mathcal{X}, \geq) $ to częściowy porządek, a $ A: \mathcal{X}\rightarrow \mathcal{X} $ jest ciągłe w topologii porządkowej oraz zachodzą warunki 
	\begin{enumerate}
		\item{istnieje $ b \in \mathcal{X} $, dla którego zachodzi $ b \leq F(b) $,}
		\item{Każdy przeliczalny łańcuch $ C \subset \mathcal{X} $  posiada supremum: $ \bigvee C \in \mathcal{X} $.}
	\end{enumerate}
Wówczas $ \underset{\mu}{\exists} \mu \in \mathrm{Fix}(A) \not= \emptyset $, $ \mu = \underset{n}{\sup} A^{n}[b]$ oraz $\mu = \inf \Bigl[ \bigl\{ x : x \geq b \bigl\} \cap \mathrm{Fix}(A)\Bigl]$.
\end{tw}

\newpage

%%%%%%%%%%%%%%%%%%%%%%%%%%%%%%%%%%%%%%%%%%%%%%%%%%%%%%%%%%%%%%%%%%%%%%%%%%%%%%%%%%%%%%%%%%%%%%%%%%%%

\section{Dowody twierdzeń}

Przypomnijmy oznaczenia:

\begin{itemize}
	\item{$ K \equiv [0, \overline{x}]$}
	\item{$ F(x, s) \equiv f(x,s) + (1 - \delta)x $}
	\item{$y = y(x,X,s) \equiv f(X, s) + (x - X)f_1 (X,s)$} 
	\item{$W = W(x,X,s) \equiv (1- \phi(y,X,s))y + 1 - \delta + d(X,s) - c $}	
	\item{$R = R(x,s) \equiv \beta \Bigl \{[1 - \phi(f(x,s),x,s) - \phi_1 (f(x,s),x,s)f(x,s)]f_1 (x,s) + 1 - \delta \Bigl \} $}
	\item{$c = c(x,s) \equiv C(x,x,s)$}

\end{itemize}

\hrule

\begin{dowod}[Wyprowadzenie równania Eulera]\label{proof_Euler}
\end{dowod}

Zauważmy, że z założenia \ref{ass on u} wynika, że supremum w wyrażeniu 

\begin{equation}\label{proof_bellman1}
\mathcal{V} = \underset{c \in M}{\sup} \Biggl\{ u + \beta \mathcal{E}_s \bigl \{ \mathcal{V}(W, g, s') \bigl\} \Biggl\}
\end{equation}
	
nie może być osiągane w dla $ c = 0 $, ponieważ pochodna $ u'(0^{+}) $ przyjmuje bardzo duże wartości~\footnote{Nie jest istotne to, że $ u'(0^{+}) = + \infty$. Wartość ta może być skończona, ale skonstruowana tak, by po prostu $ c=0 $ nie było optymalnym wyborem.}. Z drugiej strony $ c \not = F $, gdyż $ F(0,s) = 0 $. Tym samym możemy skorzystać z twierdzenia Benveniste-Scheinkmana i zróżniczkować wyrażenie \ref{proof_bellman1} po zmiennej $ x $. Korzystając dodatkowo z twierdzenia o obwiedni otrzymujemy wyrażenie

\begin{equation}\label{proof_bellman_diff}
\mathcal{V}_1 = \beta	\mathcal{E}_s \bigl \{ \mathcal{V}_1 (W, g,s') \pardiff{W}{x}  \bigl \}.
\end{equation}
		
Zauważmy też, że 

\begin{equation*}
\pardiff{W}{x} = - \pardiff{\phi}{x} y  + (1 - \phi) f_1(X,s) + 1 - \delta.
\end{equation*}		
	
W równowadze możemy dodatkowo skorzystać z zeleżności $ \phi f \equiv d $ i $x \equiv X$ i $ g \equiv F - c $. Wówczas prawdziwe są zdania

\begin{gather*}
y(x,x,s) = f(x,s), \\
W(x,x,s) = F(x,s), \\
\pardiff{W}{x}(x,x,s) = - \pardiff{\phi}{x}f(x,s) + [1- \phi(f(x,s),x,s)]f_1 (x,s) + 1 - \delta.
\end{gather*}
		
Różniczkując $ \phi = \phi (w, X, s) $ po $ x $ otrzymujemy $ \pardiff{\phi}{x} = \phi_1 \pardiff{y}{x} $. Wykorzystując tę informację otrzymujemy w równowadze zdanie

\begin{equation*}
\pardiff{W}{x}(x,x,s) = - \phi_1 f_1 f + [1- \phi]f_1 + 1-\delta = [1 - \phi - \phi_1 f]f_1 + 1-\delta.
\end{equation*}
		
Wykorzystując wprowadzone oznaczenia, widzimy że powyższy zapis po pomnożeniu przez $\beta$ jest równy $R$. Uwzględniając to w równaniu \ref{proof_bellman_diff} oraz wykorzystując warunki równowagowe otrzymujemy

\begin{equation*}
V_1 (x,x,s) =  \mathcal{E}_s \bigl \{ \mathcal{V}_1 (F(x,s) - c(x,s), g(x,s), s')R(x,s)\bigl \}. 
\end{equation*}

A ponieważ $ R $ jest stała względem $ s' $ powyższe równanie możemy skrótowo zapisać

\begin{equation}\label{eq1}
V_1 = \mathcal{E}_s \bigl \{ \mathcal{V}_1 (F - c, g, s')\bigl \}R \,.
\end{equation}

Musimy teraz uwzględnić warunki pierwszego rzędu wyprowadzone przy rozwiązywaniu zagadnienia optymalizacyjnego \ref{proof_bellman1}. Są one postaci

\begin{equation*}
u'(C(x,X,s)) - \beta \mathcal{E}_s \bigl \{ \mathcal{V}_1 (W, g, s') \bigl \} = 0 \,.
\end{equation*}

Zatem w równowadze

\begin{equation}\label{eq2}
u'(c(x,s)) = \beta \mathcal{E}_s \bigl \{ \mathcal{V}_1 (F(x,s)-c(x,s), F(x,s)-c(x,s), s') \bigl \} \,.
\end{equation}

Łącząc zdania \ref{eq1} i \ref{eq2} otrzymujemy

\begin{equation*}
\mathcal{V}(x,x,s) = u'(c(x,s))R(x,s) \,.
\end{equation*}

Podstawiając powyższy wynik do równania \ref{eq2} otrzymujemy ostatecznie interesujące nas równanie równoważne równaniu Bellmana \ref{proof_bellman1}, w którym nie występuje jawnie funkcja wartości. Tego typu równania nazywa się zwyczajowo równaniami Eulera. Jest ono postaci

\begin{equation*}
u'(c(x,s)) =   \mathcal{E}_s \bigl \{ u'(c(F(x,s)-c(x,s),s'))R(F(x,s)-c(x,s),s')	\bigl \} \,,
\end{equation*}

lub, w skrócie,

\begin{equation*}
u'(c) =   \mathcal{E}_s \bigl \{ u'(c(F-c, s'))R(F-c,s')	\bigl \} \,.
\end{equation*}

\qed
\newpage

%%%%%%%%%%%%%%%%%%%%%%%%%%%%%%%%%%%%%%%%%%%%%%%%%%%%%%%%%%%%%%%%%%%%%%%%%%%%%%%%%%%%%%%%%%%%%%%%%%%%%

\hrule
\begin{dowod}[$ A : \mathcal{C}_{F} \rightarrow \mathcal{C}_{F}  $ działa na $ \mathcal{C}_F $ w sposób ciągły]\label{A_acts_on_C_F}
\end{dowod}


Przed przystąpieniem do udowodnienia tego lematu udowodnimy pomocniczy

\begin{lemat}[O ciągłości funkcji zadanej implicit\'e]\label{Tjagiwost1}
	Niech $\theta: \mathcal{X} \times \mathcal{Y} \rightarrow \overline{\mathbb{R}}$ będzie ciągła w topologii produktu~\footnote{W $ \overline{\mathbb{R}} $ definiujemy otoczenia obu nieskończoności w oczywisty sposób. Chcemy, by $ \overline{\mathbb{R}} $ było homeomorficzne z domkniętym odcinkiem}. Niech $ \mathcal{Y} $ będzie przestrzenią zwartą. Jeśli przeciwobraz zera $ \theta^{(-1)}[\{ 0 \}] $ jest podzbiorem $\mathcal{X} \times \mathcal{Y}$, który jest funkcją $ y $ od $ x $, to funkcja ta jest ciągła.
\end{lemat}
\begin{proof}
	Ponieważ $ \{0\} \in \overline{\mathcal{T}}(\mathbb{R})$, zatem $ \theta^{(-1)}[\{ 0 \}] \in \overline{\mathcal{T}}(\mathcal{X} \times \mathcal{Y})$. 
Rozważmy dowolny domknięty zbiór $ V \in \overline{\mathcal{T}}(\mathcal{Y})$. Wówczas $ \bigl[ V \times \mathcal{X}\bigl] \cap  \theta^{(-1)}[\{ 0 \}] \in \overline{\mathcal{T}}(\mathcal{X} \times \mathcal{Y}) $. Twierdzenie Kuratowskiego o rzucie względem przestrzeni zwartej~\footnote{Patrz:  \citet[][str. 152-153]{Engelking}} orzeka, że rzut tego ostatniego zbioru na $ \mathcal{X} $ jest domknięty. Jest to równoważne temu, że $ \theta $ przerzuca zbiory domknięte w $ \mathcal{Y} $ w zbiory domknięte w $ \mathcal{X} $. Zatem robi to samo ze zbiorami otwartymi, więc jest to przekształcenie ciągłe. 
\end{proof}

Możemy teraz przejść do właściwej części dowodu \ref{A_acts_on_C_F}.

Rozpatrzmy najpierw przypadek, gdy funkcja konsumpcji poza zerem jest ściśle dodatnia, czyli $ c[K_{++}] >0 $. Połóżmy:
$$ \theta_c (y,x,s) =  \rho_c (F(x,s) - y) - u'(y)$$
Przy dowolnym ustalonym $ s $ jest to ciągłe przekształcenie zmiennych $ x $ i $ y $. Nadto, jest to funkcja ściśle malejąca z $ x $ i ściśle rosnąca z $ y $. Gdy $ y $ jest bliskie zera, $ \theta_c $ jest ujemna, a dla $ y $ bliskich $ F(x) $ -- dodatnia, co wynika z założeń o funkcji użyteczności. Na mocy własności Darboux prawdziwej dla funkcji ciągłych oraz na mocy monotoniczności $ \theta_c $ dla dowolnego ustalonego $ x $: istnieje tylko jedno $ y $ takie, że $ \theta_c (x, y(x,s)) = 0$. Zatem $ y $ to funkcja $ x $ dla każdego $s$. Z lematu \ref{Tjagiwost1} wynika teraz ciągłość $ y = y(x,s) $. Monotoniczność $ \theta_c $ względem $ x $ implikuje, że $ y = y(x,s) $ jest funkcją ściśle rosnącą z $ x $ dla każdego $ s $. Nadto, zachodzi też:

$$ \rho_c (F(x,s)-y(x,s))= u'(y(x,s))\,.$$

więc jako że $ u' \circ y $ ściśle maleje z $ x $, oraz ponieważ $ \rho_c $ jest ścisle malejącą funkcją, tedy funkcja $ F(x,s) - y(x,s) $ jest ściśle rosnąca z $ x $. 

Jeśli istnieje pewien nietrywialny przedział $ W \not = K$~\footnote{A z monotoniczności $ c $ wynika, że nie może to być inny zbiór}, na którym funkcja konsumpcji się zeruje, czyli $ c[W] = {0} $, wówczas pewna część grafu tej funkcji przy działaniu $ A $ pozostaje na miejscu, druga część natomiast podnosi się. Wynika to stąd, że w definicji $ A $ rozpatrujemy złożenie $ c $\, z $ F(x) - y $, a y wybieramy z przedziału $ (0, F(x)) $. Tym samym interesuje nas przypadek graniczny, gdy $ c(F(x)) = 0 $. Niech $ x_0 = \max (c\circ F)^{(-1)}[\{ 0\}] = \max (F^{(-1)} \circ c^{(-1)})[\{ 0\}]$. Wówczas zachodzi $ c[ [0,x_0] ] = \{0\}$. Ze ścisłej wklęsłości $ F $ wynika jednak, że $ x_0 \in W^{\circ} $. 

Wreszcie, gdy $ c \equiv 0 $, z definicji uzyskujemy $ A[c] \equiv 0$.

Zbiór $ \mathcal{C}_F $ jest zbiorem zwartym w normie supremum. Nadto, gdy $|| c_n - c ||_\infty \rightarrow 0$ to obserwujemy zbieżność punktową wartości operatora: $ A[c_n](y,s) \rightarrow A[c](y,s)$. Tym samym zachodzi również zbieżność wartości operatora w normie supremum: odpowiednie twierdzenie można znaleźć u Roydena (\citeyear[][lemat 39, str. 168]{Royden}). Tym samym $ || A[c_n] - A[c] ||_\infty \rightarrow 0 $. Zatem $ A $ jest ciągłym operatorem w topologii normy supremum~\footnote{Rozróżnianie topologii, w których obserwujemy zbieżność okazuje się ważne w kontekście rozdziału \ref{chap_diff}}.  

\qed
\newline
  %%%%%%%%%%%%%%%%%%%%%%%%%%%%%%%%%%%%%%%%%%%%%%%%%%%%%%%

\hrule
\begin{dowod}[Jeśli $ f : \mathbb{R} \rightarrow \mathbb{R}$ to funkcja ciągła i ściśle monotoniczna, to jest to homeomorfizmem, czyli $ f^{(-1)} $ jest również ciągła.]\label{proof_homeomorphism}
\end{dowod}

	Bazą topologii euklidesowej w $ \mathbb{R} $ są otwarte przedziały. Ich zbiór oznaczmy przez $ Intv $. Sprawdzenie, że jeśli $ A \in Intv $, to $ f[A] = (f^{(-1)})^{(-1)} \in Intv $, gdy $ f $ jest ciągłe jest proste. Dalej: ponieważ $f$ jest ściśle monotoniczna, to jest iniekcją. Zatem obraz funkcji $ Im(f): \mathcal{P}(\mathcal{R}) \rightarrow \mathcal{P}(\mathcal{R}) $  jest operacją przemienną z teorio-mnogościowymi sumą i przecięciem zbiorów~\footnote{Dokładnie: jest to homomorfizm algebr Boole'a}:
	
\begin{gather*}
	f[A \cup B] = f[A] \cup f[B]\,, \\
	f[A \cap B] = f[A] \cap f[B]\,. 
\end{gather*}	

Tym samym baza topologii w dziedzinie zostaje przez $ f $ przetransponowana na bazę topologii obrazu. Zatem $ f^{(-1)} $ jest ciągła.
	
\qed
\newline

%%%%%%%%%%%%%%%%%%%%%%%%%%%%%%%%%%%%%%%%%%%%%%%%%%%%%%%%%%%%%%%%%%%%%%%%%%%5

\hrule
\begin{dowod}[Zbiór funkcji $W \in \{\mathcal{C}_F,\mathcal{M}  \} $ to krata zupełna względem relacji porządku zadanego następująco: dla $ f,g \in W$ to
 $ f \geq g \leftrightarrow \underset{x \in K}{\forall} f(x) \geq g(x) $. 
Wówczas też $ 1_{\mathcal{C}_F} = F $, $ 0_{\mathcal{C}_F} = 0 $, $ 1_{\mathcal{M}} = H^{-1}\circ F $ oraz $ 0_{\mathcal{M}} = 0 $. Nadto, operator $ A:\mathcal{C}_F \rightarrow \mathcal{C}_F $ to operator monotoniczny na $ \mathcal{C}_F $, a operator $ B:\mathcal{M} \rightarrow \mathcal{M} $ to operator monotoniczny na $ \mathcal{M} $.  
]\label{proof_lattices}
\end{dowod}

Aby udowodnić zupełność krat skorzystamy dodatkowo z następującego lematu \citep[][rozdział 9.2.4, twr. 9.40, str. 429]{Carl} 

\begin{lemat}\label{lemat_o_supremach}
	Niech $ E $ to unormowana i uporządkowana przestrzeń, $ \mathcal{X} $ to przestrzeń topologiczna, $ C $ to rodzina funkcji równociągłych\footnote{Tzn. dla dowolnego $ x \in \mathcal{X} $ istnieje $ U \in \mathcal{T}(\mathcal{X}) $, że dla wszystkich $ f \in C $ zachodzi $ \mathrm{diam}\,f[U] \leq \epsilon $} w zbiorze ciągłych funkcji przeprowadzających $ \mathcal{X} $ w $ E $, $\mathcal{C}(\mathcal{X},E)$. Niech punktowe granice rosnących ciągów z $ C $ istnieją dla każdego $ x \in \mathcal{X} $. Wówczas $ \bigvee C  \in \mathcal{C}(\mathcal{X},E)$    
\end{lemat}

	Jako że rozpatrywane zbiory składają się z funkcji jednociągłych, których obraz zawiera się w $ K $, który jest uporządkowanym podzbiorem $ \mathbb{R} $, oraz jako że punktowe granice rosnących i ograniczonych ciągów liczbowych istnieją, zatem lemat \ref{lemat_o_supremach} implikuje, że supremum dowolnej podkraty jest elementem kraty. Powyższy lemat można w oczywisty sposób zastosować do stwierdzenia, że również infima są elemetnami kraty. Zatem kraty  $ \mathcal{C}_F,\mathcal{M}   $ są zupełne.
	
$ B[m](x) $ jako funkcja powstaje poprzez rozpatrzenie rzutów na oś odciętych punktów przecięcia hiperboli $ \frac{1}{y} $ z funkcją $ \tilde{\rho}_m (y,s) \equiv \mathcal{E}_s \Bigl \{\Bigl[ \frac{R}{m} \Bigl] (F(x,s)- H(y), s') \Bigl \} $. Rozpatrując funkcję $n$ mniejszą od $ m $, czyli taką, że dla każdego $x \in K$ zachodzi $ n(x) \leq m(x) $ zauważamy, że nowe punkty przecięcia znajdują się wyżej przy każdym ustalonym $x $, patrz Rysunek \ref{rys2}. Tym samym ich rzuty na oś odciętych będą miały mniejsze wartości od tych wyznaczonych przez krzywą $m$. Odpowiada to temu, że $ B[n] \geq B[m] $. 

W końcu, korzystając ze ścisłej monotoniczności funkcji $ H $ i $ H^{(-1)} $, która sprzęga oba operatory, widzimy, że jeśli $ m_1 \leq m_2 $, to tym samym $ A[H(m_1)] \leq A[H(m_2)] $, czyli $ A[c_1] \leq A [c_2]$. Wynika to z tego, że $ c_i \equiv H(m_i) $, jak i również $ B[m_1] \leq B[m_2] $ wtedy i tylko wtedy, gdy $ [H^{(-1)} \circ A \circ H](m_1) \leq [H^{(-1)} \circ A \circ H](m_2) $. Przykładając z prawej strony $H$ otrzymujemy żądane wyrażenie. 

Postać jedynek i zer rozpatrywanych krat jest oczywista. 


\qed
\newline
%%%%%%%%%%%%%%%%%%%%%%%%%%%%%%%%%%%%%%%%%%%%%%%%%%%%%%%%%%%%%%%%%%%%%%%%%%%5

\hrule
\begin{dowod}[Przekształcenie $B$ jest ciągłe w topologii porządkowej. Istnieje element $n \in \mathcal{M}$, dla którego zachodzi $B\brackets{n} \leq n$. Nadto, iterując przykładanie operatora $B$ do elementu $n$ uzyskujemy ciąg zbieżny w normie supremum do punktu stałego $B$, który jest największym elementem w podkracie $\mathrm{Fix}(B)$ kraty $\mathcal{B}$.]\label{proof_Tarski}
\end{dowod}

	Przypomnijmy, że ciągłość w normie porządkowej oznacza, że $B[\bigvee \{m_i\}] = \bigvee \{ B[m_i]\}$ , dla dowolnego przeliczalnego łańcucha $\{ m_i \}$. Zatem rozważmy taki łańcuch i zauważmy, że można jego elementy ustawić w ciąg elementów monotonicznie rosnących: 
	
\begin{equation*}
\dots \leq m_0 \leq m_1 \leq \dots \leq m_n \leq \dots
\end{equation*}	
	 
	Bez straty ogólności możemy badać tylko zachowanie podciągu:  $ \{m_i \}_{i=1}^{\infty}$. Wówczas oczywiście supremum tego ciągu w rozpatrywanym porządku to funkcja $m(x) \equiv \underset{i \rightarrow \infty}{\lim} m_i (x)$.	Na podstawie lematu \ref{lemat_o_supremach} powyższa granica jest funkcją ciągłą, a co wyniki z definicji granicy -- jest to funkcja jednoznacznie wyznaczona. Na mocy cytowanego już lematu~\footnote{Patrz końcówka dowodu lematu \ref{on A}}, granica punktowa funkcji ze zbioru zwartego w normie supremum jest również granicą w normie supremum. Zatem $\{ m_i \}$ to ciąg zbieżny do jednej granicy w normie supremum. Z ciągłości $B$ w tejże normie wnioskujemy, że również $B[m_i] \underset{\sup}{\rightarrow} B[m]$. Ale z monotoniczności $B$ wnioskujemy, że $B[m]$ musi być granicą punktową ciągu $\{ B[m_i]\}$, zatem także i supremum w rozpatrywanym porządku $B[\bigvee \{m_i\}] = \bigvee \{ B[m_i]\}$ . W analogiczny sposób dowodzimy, że również $B[\bigwedge \{m_i\}] = \bigwedge \{ B[m_i]\}$. 
		
		
	Jako element $n$ bierzemy po prostu $H^{(-1)}(F)$. Dla owego elementu musi zachodzić 

	$$B[H^{(-1)}(F)](x) < H^{(-1)}(F(x,s))$$

dla każdego $x \in K, s \in S$, bowiem w przeciwnym razie musiałoby być prawdą, że
	
$$ \infty > \frac{1}{H^{(-1)}(F)} = \mathcal{E}_s \Bigl \{ \frac{R(F-F)],s')}{m(F-F), s')}	\Bigl \} = \infty  \,.$$

Tym samym faktycznie $B[H^{(-1)}(F)] \leq H^{(-1)}(F)$. 

Przyjmijmy oznaczenie $n \equiv H^{(-1)}(F)$. Pozostaje dostosować dowód twierdzenia Tarskiego-Kantorowicza\footnote{Patrz: \citet[][str. 26]{Granas_FPT}} do naszych wymagań. Ponieważ $B[n] \leq n$, zatem również $B^{k+1}[n] \leq B^{k}[n]$ dla każdego $k$. Zatem zbiór $\Bigl \{ B^{k}[n] : k \in \mathbb{N} \Bigl \}$ to przeliczalny łańcuch. Korzystając ze zwartości $\mathcal{M}$ w topologii porządkowej, widzimy, że musi on posiadać infimum. Korzystając z $B[\bigwedge \{m_i\}] = \bigwedge \{ B[m_i]\}$ zauważamy, że

$$ \bigwedge \Bigl\{ B^{k}[n] \Bigl\} = B\Bigl[ \bigwedge B^{k-1}[n] \Bigl] =  B\Bigl[ \bigwedge B^{k}[n] \Bigl]\,,$$      

skąd wynika, że $\bigwedge \{ B^{k}[n]\} \in \mathrm{Fix}(B)$.  	

\qed
\newline
%%%%%%%%%%%%%%%%%%%%%%%%%%%%%%%%%%%%%%%%%%%%%%%%%%%%%%%%%%%%%%%%%%%%%%%%%%%5


\hrule
\begin{dowod}[Przy dodatkowym założeniu \ref{ass_on_R} istnieje ściśle dodatni punkt stały.]\label{proof_existence_of_a_fp}
\end{dowod}


W tym przypadku dowód przeprowadzimy posługując się operatorem $A$ z pracy Colemana (\citeyear{Coleman1}). 

	Przypomnijmy, że na mocy założenia \ref{ass_on_R2} istnieje $x_0$ oraz $\alpha$ takie, że dla wszystkich $s\in S $ zachodzi $0 < \alpha < F(x_o, s) - x_0$ oraz $\mathcal{E}_s \bigl \{ R(x_0,s')	\bigl \} \leq 1 $. Pokażemy najpierw, że jeśli $\underset{s \in S}{\forall} c(x_0, s) \geq \alpha$, to również $A[c](x_0, s) \geq \alpha$. Podobnie jak w dowodzie lematu \ref{on A} rozpatrzmy funkcję $ \theta_c (y,x,s) =  \rho_c (F(x,s) - y) - u'(y)$, której izokwanta na poziomie $0$ wyznacza operator $A$. Jako że $\rho_c$ rośnie z $y$ oraz $\rho \bigl( A[c](x_0,s) \bigl) = 0$, zatem to, czy $\alpha$ jest mniejsza od $A[c](x_0,s)$ możemy zbadać sprawdzając, czy $\rho_c (\alpha) \leq 0 $. To jest równoważne temu, że
	
\begin{equation}\label{ineq1}
	u'(\alpha) \geq \mathcal{E}_s \Bigl \{ u'\bigl(c(F(x_0,s) - \alpha, s')) \bigl) R(F(x_0,s) - \alpha,s')	\Bigl \}\,.
\end{equation} 

Ale wiemy, że $F(x_0, s) - \alpha > x_0$ oraz $c(x_0, s) \geq \alpha$. Stąd prawa strona powyższej nierówności jest zdominowana przez $\mathcal{E}_s \Bigl \{ u'(\alpha)R(x_0,s')\Bigl \}$. Wystarczy zatem pokazać, że owo wyrażenie jest mniejsze od prawej strony \ref{ineq1}, zatem 

$$ u'(\alpha) \geq \mathcal{E}_s \Bigl \{ u'(\alpha)R(x_0,s')\Bigl \}\,.$$

Ponieważ $u'(\alpha) \in (0, \infty)$, zatem powyższe równoważne jest

$$ 1 \geq  \mathcal{E}_s \Bigl \{R(x_0,s')\Bigl \}\,.$$

A to jest jedno z poczynionych założeń. Tym samym maksymalny punkt stały nie może być równy $0_{\mathcal{C}_F}$.

Jak wynika z rozważań poczynionych w Dowodzie \ref{A_acts_on_C_F}, graf dowolnej funkcji $c$, który częściowo pokrywa się osią odciętych układu współrzędnych,  zostaje przy działaniu $A$ podniesiony, tak że na coraz mniejszym fragmencie dotyka on owej osi. Tym samym żadna funkcja, której graf dotyka osi odciętych w sposób nietrywialny~\footnote{Graj każdej funkcji dotyka osi odciętych w zerze: konsumpcja przy zerowym nakładzie kapitału jest zerowa}, nie może być punktem stałym operatora $A$. Tym samym rozpatrywany punkt stały jest funkcją ściśle większą od zera.  


\qed
\newline
%%%%%%%%%%%%%%%%%%%%%%%%%%%%%%%%%%%%%%%%%%%%%%%%%%%%%%%%%%%%%%%%%%%%%%%%%%%5

\hrule
\begin{dowod}[Operator $x_0$-monotoniczny i pseudo-wklęsły $B$ posiada co najwyżej jedną, niezerową na $W_{++}$ równowagę.]\label{proof_uniqueness}
\end{dowod}

	Załóżmy, że istnieją dwa różne punkty stałe $m, n \in \mathrm{Fix}(B)$ i $m\restriction_{W_{++}} > 0$ oraz $n\restriction_{W_{++}} > 0$. Niech będą to te punkty stałe, o których wspominamy w punkcie \ref{cond3} definicji $x_0$-monotoniczności, czyli $x_0 = \max \{ x_0 (m), x_0(n) \}$~\footnote{Biorąc maksimum z dwóch punktów "typu $x_0$" uzyskujemy punkt "typu $x_0$" dla obu funkcji} oraz dla każdego $s$ iloraz $\frac{m(\circ, s)}{n(\circ, s)}$ jest na $[x_0, \sup W]$ zawarty w skończonym przedziale zawartym w $\mathbb{R}_{++}$. 
	 
	
	Przypuśćmy na razie, że funkcje $m$ i $n$ różnią się na zbiorze $[x_0, \sup W] \times S$. Bez straty ogólności możemy załóżyć, że istnieje punkt $(w,z) \in [x_0, \sup W] \times S$, dla którego $m(w,z) < n(w,z)$~\footnote{Gdyby tak nie było, to obie funkcje byłyby albo sobie równe, co kończyłoby dowód, albo $m(w,z) > n(w,z)$ i wystarczyłoby zmienić nazwy funkcji}. 
	
	Będziemy chcieli teraz tak przeskalować $n$, żeby jej graf znalazł się pod grafem $m$ na odcinku $[x_0, \sup W] \times \{ z\}$.
	
	Na mocy warunku \ref{cond3} z definicji $x_0$-monotoniczności $B$~\footnote{Pojawiające się we wspomianym warunku założenie o ograniczeniu wartości ilorazu jest potrzebne, albowiem a priori nie wiemy, która z rozpatrywanych funkcji będzie większa}, dla tak dobranego $z $ istnieje punkt $\omega_{z} \in [x_0, \sup W]$ spełniający
  
\begin{equation*}
	\underset{x \in [x_0, \sup W]}{\inf} \Bigl[ \frac{m(x,z)}{n(x,z)}\Bigl] =  \frac{m(\omega_z, z)}{n(\omega_z,z)} \equiv t_z\,,
\end{equation*}

oraz $t_z >0$. Stąd dla wszystkich $x \in [x_0, \sup W]$ spełnione jest $m(x,z) \geq t_s n(x,z)$, z równością dla pewnego punktu $(\tilde{x}, z)$. Jako że $B$ jest $x_0$-monotoniczny, prawdą jest również, że przyłożenie do funkcji $t_z n(\circ, s)$ operatora $B$ nie podniesie jej grafu ponad graf $m(\circ,z)$ na przedziale $[x_0, \sup W]$, czyli będzie  $m(x,z) \geq B[t_z n](x,z)$. Zauważmy również, że z konieczności $t_z \in (0,1)$. Gdyby bowiem $t_z \geq 1$, to dla wszystkich $x \in [x_0, \sup W]$ zachodziłoby $m(x,s) \geq n(x,s)$, co jest sprzeczne z założeniem z początku dowodu. 

Skoro więc $t_{z} \in (0,1)$, to możemy skorzystać z pseudo-wklęsłości operatora $B$. Wówczas dla wszystkich $x \in [x_0, \sup W]$ musiałoby zachodzić

\begin{equation*}
	m (x,z) \geq B[t_{z} n](x,z) > t_{z} B[n](x,z) = t_{z} n(x,z) \,,
\end{equation*}   	

bowiem $n = B[n]$. Wiadomo jednak, że istnieje para $(w,z)$ w powyższym zbiorze, dla której skrajnie prawa i skrajnie lewa część powyższego napisu są równe. To jest sprzeczność.

Pozostaje rozpatrzeć przypadek, gdy funkcje $m$ i $n$ są tożsamościowo równe na zbiorze $[x_0, \sup W] \times S$, oraz istnieje punkt $(w,z) \in (0, x_0] \times S$, na którym są różne. Dopiero w tym momencie wykorzystujemy to, że możemy porównywać grafy funkcji również dla wszystkich wartości $x_1$ mniejszych od $x_0$ ~\footnote{Faktycznie, na pierwszy rzut oka wprowadzanie do defincji punktu $x_1$ mogłoby się wydać zbyteczne}. Wystarczy teraz za $x_1$ obrać dowolny taki punkt, że dla pewnego $z \in S$ zachodzi $n(x_1,z) > m(x_1,z)$. Rozpatrujemy teraz zwarty przedział $[x_1, x_0]$ i stwierdzamy, że w tym przedziale istnieje $\omega$ dla której 

$$\underset{x \in [x_1, x_0]}{\min} \frac{m(x,z)}{n(x,z)} = \frac{m(\omega,z)}{n(\omega,z)} \,.$$

I dalej powielamy już przeprowadzone rozumowanie. To kończy dowód.
\qed
\newline
%%%%%%%%%%%%%%%%%%%%%%%%%%%%%%%%%%%%%%%%%%%%%%%%%%%%%%%%%%%%%%%%%%%%%%%%%%%5


\hrule
\begin{dowod}[Operator $B: \mathcal{M} \rightarrow \mathcal{M}$ jest pseudo-wklęsły]\label{proof_quasi_concavity}
\end{dowod}


	Niech $t \in (0,1)$, $m \in \mathcal{M}$. Rozpatrujemy $m$ tylko na przedziale $K_{++}$. Rozpiszmy wyrażenia definiujące $B[m]$ oraz $B[tm]$
	
\begin{gather*}
	\frac{1}{B[m]} 	= \mathcal{E}_s \Bigl[ \frac{R}{m}\Bigl]	\Bigl( F - H(B[m])\Bigl) \,,\cr
	\frac{1}{B[tm]} = \mathcal{E}_s \Bigl[ \frac{R}{tm}\Bigl]	\Bigl( F - H(B[tm])\Bigl) \,,\cr 
	\frac{1}{B[tm]} =  \frac{1}{t}\mathcal{E}_s \Bigl[ \frac{R}{m}\Bigl]	\Bigl( F - H(B[tm])\Bigl)\,. \cr 
\end{gather*}	

	Oczywiście $B[tm] > tB[m]$ jest równoważne $\frac{1}{B[tm]}< \frac{1}{tB[m]}$. Na mocy powyższych wzorów, musi zatem zachodzić:
	
\begin{equation}
\mathcal{E}_s \Bigl[ \frac{R}{m}\Bigl]	\Bigl( F - H(B[m])\Bigl)  > \mathcal{E}_s \Bigl[ \frac{R}{m}\Bigl]	\Bigl( F - H(B[tm])\Bigl)\,,
\end{equation}
	
	a to jest prawda, bowiem $tm < m$ oraz funkcja $\mathcal{E}_s \Bigl[ \frac{R}{m}\Bigl]	\Bigl( F - H(y)\Bigl)$ ściśle rośnie z y.	
	



\qed
\newline

%%%%%%%%%%%%%%%%%%%%%%%%%%%%%%%%%%%%%%%%%%%%%%%%%%%%%%%%%%%%%%5

\hrule
\begin{dowod}[Operator $B: \mathcal{M} \rightarrow \mathcal{M}$ posiada własność $x_0$-monotoniczności]\label{proof_x0_monotonicity_of_B}
\end{dowod}

	Najpierw pokażemy, że dla dowolnego $m\in \mathrm{Fix}(B)$, $m\restriction_{K_{++}}$ istnieje pewien punkt $x_0 >0$, dla którego dla wszystkich $(x,s) \in [0, x_0]\times S$ zachodzi $F(x,s)- H(m(x,s)) \geq x$. 
	Załóżmy bowiem, że tak nie jest. Wówczas dla każdego $x_0 \in K_{++}$ istniałby punkt $(w,z) \in [0, x_0] \times S$, dla którego $F(w,z)-H(m(w,z)) < w$. Naśladując argument z dowodu lematu~\footnote{O ścisłej monotoniczności funkcji z argumentem $y$ funkcji $\mathcal{E}_s \Bigl[ \frac{R}{m}\Bigl]	\Bigl( F - H(y)\Bigl)$} \ref{on_pseudo_concavity_of_B} stwierdzamy, że musi być wówczas prawdą to, że 
	
\begin{equation*}
	\frac{1}{m(w,z)} > \mathcal{E}_s \Bigl[\frac{R}{m} \Bigl]\Bigl( w, s'\Bigl)\,.
\end{equation*}	
	
Ponieważ powyższe prawdziwe jest dla dowolnego $x_0$, możemy wziąć ów punkt na tyle mały, aby, korzystając z ciągłości $R$, wyrażenie 

\begin{equation*}
	R(\tilde{w}, s')\pi(s'|s) > 1
\end{equation*}

było prawdziwe dla wszystkich $\tilde{w} \in [0,x_0]$. Kładąc $\tilde{w} = w$ uzyskujemy tym samym, że 

\begin{equation*}
	\frac{1}{m(w,z)} > \mathcal{E}_s \Bigl[\frac{R}{m} \Bigl]\Bigl( w, s'\Bigl) > \underset{s' \in S}{\sum} \frac{1}{m(w,s')}\,,
\end{equation*} 
	
co jest jawną sprzecznością. Tym samym prawdą jest, że dla dowolnego $m\in \mathrm{Fix}(B)$, $m\restriction_{K_{++}}$ istnieje pewien punkt $x_0 >0$, dla którego dla wszystkich $(x,s) \in [0, x_0]\times S$ zachodzi $F(x,s)- H(m(x,s)) \geq x$. 	
	
	Wybierzmy teraz pewien punkt $x_1 \in [0,x_0]$ oraz funkcję $n \in \mathcal{M}$, której graf leży pod grafem $m$ na przedziale dla $x \geq x_1$ i wszystkich $s\in S$. Ponieważ funkcja $F(x,s)-H(m(x,s))$ jest ściśle rosnąca \footnote{Bo $F(x,s)- c(x,s)$ taka jest}, zatem $F(x,s)-H(m(x,s))\geq x_1$ na rozpatrywanym prostokącie $[x_1, \infty) \times	S$. Z doboru $n$ widzimy więc, że również $m\bigl( F(x,s) - m(x,s), s' \bigl) \geq n\bigl( F(x,s) - m(x,s), s' \bigl)$, skąd
	
	$$ \frac{1}{m(x,s)} \leq \mathcal{E}_s \Bigl[ \frac{R}{n} \Bigl] \Bigl( F(x,s) - m(x,s)\Bigl)\,.$$
	
  Powołując się ponownie na argument z Dowodu \ref{proof_quasi_concavity} stwierdzamy, że oznacza to po prostu tyle, że na prostokącie $[x_1, \infty) \times S$ zachodzi $m \geq B[n]$, co było do udowodnienia.

\qed
\newline
%%%%%%%%%%%%%%%%%%%%%%%%%%%%%%%%%%%%%%%%%%%%%%%%%%%%%%%%%%%%%%%%%%%%%%%%%%%5

\hrule
\begin{dowod}[Równanie $m(x) = m\big( F(x) - H(m(x)) \big)$ nie posiada rozwiązania w $B\brackets{\mathcal{M}\setminus\{ 0_{\mathcal{M}}\}}$]\label{proof_no_solution}
\end{dowod}

Ponieważ $F(x) - H(m(x)) = F(x) - c(x)$, więc wyrażenie to jest ściśle rosnące. Zatem można przyłożyć do obu stron $m(x) = m\big( F(x) - c(x) \big)$ odwrotność $m$. Zatem 

\begin{equation*}
	x = F(x) - c(x)\,,
\end{equation*}

skąd $c(x) = F(x) - x$. To jest sprzeczność, bowiem $c$ jest ściśle rosnąca, a $F - \mathrm{Id}$ najpierw rośnie, a potem maleje.

\qed
\newline


%%%%%%%%%%%%%%%%%%%%%%%%%%%%%%%%%%%%%%%%%%%%%%%%%%%%%%%%%%%%%%%%%%%%%%%%%%%5

\hrule
\begin{dowod}[Dla funkcji $m \in \Bigl(\mathcal{M}\setminus\{ 0_{\mathcal{M}}\}\Bigl)\cap \mathcal{C}^{m}(K)$, funkcja $B\brackets{m}$ też jest klasy $\mathcal{C}^{m}(K)$, o ile funkcje $R, F, H$ są klasy $\mathcal{C}^{m}(K)$.]\label{proof_diff_of_elements_of_M_with_C}
\end{dowod}

	Operator $B$ w przypadku deterministycznym dany w następujący sposób

\begin{equation*}
	B[m] =   \Bigl [ \frac{m}{R}	\Bigl ] \bigl (F-H(B[m]) \bigl ) \,.	
\end{equation*} 

	Pomimo uproszczenia, ów operator nadal zdefiniowany jest {\it implicit\'e}. 

	Najłatwiejsza droga do stwierdzenia różniczkowalności $B[m]$ wiedzie teraz przez rozpatrzenie następującej funkcji $\Omega: K \times K \rightarrow K$ 
	
\begin{equation*}
	\Omega(x,y) \equiv \Bigl [ \frac{m}{R}	\Bigl ] \bigl (F(x)-H(y) \bigl ) - y\,.
\end{equation*}
	
	Wiemy już, że $\Omega(x, y(x)) = 0$ definiuje funkcję $y(x) = B[m](x)$ . Chcemy pokazać, że $0$ jest wartością regularną $\Omega$, czyli że różniczka tej funkcji jest surjekcją. Wtedy bowiem moglibyśmy skorzystać z twierdzenia o funkcji uwikłanej. Policzmy więc pochodne cząstkowe wchodzące w skład macierzy odpowiadającej różniczce $\diff\Omega$:
	
\begin{equation}\label{partial_diffs}
\begin{split}
 	\pardiff{\Omega}{x} &= 	F'(x)\Bigl [ \frac{m'R - m R'}{R^2}	\Bigl ] \bigl (F(x)-H(y) \bigl )\,, \\
 	\pardiff{\Omega}{y} &=  -H'(y)\Bigl [ \frac{m'R - m R'}{R^2}	\Bigl ] \bigl (F(x)-H(y) \bigl )-1 \,. \\
\end{split}
\end{equation}

	Zauważmy, że z założenia \ref{ass on prod} o postaci funkcji produkcji $f$ wynika, iż dla $x \in K$ zachodzi $F'(x) > 0$. Jest tak bowiem $f$ jest ściśle wklęsła i ściśle rosnąca. Nadto, na mocy rozważań z rozdziału \ref{conjugate_operator} wiemy, że również $H'$ jest dobrze na $K$ określone i większe od zera. Pozostaje zbadać zachowanie wyrażenia $\frac{m'R - m R'}{R^2}$. Funkcje z $\mathcal{M}$ są niemalejące, zatem $m' \geq 0$ na $K$. Z definicji zaś $R>0$. Na mocy założenia \ref{ass_on_R}, $R'<0$. Tym samym $\frac{m'R - m R'}{R^2} > 0$ na $K^{\circ}$. Widzimy tym samym, że właśnie założenie \ref{ass_on_R} domyka dowód tego, że $\mathrm{rank}[\diff{\Omega}] = 1$, bowiem $\pardiff{\Omega}{y} < 0$. Tym samym możemy skorzystać z twierdzenia o funkcji uwikłanej~\footnote{Treść i dowód tego twierdzenia wyłożona jest w każdym szanującym się podręczniku do analizy matematycznej. Np., patrz: \citet[][str. 40]{Spivak}} i zapisać {\it explicit\'e} wzór na pochodną funkcji $B[m]$:
	
\begin{equation*}
	B[m]'(x) \equiv	-\frac{\pardiff{\Omega}{x}}{\pardiff{\Omega}{y}} \Bigl( x, B[m](x)\Bigl) = \frac{F'(x)}{H'(B[m](x)) + \frac{1}{\Bigl [ \frac{m'R - m R'}{R^2}	\Bigl ] \bigl (F(x)-H(B[m](x)) \bigl )}}.
\end{equation*}

Pozostaje pokazać, że powyższa funkcja ciągle przedłuża się w zerze i w $\bar{x}$: tylko wtedy bowiem $B[m] \in \mathcal{C}^{1}(K)$. To, że jest to prawda w $\bar{x}$ wynika z tego, że wówczas na pewno $m(\bar{x})>0$. W zerze natomiast: jeśli $m'(0) > 0$, to jest to oczywiste. Jeśli zaś $m'(0)=0$, to oczywiście $B[m]'(0)$ jest równe zero, bowiem tyle należy się spodziewać w granicy, gdyż mianownik wyrażenia opisującego $B[m]'(x)$ dąży wówczas do $+\infty$. To kończy dowód.	
	
\qed
\newline

%%%%%%%%%%%%%%%%%%%%%%%%%%%%%%%%%%%%%%%%%%%%%%%%%%%%%%%%%%%%%%%%%%%%%%%%%%%5

\hrule
\begin{dowod}[Niech $ U \in \mathcal{T}(\mathbb{R}^n) $ to zbiór ograniczony i spójny, taki że 
$ c\brackets{U} \equiv \sup \Bigl\{  \frac{\lambda(x,y)}{|x-y|} : x,y \in U, x \not= y \Bigl\} < \infty$, gdzie $ \lambda(x,y) = \inf\Bigl\{ \int_{\gamma_{x,y}}1 : \gamma_{x,y} - \mathrm{krzywa\,klasy\,}\mathcal{C}^1, \, \gamma_{x,y}(0)=x \,, \gamma_{x,y}(1)=y \Bigl\} $. 
Wówczas $ g \mapsto l_U \brackets{g} $ jest ciągła z przestrzeni $ \mathcal{C}^1 (\overline{U}) $ wyposażoną w półnormę $ p(g) = \underset{1 \leq k \leq n}{\sum} || \pardiff{g}{x_k} ||_{\infty}$ w liczby rzeczywiste.]\label{proof_continuous in c1}
\end{dowod}

W dowodzie będziemy pomijać indeksowanie zbiorem $ U $.

	Niech $ \{ g_n \} \subset \mathcal{C}^1(\overline{U}) \ni g$ oraz $ \lim p(g_n - g) = 0 $. Niech $ \epsilon > 0 $. Z definicji infimum wiemy, że istnieją $ \tilde{x}, \tilde{y} \in \overline{U} $, dla których 
	
	$$ \frac{|g(\tilde{x})-g(\tilde{y})|}{|\tilde{x}-\tilde{y}|} < l[g] + \frac{\epsilon}{2} \,.$$	

Natomiast ze zbieżności $ \{ g_n \} $ wynika, że istnieje $ N $, że dla wszystkich $ n \geq N $ zachodzi

$$ p(g_n - g) < \frac{\epsilon}{2 c[U]} \,.$$

Zatem też z definicji \ref{l_function} funkcji $ l $ wynika, że 

\begin{equation}\label{eq5}
	l[g_n] \leq c[U]p(g_n - g) + \frac{|g(\tilde{x}) - g(\tilde{y})|}{|\tilde{x}-\tilde{y}|} \leq l[g] + \epsilon\,.
\end{equation}	
	
Jeśli więc $ l[g] = 0 $, to oczywiście $ |l[g_n] - l[g]| < \epsilon$. Niech teraz $ l[g] >0 $. Dostosujmy $ \epsilon $ tak, aby zachodziło $ 2\epsilon < l[g] $. Wówczas, posługując się odwrotną nierównością trójkąta, wyprowadzamy dolne ograniczenie na przyrost $ l $: dla dowolnego $ n \geq N $ i $ x \not= y $ zawartych w $ \overline{U} $

\begin{equation*}
\begin{split}
	\frac{|g_n (x) - g_n (y)|}{|x-y|} &\geq \Bigl | 	\frac{|g(x) - g(y)|}{|x-y|}	 - \frac{|[g_n-g] (x) - [g_n-g] (y)|}{|x-y|}  \Bigl | \geq \\ 
\geq & l[g] - c[U]p(g_n - g) > l[g] - \frac{\epsilon}{2}\,.\\
\end{split}	
\end{equation*}	
	
Biorąc infimum po prawej stronie powyższego ciągu oznaczeń uzyskujemy 	
	
\begin{equation}\label{eq6}
	l[g_n] \geq l[g] - \frac{\epsilon}{2}\,.
\end{equation}	
	
Z równań \ref{eq5} oraz \ref{eq6} wynika żądana ciągłość $ l $.
	

\qed
\newline

%%%%%%%%%%%%%%%%%%%%%%%%%%%%%%%%%%%%%%%%%%%%%%%%%%%%%%%%%%%%%%5

\hrule
\begin{dowod}[$Y_0\cap \mathcal{C}^{m}(K)$ jest otwarty w $\mathcal{T}\Big( \mathcal{C}^{1}(K)\Big)$.]\label{proof_openess_of_Y_0}
\end{dowod}

W dowodzie wystarczy się posłużyć obserwacją, że operacja włożenia $\mathrm{Id}:\mathcal{C}^{m}(K) \rightarrow \mathcal{C}^{1}(K)$ jest ciągła w sensie odpowiednich topologii \citet[][lemat 2.4, str. 460]{Lanza2}. Ale ponieważ przeciwobraz dowolnego podzbioru $A \subset \mathcal{C}^{1}(K)$ przy $\mathrm{Id}$ dany jest przez 

\begin{equation*}
	\mathrm{Id}^{-1}[A] = A \cap \mathcal{C}^{m}(K)\,.
\end{equation*}

	Zatem z ciągłości $\mathrm{Id}$ wnioskujemy, że $\{ U \cap\mathcal{C}^{m}(K) : U \in \mathcal{C}^{1}(K) \} \subset \mathcal{T}\Big(\mathcal{C}^{m}(K)\Big)$, co zgodnie z oznaczeniami z rozdziału \ref{preliminaries} zapisujemy tak

\begin{equation*}
 	\mathcal{T}\Big(\mathcal{C}^{1}(K)\Big)\restriction \mathcal{C}^{m}(K) \subset	\mathcal{T}\Big(\mathcal{C}^{m}(K)\Big) \,. 
\end{equation*}

Tym samym zbiór $Y_0$ otwarty w $\mathcal{T}\Big(\mathcal{C}^{1}(K)\Big)$ po przecięciu z $\mathcal{C}^{m}(K)$ jest otwarty w $\mathcal{T}\Big(\mathcal{C}^{m}(K)\Big)$, czyli

\begin{equation*}
	Y_0 \cap \mathcal{C}^{m}(K) \in   \mathcal{T}\Big(\mathcal{C}^{m}(K)\Big).
\end{equation*}



\qed
\newline

%%%%%%%%%%%%%%%%%%%%%%%%%%%%%%%%%%%%%%%%%%%%%%%%%%%%%%%%%%%%%%%%%%%%%%%%%%%5


\hrule
\begin{dowod}[Niech $m, q \in \mathbb{N}$ oraz $q > 0$. Niech $\mathcal{J}: \mathcal{C}^{m+q}(\overline{U}_1, \overline{U}_2) \cap \mathcal{I} \rightarrow \mathcal{C}^{q}(\overline{U}_2,\overline{U}_1)$ będzie dany przez $\mathcal{J}\brackets{g} = g^{(-1)}$. Niech $g_0 \in \mathcal{C}^{m+q}(\overline{U}_1, \overline{U}_2)\cap \mathcal{I}$. Wówczas istnieje otoczenie $\mathcal{W}_{g_0}$ punktu $g_0$ w przestrzeni $\mathcal{C}^{m+q}(\overline{U}_1, \overline{U}_2)$ oraz jednoznacznie wyznaczony operator $\hat{\mathcal{J}}: \mathcal{W}_{g_0} \rightarrow \mathcal{C}^{q}(\overline{U}_2, \mathbb{R})$ klasy $\mathcal{C}^{q}$, który jest rozszerzeniem $J$ na rozpatrywanym zbiorze $\mathcal{W}_{g_0}$, czyli zachodzi $\hat{\mathcal{J}}\brackets{g} = \mathcal{J} \brackets{g} \underset{g \in \mathcal{W}_{g_0} \cap \mathcal{I}}{\forall}$. Różniczka operatora $\hat{\mathcal{J}}$ w dowolnym punkcie $g \in \mathcal{W}_{g_0} \cap \mathcal{I}$ dana jest formułą $	\mathrm{d}\, \hat{\mathcal{J}}\brackets{g}(h) \equiv - \brackets{\mathrm{d}\, g \circ g^{-1}}^{(-1)} \cdot \, h \circ g^{(-1)}$ dla dowolnego $h$ ze zbioru $\mathcal{C}^{m+q}(\overline{U}_1, \mathbb{R})$.]\label{proof_macho_geek_ze_mnie} 	
\end{dowod}

	Dowód w przypadku dla $q > 0$ został przeprowadzony w pracy Lanza de Cristoforis (\citeyear[][twr. 5.9., str 477]{Lanza2})~\footnote{Lanza de Cristoforis skupił się na analizie zagadnienia różniczkowalności operatora odwracania funkcji dla funkcji z przestrzeni funkcji Schaudera, $\mathcal{C}^{k,\alpha}(K)$. Jest to przestrzeń funkcji $k$-krotnie różniczkowalnych, z $k$-tą pochodną będącą funkcją h\"olderowską stopnia $\alpha\in(0,1]$. Następnie ogranicza się do badania pewnej podprzestrzeni tej przestrzeni funkcyjnej. W lemacie 2.25. pokazuje on jednak, kiedy dla $\alpha =1 $ rozpatrywana przestrzeń jest identyczna z przestrzenią funkcji $k+1$ razy różniczkowalnych. Okazuje się, że ten przypadek koresponduje z sytuacją, którą jesteśmy zainteresowani. Ponieważ autor rozpatruje przypadek funkcji określonych na $\mathbb{R}^{n}$, w pracy dużo miejsca poświęconego jest możliwości ciągłego rozszerzenia operatora określonego na pewnym podzbiorze otwartym $U \subset \mathbb{R}^n$ na kulę o środku w $0$ i odpowiednio dużym promieniu. Naszego przypadku owe dywagacje nie dotyczą, gdyż rozpatrujemy przypadek funkcji określonych na otwartym przedziale na prostej rzeczywistej, mianowicie na zbiorze $K^{\circ}$. Nie chcąc wchodzić w szczegóły skomplikowanej skądinąd pracy, zachęcamy czytelnika do przekonania się, że dowolny lemat z owej pracy jest prawdziwy, jeśli tylko zamienić znak $\mathcal{C}^{k,1}(K)$ na $\mathcal{C}^{k+1}(K)$. Wynika to rozumowania przedstawionego w tej właśnie pracy. Ponieważ jednak w pracy nie pojawiła się nigdzie przestrzeń funkcji ciągłych z normą supremum, twierdzenie dla tego przypadku nadal pozostaje nieudowodnione i w niniejszej pracy zostanie to uczynione.}. Nie został tam jednak przedstawiony dowód w przypadku $q=0$, który właśnie zamierzamy udowodnić. Przypadek ten jest bowiem potrzebny do skonkludowania, że operator $\mathcal{J}_2$ jest różniczkowalny.

	W dowodzie posłużymy się lematem Omega~\footnote{Patrz: lemat \ref{First diff}}, twierdzeniem Graves'a o funkcji uwikłanej w przestrzeniach Banacha \citep[][twr. VIII.3.1., str. 215]{Maurin} oraz rozumowaniem przedstawionym w pracy Lanza de Cristoforis (\citeyear[][twr. 5.9., str 477]{Lanza2}), które znamienicie upraszcza się.  Nie rozpatrujemy bowiem abstrakcyjnych przestrzeni Schauder'a funkcji, których pochodne odpowiedniego stopnia są funkcjami h\"olderowskimi, określonych na obszarze w $\mathbb{R}^{n}$, tylko przypadek jednowymiarowy -- funkcji określonych na zwartym odcinku $K$. 

	Cytowana przez nas postać lematu Omega wymaga jednak uogólnienia:

\begin{tw}[Uogólniony Lemat Omega]\label{true_omega}
 
	Niech $ M $ to zwarta przestrzeń topologiczna, a $ \mathbb{E} $ oraz $ \mathbb{F} $ to przestrzenie Banacha. Niech również $ U $ jest otwarty w topologii na $ \mathbb{E} $. Wówczas jeśli $ g: U \rightarrow F $ jest klasy $ \mathcal{C}^{m}(U) $, gdzie $ m >0 $, to przekształcenie $ \Omega_g : \mathcal{C}^{m}(M, U) \rightarrow \mathcal{C}^{0}(M, \mathbb{F}) $ dane wzorem $ \Omega_g (f) = g \circ f $ jest klasy $ \mathcal{C}^{m} $. W dziedzienie i przeciwdziedzinie rozpatrujemy odpowiednio normy $||\circ||_{\mathcal{C}^{m}}$ i $||\circ||_{\infty}$. 
	
Wzór na różniczkę operatora $ \Omega_g $ w punkcie $ f $ na funkcji $ h $ dany jest przez:
\begin{equation*}
 [\diff{\Omega_g}(f)\cdot\,h](x) \equiv \diff{g}(f(x))h(x)\,.
\end{equation*}
\end{tw} 	

Dowód twierdzenia jest zasadniczo taki sam, jak oryginalnego twierdzenia, patrz: \citet{Mardsen}. Autorzy sugerują wykorzystanie w dowodzie innej normy, niż wykorzystywana przez nas suma norm supremum odpowiednich pochodnych: proponują wykorzystanie maksimum z wartości norm supremum odpowiednich pochodnych. Jednakże obie normy zadają równoważną topologię na rozpatrywanej przestrzeni. Dowód jest bardzo prostym ćwiczeniem: szczegóły można znaleźć w książce Prusa i Stachury (\citeyear[][ćw. 2.B.33., str. 28]{Prus}). 

Naśladując teraz rozumowanie Lanza de Cristoforis zdefiniujmy operator określony na produkcie przestrzeni $\Psi: \Bigl[\mathcal{C}^{m}(K) \cap Y_0 \Bigl]\times \mathcal{C}^{0}(K) \rightarrow \mathcal{C}^0 (K)$, zadany wzorem

\begin{equation*}
	\Psi[g,f] = g \circ f - \mathrm{Id}\,.
\end{equation*}

Zmierzając w stronę wykorzystania twierdzenia o funkcji uwikłanej zauważmy, że równanie $\Psi = 0$ jest spełnione wtedy, gdy $f = g^{(-1)} = \mathcal{J}[g]$. Operator $\Psi$ jest różniczkowalny ze względu na obie zmienne, a obie pochodne cząstkowe są ciagłe. Jest to prawdą, bowiem pochodną względem drugiego argumentu potrafimy obliczyć dzięki uogólnionemu lematowi Omega:

\begin{equation*}
	\pardiff{\Psi}{f}(g_0,f_0) \cdot\, h = \diff{\Omega_{g_0}}(f_0) \cdot\, h =  \Bigl[g_0'\circ f_0 \Bigl] h\,.
\end{equation*}

Tutaj $h \in \mathcal{C}^{0}(K)$. Owo wyrażenie jest ciągłe w normie $|| \circ ||_{\mathcal{C}^{m}} + || \circ ||_{\infty}$, indukującej topologię na produkcie $\mathcal{C}^{m}(K) \times \mathcal{C}^{0}(K)$. Aby to zobaczyć rozważmy dowolnie małą kulę w $\mathcal{C}^{m}(K) \times \mathcal{C}^{0}(K)$. Wówczas dla dowolnych $(g_0, f_0)$ i $(g_1, f_1)$ z tej kulki zachodzi:

\begin{align*}
	| g_0' \circ f_0 - g_1'\circ f_1 | &\leq |g_0' \circ f_0 - g_1' \circ f_0| + |g_1' \circ f_0 - g_1' \circ f_1| \,.
\end{align*}

	Ponieważ kulka jest mała, więc funkcje $g_0'$ i $g_1'$ są bliskie w normie supremum, stąd pierwszy element prawej strony nierówności jest mały, gdyż obraz $f_0$ zawiera się w dziedzinie obu funkcji $g_0'$ i $g_1'$ . Ponieważ $g_1'$ jest ciągła, zatem drugi element nierówności będzie mały, o ile funkcje $f_0'$ i $f_1'$ będą bliskie	w normie supremum, nad czym panujemy, bowiem możemy wybrać wielkość kuli. Wynika stąd, że $||g_0' \circ f_0 - g_1'\circ f_1  ||_{\infty}$ może być dowolnie małe, więc $\pardiff{\Psi}{f}(g_0,f_0) $ jest ciągłe. Zauważmy również, że $\pardiff{\Psi}{f}(g_0,f_0) $ jest liniowym izomorfizmem przestrzeni $\mathcal{C}^{0}$ na siebie. Jest to konsekwencją tego, że liniowe przekształcenie $g_0': \mathbb{R} \rightarrow \mathbb{R}$ jest izomorfizmem, bowiem $g_0 \in Y_0$.
	
 
Aby policzyć pochodną cząstkową względem pierwszej zmiennej wystarczy z kolei zauważyć, że względem tego argumentu funkcja $\Psi$ jest liniowa, co wynika z definicji dodawania funkcji w algebrze funkcji: $\Psi[\alpha g_1 + \beta g_2, f] = (\alpha g_1 + \beta g_2) \circ f = \alpha \Psi[g_1, f] + \beta \Psi[g_2,f]$. Tym samym, powołując się na argument z Dowodu \ref{proof_nonautonomous_nemytskii2} stwierdzamy, że 

\begin{equation*}
	\pardiff{\Psi}{g}[g_0,f_0] \cdot\, h = \Psi[h, f_0] = h \circ f_0\,,
\end{equation*}

gdzie $h \in \mathcal{C}^{m}$. Powyższe wyrażenie jest ciągłe w normie $|| \circ ||_{\mathcal{C}^{m}} + || \circ ||_{\infty}$. Żeby to zauważyć skorzystamy z wielokrotnie już przywoływanego wzoru Lagrange'a przy obliczaniu różnicy normy operatorów liniowych $\Psi[\circ, f]$~\footnote{Definicja normy operatora liniowego, patrz: \ref{preliminaries}}:

\begin{align*}
\begin{split}
	\Big|\Big|\Big| \pardiff{\Psi}{g}(g_0, f_0) - \pardiff{\Psi}{g}(g_1, f_1)\Big|\Big|\Big|_{\infty} &= \underset{||h||_{\mathcal{C}^{m}} = 1}{\sup} \Big|\Big| h \circ f_0 - h \circ f_1 \Big|\Big|_{\infty} \leq \\
\Big|\Big| f_0 - f_1\Big|\Big|_{\infty} \underset{||h||_{\mathcal{C}^{m}}=1}{\sup}  \Big|\Big| \int_0^1 h' \circ \omega \,\mathrm{d}\, \tau  \Big|\Big|_{\infty} 
&=  \Big|\Big| f_0 - f_1\Big|\Big|_{\infty} \cdot \underset{||h||_{\mathcal{C}^{m}}=1}{\sup} \underset{x\in K}{\sup}\Big| \int_0^1 h' \circ \omega \,\mathrm{d}\, \tau\Big| \leq\\
\Big|\Big| f_0 - f_1\Big|\Big|_{\infty} \cdot \underset{||h||_{\mathcal{C}^{m}}=1}{\sup} \int_0^1 \Big|\Big|h' \Big|\Big|_{\infty}\,\mathrm{d}\, \tau 
&\leq \Big|\Big| f_0 - f_1\Big|\Big|_{\infty} \cdot \underset{||h||_{\mathcal{C}^{m}}=1}{\sup}\Big|\Big|h \Big|\Big|_{\mathcal{C}^{m}} = \Big|\Big| f_0 - f_1\Big|\Big|_{\infty} \,.
\end{split}
\end{align*}

W powyższym wzorze funkcja $\omega$ to po prostu kombinacja wypukła funkcji $f_0$ i $f_1$, która oczywiście przyjmuje wartości w dziedzinie $h$.

Skoro obie pochodne cząstkowe są ciągłe, to operator posiada różniczkę w sensie Fr\' echet dobrze określoną na produkcie $\mathcal{C}^{m} \cap Y_0 \times \mathcal{C}^{0}$, patrz: \citet[][lem. 2.4.12 , str. 78]{Mardsen}~\footnote{Jest to dobrze znane twierdzenie orzekające, kiedy pochodne w sensie G\^ateau i Fr\' echet wyznaczają to samo przybliżenie liniowe danego operatora.}. 

Ponieważ istnieje punkt $f_0 = g_0^{(-1)}$ taki, że $\Psi[g_0,f_0]=0$ oraz przekształcenie $\Psi$ jest klasy $\mathcal{C}^1$ na pewnym otwartym podzbiorze $\mathcal{U}$ produktu $\mathcal{C}^{m} \cap Y_0 \times \mathcal{C}^{0}$ zawierającym parę $(g_0, g_0^{(-1)})$ oraz ponieważ przekształcenie $\pardiff{\Psi}{f}[g_0,g_0^{(-1)}]$ jest odwracalne~\footnote{Bowiem $g_0\in Y_0 $}, zatem na mocy twierdzenia o funkcji uwikłanej \citep[][twr. VIII.3.2, str. 216]{Maurin} zachodzi:

\begin{enumerate}
	\item{Operator $\Psi$ generuje w pewnym otoczeniu punktu $g_0$ odwzorowanie uwikłane $\Xi$, ciągłe na tym otoczeniu. Dokładniej: zbiór $\mathcal{U}$ zawiera w sobie kostkę $B(g_0, r_1) \times B(g_0^{(-1)}) \subset \mathcal{U} \subset \Big[\mathcal{C}^{m}(K) \cap Y_0 \Big]\times \mathcal{C}^{0}(K)$, gdzie przez $B$ oznaczamy kule wokół pewnego punktu o zadanym promieniu, oraz $\Xi: B(g_0,r_1) \rightarrow B(g_0^{(-1)}, r_2)$. }
	\item{Spełniona jest zależność $\Psi[g, \Xi[g]] = 0$ dla $g \in B(g_0,r_1) $, czyli graf $\Xi$ jest podzbiorem przeciworazu $0$ operatora $\Psi$.}
	\item{Odwzorowanie $\Xi$ jest klasy $\mathcal{C}^1$. Jego różniczka w $g_0$ ewaluowana na pewnej funkcji $h \in \mathcal{C}^{m}(K,\mathbb{R})$ dana jest przez
$$ \mathrm{d}\,\Xi (g_0)\cdot h = - \Bigl[ \pardiff{\Psi}{f}[g_0,g_0^{(-1)}]\Bigl]^{(-1)}\cdot \pardiff{\Psi}{g}[g_),g_0^{(-1)}] \cdot h = - \Big[g_0'\circ g_0^{(-1)}\Big]^{(-1)} \cdot \Big( h \circ g_0^{(-1)}\Big).$$	
	}
\end{enumerate}

	Zauważmy, że funkcja $\Xi$ jest poszukiwanym przez nas jednoznacznie wyznaczonym rozszerzeniem funkcji $\mathcal{J}$. Zatem przyjmujemy, że $\hat{\mathcal{J}} \equiv \Xi$. Podobnie przyjmujemy, że $\mathcal{W}_{g_0} \equiv B(g_0, r_1)$.
	
	 Ponieważ $\mathcal{J} = \hat{\mathcal{J}}\restriction_{\mathcal{I}}$, oraz $\hat{\mathcal{J}}$ jest ciągłe na również na topologii podprzestrzeni indukowanej przez $\mathcal{I}$, zatem potrafimy znaleźć jednoznacznie wyznaczoną różniczkę $\mathcal{J}$, która jest równa różniczce $\hat{\mathcal{J}}$ ewaluowanej w punktach $\mathcal{W}_{g_0} \cap \mathcal{I}$. To kończy dowód i stanowi dopełnienie twierdzenia z pracy \citet{Lanza2} przy redukcji do przypadku przestrzeni funkcji różniczkowalnych.
\qed
\newline

%%%%%%%%%%%%%%%%%%%%%%%%%%%%%%%%%%%%%%%%%%%%%%%%%%%%%%%%%%%%%%%%%%%%%%%%%%%%%5

\hrule
\begin{dowod}[Nieautonomiczny operator Niemyckiego $\Upsilon: \mathcal{C}^{1}(K) \rightarrow \mathcal{C}^{1}(K)$, $\Upsilon\brackets{m}(x)=\frac{m(x)}{R(x)}$ jest różniczkowalny w normie $|| \circ||_{\mathcal{C}^{1}}$. Różniczka operatora ewaluowana w punkcie $k \in \mathcal{C}^{1}(K)$ dana jest wzorem $ \mathrm{d}\, \Upsilon\brackets{k}\equiv \Upsilon $]\label{proof_nonautonomous_nemytskii2}
\end{dowod}

	Próba udowodnienia odpowiednika lematu Omega (patrz: Twr. \ref{First diff}) na przypadek nieautonomicznego operatora Niemyckiego zapewne nie nastręczałaby większych trudności: wystarczyłoby naśladować rozumowanie przedstawione w \citep{Mardsen}. Niemniej występujący w naszym problemie operator $\Upsilon$ jest wyjątkowo prostej formy i dowód jego różniczkowalności jest natychmiastowy.

	Zauważmy bowiem, że $\Upsilon$ jest liniowy. Wynika stąd, że jego różniczka w dowolnym punkcie to po prostu ten sam operator. Żeby to pokazać weźmy dowolny opetor liniowy. Jeśli jest różniczkowalny, to zgodnie z definicją różniczkowalności w sensie Fr\' echet zachodzi
	
\begin{equation}\label{eq_10}
	\underset{|| h|| \rightarrow 0}{\lim} \frac{||A[m+h]-A[m]-L_m(h)||}{||h||} = 0\,.
\end{equation}	
	
Ale z liniowości	
	
\begin{equation}\label{eq_11}
	A[m+h]-A[m]-L_m(h) = A[m] + A[h] - A[m] - L_m (h) = A[h] - L_m(h)\,.
\end{equation}
	
Korzystając z jednoznaczności wyznaczenia różniczki~\footnote{O czym wspominaliśmy w rozdziale \ref{chap_diff}.} wystarczy zgadnąć postać $L_m$ czyniącą zadość równaniu \ref{eq_10}. Rozważając równanie \ref{eq_11} zauważamy, że $L_m \equiv A$ czyni zadość stawianym mu wymaganiom.

Wystarczy więc dowieść ciągłości tego operatora w rozpatrywanej przez nas normie $|| \circ||_{\mathcal{C}^{1}}$. Niech więc $|| m - n||_{\mathcal{C}^{1}} < \delta$. Wówczas zachodzi

\begin{equation*}
	|| \Upsilon[m] - \Upsilon[n] ||_{\mathcal{C}^{1}} = \Bigl|\Bigl| \frac{m-n}{R}\Bigl|\Bigl|_{\infty}	+ \Bigl|\Bigl|\Big(\frac{m-n}{R}\Big)'\Bigl|\Bigl|_{\infty} 
\leq 2\Bigl \{  2\underset{x \in K}{\inf}|\frac{1}{R}| +  \underset{x \in K}{\inf}|\frac{R'}{R^2}| \Bigl\} || m - n||_{\mathcal{C}^{1}} \,.
\end{equation*}

Ponieważ zaś i $|R|$ i $|R'|$ przyjmują wartości ze skończonego odcinka w $\mathbb{R}_{++}$, więc wyrażenie w nawiasie klamrowym jest mniejsze od nieskończoności. Zatem różniczka operatora $\Upsilon$ jest ciągła.

\qed
\newline

%%%%%%%%%%%%%%%%%%%%%%%%%%%%%%%%%%%%%%%%%%%%%%%%%%%%%%%%%%%%%%%%%%%%%%%%%%%5
\hrule
\begin{dowod}[Operator $\overline{C}: \mathcal{C}(K) \rightarrow \mathcal{C}(K)$ jest klasy $C^1$, jeśli funkcja $F^{(-1)}$ jest dwukrotnie różniczkowalna na $\mathbb{R}_{+} \cup (-\epsilon, 0\rbracket$]\label{proof_diff_of_C_barred_in_C_0}   
\end{dowod}

Powołajmy się na twierdzenie z pracy Appell'a i Zabreiki (\citeyear[][twr. 6.7, str. 126]{Appell})

\begin{tw}\label{C_0_diff}

	Niech $G$ to nieautonomiczny operator Niemyckiego indukowany przez funkcję $g: K \times \mathbb{R}$, czyli $G[h](y) = g(y, h(y))$. Załóżmy że istnieje i jest ciągła funkcja $g_2: K \times \mathbb{R}$ -- pochodna $g$ względem drugiego parametru. Wówczas zdanie
	
\begin{equation}\label{eq7}
	\underset{s \in K, |u| \leq r}{\sup} |g(s,x(s)+u) - g(s, x(s)) - g_2 (s,x(s))u| = o(r), r \rightarrow 0 
\end{equation}
	
jest równoważne różniczkowalności operatora $G$	w punkcie $x \in C[K]$. Różniczka operatora $ G $ przybliżanego w punkcie $ k \in \mathcal{C}(K) $, ewaluowana w punkcie $ w $ dana jest przez

\begin{equation*}
	\Bigl[\mathrm{d}\, G[k](w)\Bigl] (y) \equiv g_2 \bigl(y, k(y)\bigl)w(y)\,.
\end{equation*}
\qed
\end{tw}

W równaniu \ref{eq7} posłużyliśmy się tak zwaną notacją Landau'a małego $o$. Należy to rozumieć tak:

\begin{equation*}
 f(x) = o(g(x)), x \rightarrow x^* \iff \underset{x \rightarrow x^*}{\lim} \frac{f(x)}{g(x)} = 0\,.
\end{equation*}

Zwróćmy uwagę, że interesujący nas operator $\overline{C}$ jest indukowany przez funkcję $g: K \times K \rightarrow \mathbb{R}_{+}$ daną przez $g(y, h) = F^{(-1)}\Bigl( H(y) + h \Bigl)$. Zatem zmienna $h$ nie przyjmuje docelowo wartości z $\mathbb{R}$, tak jak w powyższym twierdzeniu. Na mocy założeń o funkcji $F$ wiemy jednak, że funkcja $F^{(-1)}$ jest różniczkowalna na $\mathbb{R}_{+}$. Nadto, ponieważ we wzorze \ref{eq7} interesuje nas zachowanie się wyrażenia na bardzo małych kulach, możemy albo przyjąć, że przedłużamy funkcję $F^{(-1)}$ na pewien przedział $(-\epsilon,0]$ w sposób taki, że przedłużenie zachowuje stopień gładkości tej funkcji, albo założymy, że rozpatrujemy topologię podprzestrzeni zawężającą kule z topologii euklidesowej na prostej do ich śladów na półprostej $\mathbb{R}_{+}$. Oba podejścia są dobre i zgodne z dowodem twierdzenia \ref{C_0_diff}, choć bardziej eleganckie jest przedłużenie $F^{(-1)}$. 

	Możemy teraz dowieść twierdzenia, zachowując nawet spory margines ogólności. W naszym przypadku bowiem $g_2 = (F^{(-1)})' \Bigl( H(y) + h \Bigl)$ jest, na mocy założenia, funkcją klasy $\mathcal{C}^{1}$. Możemy więc dowieść naszego twierdzenia pokazując w ogólności, że dwukrotnie różniczkowalna względem drugiego argumentu funkcja indukująca operator Niemyckiego spełnia warunek \ref{eq7}. W tym celu wystarczy dwukrotnie skorzystać z wzoru Lagrange'a. Drugą pochodną $g$ względem drugiej zmiennej oznaczamy poniżej przez $g_{22}$. Wówczas zachodzi

\begin{align*}
\begin{split}
|g(y, h(y)+u) - g(y,h(y)) - g_2 (y,h(y))u | &= 
\Bigl|\int _0^{1} \Bigl[ g_2(s, h(y)+ \tau u)- g_2(y,h(y))\Bigl]u\, \mathrm{d} \tau \Bigl| \leq\\
\leq |u| \int _0^{1} \Bigl| g_2(s, h(y)+ \tau u)- g_2(y,h(y))\Bigl| \mathrm{d} \tau &\leq 
r \int _0^{1}  \int _0^{1} \Bigl| g_{22}(s, h(y)+ \mu \tau u)\Bigl| \tau | u | \,\mathrm{d}\mu \, \mathrm{d} \tau \leq\\
\leq r^2 \int _0^{1}  \int _0^{1} || g_{22} || \tau \, \mathrm{d}\mu \, \mathrm{d} \tau &= 
r^2 \int _0^{1}  || g_{22} || \tau \, \mathrm{d}\tau  = \\
= r^2 \frac{|| g_{22} ||}{2} &= r^2 Z \,,
\end{split}	
\end{align*}	

gdzie $||g_{22} ||$ dane jest przez

\begin{equation*}
	||g_{22} || = \underset{s \in K, h \in K \cup [-\delta,0]}{\max} |g_{22}(s,h)|, 0 < \delta < \epsilon \,.
\end{equation*}

Ponieważ z $r$ będziemy zmierzać do zera, zatem od pewnego miejsca przedziały o średnicy $2r$ będą się zawierały w zbiorze $K \cup [-\delta,0]$, co oznacza, że faktycznie szacujemy wyrażenie podcałkowe z góry. Powyższe oszacowanie nie zależy ani od $s$ ani od $u$, oraz zachodzi oczywiście

\begin{equation*}
	\underset{r \rightarrow 0}{\lim} \frac{r^2 Z}{r} = 0\,.
\end{equation*}

Skoro więc dominanta pierwotnego wyrażenia jest klasy $o(r)$, zatem i pierwotne jest klasy $o(r)$. Oznacza to, że spełnione są założenia twierdzenia \ref{C_0_diff}, czyli operator $\overline{C}: (K) \rightarrow \mathcal{C}(K)$ jest klasy $C^1$.
\qed
\newline




%%%%%%%%%%%%%%%%%%%%%%%%%%%%%%%%%%%%%%%%%%%%%%%%%%%%%%%%%%%%%%%%%%%%%%%%%%%5

\hrule


\listoffigures

%\listoftables

\bibliographystyle{ecca_pl}
\bibliography{myrefs}

%\includepdf{lastpage.pdf}
%%%%%%%%%%%%%%%%%%%%%%%%%%%%%%%%%%%%%%%%%%%%%%%%%%%%%%%%%%%%%%%%%%%%%%%%%%%%%%%%%%%%%%%%%%%%%%%%%%%%%%%%%%%%%%% 




\end{document}




%%%%%%%%%%%%%%%%%%%%%%%%%%%%%%%%%%%%%%%%%%%%%%%%%%%%%%%%%%%%%%%%%%%%%%%%%%%5
\chapter{Trash}
%%%%%%%%%%%%%%%%%%%%%%%%%%%%%%%%%%%%%%%%%%%%%%%%%%%%%%%%%%%%%%%%%%%%%%%%%%%%
\section{Przestrzenie Schaudera i przestrzenie funkcji różniczkowalnych}\label{Schauder}

Rozpatrzym teraz przypadek różniczkowalności operatora $ C[h] = F^{(-1)}\Bigl\{ H + \bigl [\frac{h^{(-1)}}{R}\bigl ]^{(-1)} \Bigl\}$. Nie zdefiniowaliśmy jeszcze jego dziedziny: moglibyśmy użyć, jak w poprzednim przypadku, przestrzeni $ \mathcal{C}^{0,1}(K) $. Jednakże podwójnie zastosowana operacja inwersji jest z technicznych względów kłopotliwa: nie znamy\footnote{Podstawowym problemem jest to, że operacja superponowania funkcji wymaga różniczkowalności operatora działania z lewej strony w algebrze funkcji. A to właśnie operację superponowania wykorzystuje się do dowodu różniczkowalności operatora odwracania funkcji.} twierdzeń, które mogłyby zapewnić różniczkowalność operacji inwersji w taki sposób, żeby operacja ta była klasy $ \mathcal{C}^{r} $ z przestrzeni $ (\mathcal{C}^{0,1}(K), || \circ ||_{\mathcal{C}^{r,1}}) $ w siebie. W bardzo skąpej literaturze poświęconej temu zagadnieniu\footnote{Patrz: \citet{Lanza1, Lanza2}} udowadnia się, że możemy zyskać jeden stopień różniczkowalności inwersji poświęcając jeden stopień różniczkowalności zadający normę w obrazie. Dla przykładu: inwersja będzie raz ciągle różniczkowalna z przestrzeni wyposażonej w normę $ || \circ ||_{\mathcal{C}^{r,1}} $ w przestrzeń wyposażoną w normę $ || \circ ||_{\mathcal{C}^{r-1,1}} $, gdzie $ r > 0 $. To podejście wydaje nam się bardzo nienaturalne w kontekście rozpatrywania różniczkowalności operatora w punkcie stałym, wyznaczanym iteracyjnie. Nadto, po prostu łatwiej jest ustalić uwagę na jednej normie. 


\section*{Blae}


Metodą stopniowej komplikacji zagadnienia zajmijmy się najpierw zagadnieniem różniczkowalności powyższego operatora dla elementów $ h $ z przestrzeni funkcji $ \mathcal{C}^{0,1}(K, \mathbb{R}) $. 

Zajmijmy się na chwilę dowolnym operatorem Niemyckiego $ N : \mathcal{C}^{0,1}(K, \mathbb{R}) \rightarrow \mathcal{C}^{0,1}(K, \mathbb{R}) $ postaci $ N(h)(y) = g(y, h(y))$, gdzie $ g: K \times \mathbb{R} $. Opierając się na pracy Rity Nugari (\citeyear{Nugari1}), rozpatrzmy dwa zestawy warunków nakładanych na postać $ g $

\begin{ass}
Niech $ g = g(y, h) $ 
\begin{enumerate}
	\item{\label{cond H} jest ciągła na $ K \times \mathbb{R} $ i Lipschitza z $y$\\ oraz jednostajnie ciągła z $ h $ na każdym zwartym podzbiorze $ \mathbb{R} $}
	\item{\label{cond K} jest Lipschitza z $ y $, jednostajnie ciągła z $ h $ na każdym zwartym podzbiorze $ \mathbb{R} $, \\ lokalnie Lipschitza z $ h $, globalnie Lipschitza z $ x $ na $ K $  }
\end{enumerate}
\end{ass}

Powyższe warunki pozwalają wysłowić następujące~\footnote{Dowód: \citet[][str. 94]{Nugari1}}

\begin{tw}[O różniczkowalności nieautonomicznego~\footnote{Nomenklatura pochodzi z równań różniczkowych: $ w' =g(t,u(t)) $ to równanie nieautonomiczne, $ w'=g(u(t)) $ to równanie autonomiczne} operatora Niemyckiego]\label{Diff Nugari}

	Niech $ g = g(y, h) $ będzie podwójnie różniczkowalna ze względu na $ h $ oraz załóżmy, że $ g $ i $ \pardiff{f}{u}$ spełniają zestaw założeń \ref{cond K}, oraz $ \frac{\partial^{2} g }{\partial h^2} $ spełnia zestaw założeń \ref{cond H}, to $ N: \mathcal{C}^{0,1}(K) \rightarrow \mathcal{C}^{0,1}(K) $ jest klasy $ C^{1} $ w normie $ ||g||_{\mathcal{C}^{0,1}}$. 
	
Różniczka operatora $ N $ przybliżanego w punkcie $ k \in \mathcal{C}^{0,1}(K) $, ewaluowana w punkcie $ w $ dana jest przez

$$ \Bigl[\mathrm{d}\, N[k](w)\Bigl] (y) \equiv \pardiff{g}{h}\bigl(y, k(y)\bigl)w(y)   $$
\end{tw}

Niech więc $ g(y,h) \equiv F^{(-1)} \Bigl \{ H(y) + h \Bigl \}$. Z zestawu założeń \ref{ass on prod} wynika tylko, że funkcja $ F $ jest klasy $ \mathcal{C}^{1}(\mathbb{R}_{+})$. To mało: musimy dodatkowo założyć conajmniej, że $ F $ jest dwukrotnie różniczkowalna oraz $ F' $ przyjmuje założenia wartości ze skończonego przedziału leżącego w $ \mathbb{R}_{++} $. Wówczas oczywiście $ F^{(-1)} $ jest dwukrotnie różniczkowalna. Dla wygody, załóżmy jednak, że $ F^{(-1)} $ jest klasy $ \mathcal{C}^{3}(\mathbb{R}) $ : oprócz odpowiedniej gładkości funkcji zakładamy również, że potrafimy ową funkcję rozszerzyć na ujemną część osi rzeczywistej: po prostu konstruujemy tam $ F^{(-1)} $ tak, żeby była nadal funkcją ściśle rosnącą, oraz żeby była klasy $ \mathcal{C}^{3}$. Rozszerzenie takie można zrealizować np. poprzez przedłużenie $ F^{(-1)} $ wielomianem Taylora wyznaczonym dla $ f $ w zerze. Prześledźmy dokładnie założenie o istnieniu trzeciej pochodnej. Wiadomo, że $ (F^{(-1)})' = \frac{1}{F' \circ F^{(-1)}} $. Zatem też

$$ (F^{(-1)})'' = \frac{-F''\circ F^{(-1)}}{\Big\{ F' \circ F^{(-1)}\Big\}^{3}}  $$ 

Oraz

\begin{equation}\label{ass on F inversed}
(F^{(-1)})''' = \frac{3 \Big[ F'' \circ F^{(-1)}\Big]^2 - \Big[ F''' \circ F^{(-1)} \Big] \Big[ F' \circ 	F^{(-1)}\Big] }{\Big\{ F' \circ F^{(-1)}\Big\}^5} 
\end{equation}
Tym samym wyrażenie \ref{ass on F inversed} musi być ciągłe i skończone na prostej rzeczywistej.

Wprowadźmy oznaczenie $ \tilde{F} \equiv F^{(-1)}$. Wówczas też

\begin{equation*}
\begin{split}
\pardiff{g}{y} &= \tilde{F}'(H(y)+h)H'(y) \\
\pardiff{g}{h} &= \tilde{F}'(H(y)+h)  	\\
\frac{\partial^{2} g }{\partial h^2} &= \tilde{F}''(H(y)+h) \\
\end{split}
\end{equation*}

Tym samym jeśli $ H'(y) $ będzie Lipschitza, to również $ \pardiff{g}{y} $ będzie Lipschitza, bowiem zbiór $ \mathcal{C}^{0,1} $ z algebraicznym mnożeniem jest algebrą Banacha~\footnote{Czyli mnożenie nie wyprowadza poza zbiór, na którym jest określone. Patrz: \citet{Goebel}. Definicja algebry Banacha -- patrz: \citet[][str. 263]{Rudin}}. Oczywiście ciągła różniczkowalność funkcji pociąga za sobą własność Lipschitza na zwartych przedziałach~\footnote{Co wykorzystaliśmy m. in. przy wyprowadzaniu równania \ref{lipschitzness of M}}. Nadto, jeśli założymy dodatkowo, że funkcja użyteczności jest klasy $ \mathcal{C}^{3}(\mathbb{R}_{++}) $, to różniczkując wzór \ref{H'} na $ H' $ otrzymamy

\begin{equation}\label{H''}
	H''(\omega) = \frac{2\omega u''(u'^{(-1)}(\frac{1}{\omega})) - \frac{u'''(u'^{(-1)}(\frac{1}{\omega})}{u''(u'^{(-1)}(\frac{1}{\omega})}}{\Bigl[ \omega^2 u''(u'^{(-1)}(\frac{1}{\omega}) \Bigl]^2} \equiv \frac{N}{D}
\end{equation}

Podobnie jak przy sprawdzaniu założeń nakładanych na $ H $ przy okazji stosowania doń Lematu Omega (patrz: Twierdzenie \ref{First diff}), interesuje nas możliwość określenia $ H''(0^{+}) $ tak, by zachować ciągłość tej funkcji. Znowu: punkt zero jest punktem granicznym i przyjmujemy, że dla liczb ujemnych możemy w dowolny sposób ciągle przedłużyć tę funkcję - wykonalność tej procedury jest oczywista. 

Skoro asymptotycznie wokół zera $ u(\omega) \approx M \log \Bigl[ \frac{\omega}{C} + 1 \Bigl] $, zatem przyjmijmy, że również i wyższe pochodne można przybliżać w podobny sposób. Wówczas

\begin{align*}
\begin{split}
	u'(\omega) 					&\approx \frac{M}{\omega + C} 		\\
	u'^{(-1)}(\frac{1}{\omega}) &\approx M\omega - C 				\\
	u''(\omega ) 				&\approx \frac{-M}{(\omega + C)^2} 	\\
	u''\Bigl( u'^{(-1)}(\frac{1}{\omega})\Bigl) 2\omega &\approx  \frac{-2}{M \omega} 	\\
	u'''(\omega) 				&\approx \frac{2 M}{(\omega + C)^3}	\\
	u'''\Bigl( u'^{(-1)}(\frac{1}{\omega})\Bigl) &\approx \frac{2M}{(M\omega)^3}		\\
\end{split}
\end{align*}

Składając wszystkie powyższe elementy widzimy, że licznik $ N $ w wyrażeniu \ref{H''} asymptotycznie zachowuje się następująco

$$ L \approx \frac{-2}{M\omega} - \frac{\frac{2M}{(M\omega)^3}}{\frac{-M}{(M\omega)^2}} = \frac{-2}{M\omega} + \frac{2}{M\omega} = 0 $$ 

Mianownik $ D $ w wyrażeniu \ref{H''} asymptotycznie zachowuje się jak $ M^2 $. Tym samym widzimy, że zakładając, że asymptotycznie w okolicy zera funkcja użyteczności do trzeciej pochodnej łącznie zachowuje się, jak wyznaczona przez nas asymptotyczna funkcja, to $ H'' $ w sposób ciągły rozszerza się na wartość w zerze i $ H''(0^{+}) = 0 $.
Spawdziliśmy tym samym to, że $ H' $ jest Lipschitza. Z poczynionych założeń o różniczkowalności $ F^{(-1)} $ i $ H $ wynika więc, że $ f(y,h) = F^{(-1)} \Bigl \{ H(y) + h \Bigl \}$ spełnia zestaw założeń twierdzenia \ref{Diff Nugari}. Podsumowując:

\begin{tw}
	Jeśli trzy pochodne funkcji użyteczności zachowują się asymptotycznie w zerze tak jak odpowiadające im pochodne asymptoty $ M \log \Bigl[ \frac{\omega}{C} + 1 \Bigl]  $, oraz wyrażenie \ref{ass on F inversed} opisujące $ \Bigl(F^{(-1)}\Bigl)''' $ jest ciągłą funkcją na $ \mathbb{R} $, to operator $ \overline{C}: \mathcal{C}^{0,1}(K) \rightarrow \mathcal{C}^{0,1}(K)   $ dany przez $ \overline{C}[h](y) \equiv F^{(-1)} \Bigl \{ H(y) + h(y) \Bigl \} $ jest klasy $ \mathcal{C}^{1} $. Pochodna operatora $ \overline{C} $ w punkcie $ h \in \mathcal{C}^{0,1}(K)   $ ewaluowana na $ g \in \mathcal{C}^{0,1}(K) $ dana jest wzorem:
	
\begin{equation}
\Bigl[\mathrm{d}\, \overline{C}[k](g)\Bigl] (y) \equiv \frac{g(y)}{F'\Bigl( F^{(-1)}\bigl( H(y) + k(y)\bigl) \Bigl)} 
\end{equation}	
\end{tw}



%%%%%%%%%%%%%%%%%%%%%%%%%%%%%%%%%%%%%%%%%%%%%%%%%%%%

\section*{Klocek 4}

\begin{wniosek}[Kryterium lokalnej jednoznaczności równowag]
	Niech $\mathcal{Z} \in \Big\{\mathcal{C}(K), \mathcal{C}^{0,1}(K)\Big\}$. Niech $\overline{C}: \mathcal{Z} \rightarrow \mathcal{Z}$ to operator postaci $\overline{C}[h] \equiv F^{(-1)}\Bigl(H + h \Bigl)$. Wówczas jeśli $h_{*} \in \mathrm{Fix}(\overline{C})$ oraz różniczka przekształcenia $\overline{C}[h] - h$ w punkcie $h_{*}$ jest liniowym przekształceniem odwracalnym pomiędzy przestrzeniami Banacha $\mathcal{Z}$, to $h_{*}$ jest jedyną równowagą w pewnym otwartym podzbiorze $\mathcal{Z}$.   		
\end{wniosek}

\begin{proof}
	Wystarczy powołać się na twierdzenie o funkcji odwrotnej dla przekształceń różniczkowalnych w sensie Fr\' echeta pomiędzy tą samą przestrzenią Banacha $\mathcal{Z}$, patrz: \citet[][twr. VIII.2.7., str. 213]{Maurin}. Jeśli lokalnie $\overline{C} - Id$, to przekształcenie odwracalne, to jest też lokalnie jednoznaczne na pewnym otwartym podzbiorze $U \subset \mathcal{Z}$, więc istnieje tylko jeden punkt w $U$, dla którego $C[h_{*}] = h_{*}$. Musi to być zatem rozpatrywany przez nas punkt stały.
\end{proof}



%%%%%%%%%%%%%%%%%%%%%%%%%%%%%%%%%%%%%%%%%%%%%%%%%%%%5
Niech więc $ \mathcal{X} \times Y $ to przestrzeń wszystkich zasobów i produktów w gospodarce, przy czym $ \mathcal{X} $ to wyróżnione przez nas czynniki produkcji, a $ Y $ to wytworzone towary. Niech $ \mathcal{X} $ i $ Y $ to przestrzenie Banacha nad ciałem liczb rzeczywistych~\footnote{Dokładna definicja -- zobacz dodatek matematyczny, rozdział \ref{chap_appendice}.}. Wówczas funkcja produkcji to $ f: \mathcal{X} \rightarrow \mathcal{Y} $. Żądania o stopniu jednorodności oznacza spełnienie przez nią zależności

$$ f(\alpha x) = \alpha f(x) $$

Gdzie $ \alpha \in \mathbb{R}$. Przyjmując założenie o różniczkowalności funkcji produkcji, łatwo otrzymujemy tzw. tożsamość Eulera:

$$f(x) =  \partialdiff{\alpha}\alpha {f( x)} = \partialdiff{\alpha}{f( \alpha x)} =  \diff{f( \alpha x)} \circ \diffover{\alpha}{ \alpha x} = \diff{f( \alpha x)} x = \partialdiff{k}{ f(\alpha x) } k +  \partialdiff{l}{ f(\alpha x) }l$$

Gdzie $ x = (k,l) $ to wektor zasobów kapitału i pracy. Przechodząc z $ \alpha $-ą w granicy do jedności otrzymujemy 

$$ f(x) = \partialdiff{k}{ f(x) } k +  \partialdiff{l}{ f(x) }l$$

Prawa strona powyższego równania, w warunkach doskonałej konkurencji, równa się również wydatkom poniesionym przez firmę. 
%%%%%%%%%%%%%%%%%%%%%%%%%%%%%%%%%%%%%%%%%%%%%%%%%%%%%%%

Dodajmy również, że rozwiązanie tego problemu dostarcza kryteriów lokalnej jednoznaczności poszukiwanych równowag w przestrzeniach funkcyjnych nie pokrywających się z przestrzeniami zdefiniowanymi przez Colemana $\mathcal{C}_F (K)$ i $\mathcal{M}$~\footnote{Pamiętajmy, że po odrzuceniu egzogenicznych szoków, definicje owych zbiorów takie, jak można je znaleźć w rozdziale \ref{chap1sec1} należy uprości o usunięcie parametru losowego $s$.}, a w pewnym sensie dalece ogólniejszych.